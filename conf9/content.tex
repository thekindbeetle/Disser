\section*{Introduction}

One of the current trends of GIS is transport logistics, which should take into account the topology of the terrain, road network and other delivery options \cite{eng_Markelov_2015}. Also effective GIS routing can reduce emissions from vehicles in the atmosphere, and therefore the risk of environmental impact, regarded as one of the problems of geological and related sciences \cite{eng_Zhukov_2013}.

Transport logistics is the system of delivery, or the movement of any material objects or substances from one point to another using the best route. Transport logistics is part of the distribution, courier services, geological prospecting and other transportation systems that include freightage. An important step in the process of transport logistics is a so-called phase "last mile" - the last stage of delivery from the distribution center to customers. This stage is the least effective of the entire supply chain, and can cost up to 28\% of the cost of delivery. \cite{Scott_2009}

\section*{Methods and Theory}

Planning and monitoring of the delivery scheduling process are very important for transport logistics during the last mile route, because the quality of the route can reduce transport costs and monitoring increases the level of service in reactions to the unplanned situation.

Unfortunately, planning in most companies occurs inefficiently today, because logistics decide which vehicle will carry the load, but do not create a specific route, leaving this decision to the drivers. This is primarily due to the fact that logistics develop planned routes manually without involving automated systems. The second reason, which follows from the first is in that logistics cannot guarantee the principal feasibility of routes as they oriented more towards mass-dimensional parameters, and takes time requirements into account only partially. This is because the mass and dimensions parameters can be calculated in total regardless of the order of delivery points detour but timing may be checked only for a specific route. The construction and calculation of such route goes beyond human capabilities without the use of technology. Since the set of points of delivery for each vehicle is not guaranteed feasible, then calculating routes for individual vehicles do not make sense – it is needed to solve the problem in general, for all points of delivery and all vehicles, which is belonging to the significantly more complex class of problems – Vehicle Routing Problem (VRP).

Implementation of GIS to automatically calculation of the routes brings several significant advantages. First, it is guaranteed the principal feasibility of routes, excluding extraordinary situations. Second, it increases the level of service because of having a planned arrival time in the specific route, we can inform customers, reducing they latency. For example, if a customer ordered a delivery from 15:00 to 18:00, he will not be three hours of "sitting on chair" awaiting delivery, at this time he will stay in place of deliver only, but will take care of his business, maybe will distract on something or will depart for a short time. If we notify the customer estimated time of delivery (from 17:00 to 17:30), we will provide customers more freedom of action and reduce the likelihood that at the time of arrival of delivery, the customer will not be able to accept it in time, leading to delays.

Thirdly, the presence of the planned route will enhance monitoring and responses to unplanned situations. Without a planned route, using GPS monitoring is only possible to see where vehicle was and where it stopped. But if the driver moves plan is unknown, it is not clear whatever the stopping means: the service of delivery point, or driver postponed this point of delivery for later and just stop near it, for example, waiting for a traffic light. Also, it is impossible to predict whether vehicle has time to serve all points of delivery in time – we can see only the fact that some of the points have not visited yet, and customer orders time already left. Using the planned route, we may predict an estimated arrival time at each of next points at any time, considering a possible lag and see if it leads to violation of customer`s time windows in future. It is much easier to take timely action with this information – to inform the driver about the need to speed up or discuss with the client the opportunity to change delivery time.

We shouldn't reject also the possible economy by improving the efficiency of the planned route after the introduction of automated scheduling. However, experience shows that a driver who knows the territory entrusted to him, plans a route on a sufficiently high level. Sometimes the level of experienced driver’s planning is even better than automated solutions because driver has more information about the map of the area. However, automated planning can unlink quality of route from human factors – the level of expertise of each individual driver.

Exact solution of the vehicle routing problem is not possible for sizes of the tasks which modern courier services are facing in urban metropolitan areas, not even all heuristic algorithms can meet the modern needs for calculation speed. Depending on the specific business processes of the company and the organization of accounting and control errors, the final information about current orders is obtained only after arrival at the distribution point and time for planning, loading and sending courier is very small. So, the time calculating a standard problem in the 2-3 thousand points of delivery must be not more than 30-40 minutes.

The main problem of implementation of systems of construction of the planned route in practice is personnel resistance for innovations. Drivers, especially if they are hired carriers, but not employees, refuse to go by software-developed routes. Carriers are not very concerned about global optimality of routes or the level of service for the end customers, if no these parameters are tied with their salaries. They are used to obtain routing formed only on territorial and mass-dimensional criteria, but time limits of the end customers are treated fairly formal. Therefore, any changes in the usual plans are perceived negatively, extremely up to sabotage of the process. Any planned route may be more difficult or personal unappropriated for driver then his own route. Even a perfectly optimal for company solution may ask driver to do some unwanted action: go to the territory of another driver, to deliver some of the points from traditionally other driver’s route if he does not have time, etc. Moreover, the most existing heuristics are oriented only on the total cost, and can make such kind of mistakes in the formation of a separate route, even when there is no urgent necessity.

It was developed a new heuristic algorithm, that takes into account the logistics’ and drivers’ requirements about selecting optimal points for each route, while not rejecting a global optimality at high speed calculation and taking into account the topology of the road network \cite{eng_art1}. 

For further development GIS component must be supplemented by the optimization of voice interaction between dispatcher and driver. Analysis of a large number of voice interaction parameters is allowed to highlight the most important: the amount of time, characteristic of the stop, a fact of completion of delivery.

Practice shows that the amount of time scheduled for service at point of delivery is one of the most defining input parameters that can strongly influence the result of planning and possibility of its implementation in reality. Other parameters defined clearly enough: mass-dimensions parameters are known in advance, allowed time windows are determined by the final customer. There are a lot of already developed tools to determine the time and distance of movement on the road between two points, that can predict the outcome based on statistical data with a sufficient level of error. But there is no reliable source of information to determine the service time in points, neither the client nor the driver or logistics cannot give this info. The most common mistake is to use the same time for all points, for example 10 or 5 minutes, regardless of the weight and the amount of cargo that must be delivered or complexity of the search and the entrance to the point of delivery. The solution of categorization of points or specific orders, and using of fixed service time for all orders in the category is the better option that can be found quite often. The correct option that can bring substantial economy of transport resources in the construction of the planned route is statistical analysis of the history of service time in points. Modeling amount of time required to perform the delivery can be based on the following parameters: how much time is used to service these and similar points in the past, which drivers and vehicles were used, what weight, volume and quantity of cargo was delivered, and other. The choice of the optimal method of modeling deserves a separate study.

However, for this statistical analysis, we need to collect a historical data. Enough accurately stopping time can be determined by analyzing GPS track, but this method has several disadvantages. First, GPS data have some level of error, which may become worse depending on the quality of hardware, or depending on the serviced territory. For example we know that in the areas of high-rise building GPS signal deteriorates significantly and sometimes even lost completely. This deterioration signal can influence adversely the determination of stopping time or even the fact of stopping altogether. Second, even if we discard the GPS error as insignificant or acceptable, there is a problem of linking the stops and points of delivery. In general, this connection became clearly possible if only one stopping and one point of delivery are in a given radius. It is impossible to correlate clearly which one of stoppings needs to take service time in some often happened cases: where the stopping is occurred outside the specified radius, or several stoppings are near the point (including the reasons for false interpretation stoppings because GPS errors), or one stopping is close to several points. And it is impossible to collect and analyze automatically much of the statistical information about the point’s service time based on GPS data in today's metropolitan cities, in a situation when you need to make multiple deliveries in a single building or neighboring buildings with shared courtyard, there often enough. 

Therefore, it is necessary to supplement the GPS more information about time when the delivery driver finished executing one of the points within a single stop and began to perform the next point. Experience shows that attempts to require the driver at this time to get the phone/tablet and select the appropriate command in the mobile app, results in driver’s strategy to marks all points as done before or after the fulfillment of all deliveries at the stop, because the manipulation of the tablet takes time and hands at this moment is usually busy. Thus good interface should not distract driver from his main task. This can be a voice interface that will take command on the start and completion of the delivery, model of which was developed by the author \cite{eng_art2}.

\section*{Conclusions}

The role of GIS in solving logistical problems include the topology of the terrain was considered. The importance of automating voice interaction component between dispatcher and driver was grounded. Key parameters of voice interactions that increase the efficiency of routing was highlighted: the amount of time scheduled for service at the point of delivery, characteristic of the stop, fact of completion of delivery. Prospects of further investigation were outlined.