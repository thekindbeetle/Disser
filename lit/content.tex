\chapter{Слайд}
\setlength{\leftskip}{-3em}\cyrillicfontsf

\begin{enumerate}
	\setlength{\leftskip}{-3em}\cyrillicfontsf
	\item запис фрази вимовленої водієм: $A_i=\langle a_1,a_2,...,a_n\rangle; t=\frac{n}{s}; s=16 \text{ (kHz)};$
	\item перетворення записаної фрази на фонетичний текст, за допомогою фонемного стенографа: \\ $P_i=S(A_i); P=\langle p_1,p_2,...,p_k\rangle; p_i \in F;$
	\item класифікація фонемної репрезентації голосової команди: $y_i=C_c(P_i);$
	\begin{enumerate}
		\item розбиття фонетичного тексту на N-грами фонем різної довжини;
		\item розрахунок інтроформаційного впливу кожного N-граму фонем на можливі команди в вибраному контексті;
		\item вибір команди з найбільшою ймовірністю;
	\end{enumerate}
	\item виконання відповідної реакції (озвучення відповіді, виконання команди та/або відправка структурованих даних диспетчеру);
	\item переключення контексту на новий, відповідний до вибраної реакції: $c_{i+1} = f(c_i, y_i);$
	\item очікування та запис наступної фрази.
\end{enumerate}


Де: $A_i$ --- цифровий аудіозапис команди водія довжиною $t$ секунд,

{\settowidth{\leftskip}{Де:\ }
	
	$n$ --- кількість семплів аудіо сигналу,
	
	$s$ --- частота дискретизації аудіо сигналу,
	
	$P_i$ --- представлення команди водія у вигляді фонемного тексту --- кортежу фонем довжини $k$,
	
	$F$ --- множина фонем української мови,
	
	$S$ --- фонемний стенограф,
	
	$y_i$ --- реакція з моделі голосової взаємодії субʼєктів дистрибуції що відповідає вимовленій команді,
	
	$C_c$ --- класифікатор фонемної репрезентації голосових команд відповідно до поточного контексту $c_i$
	
	$f$ --- функція визначення наступного контексту в залежності від поточного контексту $c_i$ та вибраної реакції $y_i$
	
}

\begin{enumerate}
	\setlength{\leftskip}{-3em}\cyrillicfontsf
	\item Розрахунок визначеності для інтелектуальної системи відносно всіх вхідних N-грам фонем і можливих голосових команд:
	
	\begin{align}
		D_A&=\pm0.5(P_{A}\oslash(J_{1,p}-P_{A}) + (J_{1,p}-P_{A})\oslash P_{A} -2J_{1,p})^{\circ \frac{1}{2}}; \nonumber \\
		D_{AB}&=\pm0.5(P_{AB}\oslash(J_{p,q}-P_{AB}) + (J_{p,q}-P_{AB})\oslash P_{AB}-2J_{p,q})^{\circ \frac{1}{2}}; \nonumber \\
		I_A&=(D_A^{\circ 2}+J_{1,p})^{\circ \frac{1}{2}};\quad I_{AB}=(D_{AB}^{\circ 2}+J_{p,q})^{\circ \frac{1}{2}}, \nonumber
	\end{align}
	
	\item Отримання додаткової визначеності, що є у N-грамів фонем відносно голосових команд:
	
	\[
	D_\Delta=D_{AB} \circ (J_{p,1}I_A)-I_{AB} \circ (J_{p,1}D_A),
	\]
	
	\item Розрахунок сумарного впливу на голосову команду, реакцію інтелектуальної системи всіх наявних N-грамів фонем:
	
	\[
	D_\Sigma = XD_\Delta;\quad I_\Sigma=(D_\Sigma^{\circ 2}+J_{n,q})^{\circ \frac{1}{2}},
	\]
	
	\item Обчислення нової інформованості та визначеності голосової команди:
	
	\[
	D_Y=D_\Sigma \circ (J_{n,1}I_A) - I_\Sigma \circ (J_{n,1}D_A);\quad I_Y=(D_Y^{\circ 2}+J_{n,q})^{\circ \frac{1}{2}},
	\]
	
	\item Обчислення сумісної умовної ймовірності команди $A_i$ (при наявності всіх N-грамів фонем $B_j \in B$):
	
	\[
	Y=P_Y=0.5J_{n,p}+D_Y \oslash 2I_Y,
	\]
	
\end{enumerate}

\chapter{Текст}

\paragraph{Розпізнавання мови}

Класичні дослідження по розпізнанню мовлення показують необхідність великих контекстно-залежних словників а також великих корпусів для навчання системи \cite{Pylypenko_2008,Pylypenko_2009,Pylypenko_2010,Lydovyk_2011,Vasilyeva_2012,Womack_1999,Zirneeva_2008,Gladunov_2005}, однак акустичну модель навчену на великому корпусі можна використовувати для розпізнання фонем у інших задачах, що не потребує словника \cite{Pylypenko_2008,Robeyko_2012,Abdel_2012,Zhang_2017,Sharma_2018,Yermolenko_2008}. 

\paragraph{Speech classification}

У класифікації звуку можна виділити наступні основні категорії: класифікація емоційного забарвлення \cite{Ververidis_2004,Ververidis_2004_2,Ververidis_2006,Kaya_2017,Deb_2016,Bhaskar_2015,Devillers_2015,Weisskirchen_2017,Sharma_2018,Ozseven_2018}, визначення диктора \cite{Rabotyagov_2006} чи його особливостей (вік, стать) \cite{Kaya_2017}, виділення проміжків мовлення \cite{Hussain_2018,Khonglah_2016,Benatan_2015,Priya_2012} або інших класів звукового змісту (музика, фонових шумів, тощо) \cite{Hussain_2018,Boddapati_2017} у аудіозаписах, класифікація мовленевих актів за метою вислову \cite{Ko_2015,Choi_1999,Grosz_1995,En_2005,Kang_2013,Kim_2012,Lee_2002,Sridhar_2008,Webb_2005,Kang_2010,Hellbernd_2016,Su_2017}.

В деяких роботах аналіз може відбуватись за попередьньо отриманим текстом \cite{Ko_2015,Choi_1999,Grosz_1995,En_2005,Kang_2013,Kim_2012,Lee_2002,Webb_2005,Kang_2010}, в деяких виключно за звуковими характеристиками \cite{Ververidis_2004,Ververidis_2004_2,Ververidis_2006,Kaya_2017,Deb_2016,Devillers_2015,Weisskirchen_2017,Sharma_2018,Hussain_2018,Khonglah_2016,Benatan_2015,Priya_2012,Boddapati_2017,Hellbernd_2016}, а в деяких звук і текст використовуются разом \cite{Bhaskar_2015,Sridhar_2008,Chowdhury_2018}.

Важливе значення виділення мовлення з фонового шуму проявляєтся у задачі підсилення голосу для людей з погіршеним слухом. Зокрема це може бути реалізовано у дадатку на смартфоні \cite{Vashkevich_2018}.

\paragraph{Control instruction}

Більшість систем розпізнання та розуміння управляючих команд зосереджені в сфері робототехніки. У переважній більшісті підходів, розпізнання керуючих команд відбуваєтся за текстовим їх представленням, наприклад після автоматичного розпізнання голосу. Деякі підходи зосереджені на спробах побудувати достатньо складну але чітку граматику команд \cite{Misra_2016,Fasola_2013,Eppe_2016}, деякі гнучко класифікують заздалегідь заданий перелік команд \cite{Yongda_2018}, а найпростіші просто шукають задіні ключові слова в тексті \cite{Shwe_2003,Shulika_2018}. Існують роботи в яких віжбуваєтся класифікація голосових команд без переводу в текст, але перелік команд лише обмежуєтся декількома пунктами \cite{Shulika_2018,Gryshchuk_2006}.

\paragraph{Text processing}

Як можна бачити з прикладів задач класифікації мовленнєвих актів та розпізнання керуючих команд, аналіз змісту сказаного голосом, в переважній більшості випадків частково чи повністю відбуваєтся через аналіз вже розпізнаного тексту сказаного, а значить переходить із задачі аналізу голосу, в сферу обробки та розуміння природної мови. Існує велике різноманіття методів обробки та класифікації природно-мовних текстів \cite{Mironczuk_2018,Altinel_2018,Hartmann_2018,Kim_2014,Britz_2015_2,Britz_2015,Marchenko_2005}.

\paragraph{Phoneme usage}

Використання фонемного тексту заміть лексичного для роботи зі звуком було добре досліджено у задачі пошуку по звуковим файлам \cite{Ng_C_2000,Ng_K_2000,Witbrock_1997}. Результати показали що попри незначне зменшення якості пошуку, воно дає переваги в вирішені проблеми слів поза словником та може мати переваги в використанні на мобільних пристроях. Крім того були дослідження по автомоатичному визначенню та класифікації одиниць мовлення \cite{Sharma_2018}.

Підходи роботи з текстовими даними посимвольно \cite{Kim_2016,Zhang_2015_2,Zhang_2015,Santos_2014} можуть бути використані для пофонемної робои з фонетичним текстом.

\paragraph{CNN direct}

В деяких роботах ЗНН використовувалися для роботи зі звуком напряму чи зі спектрограмою звуку, в задачах класифікації емоційного забарвлення \cite{Weisskirchen_2017}, класифікації звуків навколишнього середовища \cite{Boddapati_2017}, Визначення мети одночасного мовлення \cite{Chowdhury_2018}, розпізнання послідовності фонем \cite{Abdel_2012,Zhang_2017}, та інших.

\paragraph{Spoken dialogue systems}

Окремо розглядаюся методи побудови голосових діалогових систем \cite{Iosif_2018}. Можна виділити голосові діалогові системи на основі правил \cite{Herbert_2018,Lopes_2015} та на основі сценаріїв \cite{Lopez_2016,Khouzaimi_2018}. Існують спіціальні метеди опису та синхронізації багатомодальних діалогових систем з використанням голосових даних \cite{Katsurada_2003,Karpov_2012}. 

\paragraph{Distribution}

Проблема управління ланцюгом поставок є надзвичайно актуальною в дистрибуції та логістиці \cite{Speranza_2018}, особливо питання стійкості \cite{Koberg_2018,Jia_2018,Bastas_2019,Wen_2018,Sullet_2018} та екологічності \cite{Tseng_2019,Papetti_2019,Hoehne_2017}. Особливо важливим є етап «останньої милі» \cite{Baldi_2018,Gdowska_2018,Boysen_2018,Hoehne_2017,Pronello_2017,Cook_2017} та вплив інтернет магазинів на ці процеси \cite{Allen_2018,Cardenas_2017}.

Для покращення процесів доставки використовується велика кількість різноманітних технологій, таких як RFID\cite{Prasanna_2012}, GPS \cite{Stopher_2018,Prasanna_2012} та GSM \cite{Prasanna_2012} трекери, принципи «інтернету речей» \cite{Liu_2018} та «big data» \cite{Govindan_2018}, використання додатків на смартфонах \cite{Stopher_2018}, web-системи управління ланцюгом поставок \cite{Papetti_2019} та інщі.


\chapter{Розпізнавання мови}


\section{Other}

\subsection{N-channel hidden Markov models for combined stressed speech classification and recognition}

Тут мабуть більше про розпізнання а не про класифікацію

\url{Womack_1999.pdf}\cite{Womack_1999}

\subsection{Sparse coding based features for speech units classification}

0. Phoneme classification, consonant-vowel classification

1. Корпус та/або предметна область

2. raw speech samples and mel frequency cepstral coefficients (MFCC)

3. continuous density hidden Markov model
(CDHMM)

4. Выделение speech units из речи, по сути может рассмытриватся как фонемний стенограф

In recent years sparse coding based signal processing has been applied to various speech processing applications
such as audio classification (Zubair et al., 2013), speaker verification (Haris and Sinha, 2012), speech
enhancement (Abrol et al., 2013; Low et al., 2013), speech recognition (Sivaram et al., 2010), speech
separation (Xu et al., 2013), speaker tracking (Barnard et al., 2014) and speech coding (Giacobello et al., 2010).
In sparse representations (SR) of speech signal, a speech frame is written as a linear combination of atoms of a
resource, known as dictionary. The sparse vector obtained for each speech frame, given a dictionary, is used as
a feature. The behavior of sparse vector is very much influenced by the choice of dictionary, which could be
either analytical or learnt. Analytical dictionaries are easy to implement and have fast transform properties. The
learnt dictionaries are derived from the data itself and thus adapt to the variations in data effectively (Tosic and
Frossard, 2011).

\url{Sharma_2018.pdf}\cite{Sharma_2018}

\subsection{Інформаційна технологія обробки та аналізу характеристик мовленнєвої інформації}

\url{Zirneeva_2008.pdf}\cite{Zirneeva_2008}

\subsection{Застосування вейвлет-аналізу для попередньої обробки мовних голосових сигналів в задачах сегментації, класифікації та пофонемного розпізнавання}

\url{Yermolenko_2008.pdf}\cite{Yermolenko_2008}

\subsection{Апаратно-програмні засоби роздільної локалізації фонем в системах мовної взаємодії людини з ЕОМ}

\url{Gladunov_2005.pdf}\cite{Gladunov_2005}

\section{Пилипенко і ко.}

\subsection{Автоматизированный стенограф украинской речи}

\url{Pylypenko_2008.pdf}\cite{Pylypenko_2008}

\subsection{Распознавание ключевых слов в потоке речи при помощи фонетического стенографа}

\url{Pylypenko_2009.pdf}\cite{Pylypenko_2009}

\subsection{Аннотация и учет речевых сбоев в задаче автоматического распознавания спонтанной украинской речи}

\url{Pylypenko_2010.pdf}\cite{Pylypenko_2010}

\subsection{Автоматическое распознавание спонтанной украинской речи (на материале акустического корпуса украинской эфирной речи)}

\url{Lydovyk_2011.pdf}\cite{Lydovyk_2011}

\subsection{Корпус украинской эфирной речи}

\url{Speech_technology_2_2012.pdf#page=12}\cite{Vasilyeva_2012}

\subsection{Преобразование между орфографическим и фонемным текстами для моделирования спонтанного произношения}

\url{Speech_technology_2_2012.pdf#page=33}\cite{Robeyko_2012}

\subsection{Див. також:}

\url{http://speechtechnology.ru}

\url{https://scholar.google.com.ua/citations?user=LOfEszUAAAAJ&hl=ru}

\url{http://dspace.nbuv.gov.ua/browse?value=Пилипенко, В.В.&type=author}

\section{CNN}

\subsection{Applying convolutional neural networks concepts to hybrid NN-HMM model for speech recognition}

0. Розпізнання звуку (послідовності фонем)

1. TIMIT

2. Частоти звуку

3. CNN + HMM (одна пара згортка+агрегація, 3 повнозвязних шари, ПММ для фінального розпізнання фонем)

\url{Abdel_2012.pdf}\cite{Abdel_2012}

\subsection{Towards end-to-end speech recognition with deep convolutional neural networks}

0. Розпізнання звуку (послідовності фонем)

1. TIMIT

2. Частоти звуку

3. Глибока CNN (згортка, агрегація, 3 згортки, 9 повнозвязних)

Кращий результат ніж в \cite{Abdel_2012}

\url{Zhang_2017.pdf}\cite{Zhang_2017}

\chapter{Speech classification}

\section{Emotional}

\subsection{Automatic emotional speech classification}

0. Emotional speech classification

1. Danish Emotional Speech database. five emotional states such as anger, happiness, neutral, sadness, and surprise

2.  Features only. 87 features has been calculated. Sequential Forward Selection
method (SFS) has been used in order to discover a set of 5
to 10 features which are able to classify the utterances in the best
way

3. nearest mean
and Bayes classifier

\url{Ververidis_2004.pdf}\cite{Ververidis_2004}

\textbf{Automatic speech classification to five emotional states based on gender information}

\url{Ververidis_2004_2.pdf}\cite{Ververidis_2004_2}

\subsection{Emotional speech recognition: Resources, features, and methods}

0. Emotional speech classification

1. багото різних

2. Features only.  Typical features are the pitch, the
formants, the vocal tract cross-section areas, the mel-frequency cepstral coefficients, the Teager energy operator-based features,
the intensity of the speech signal, and the speech rate. 

3.  hidden Markov models, artificial neural networks, linear discriminant
analysis, k-nearest neighbors, support vector machines

\url{Ververidis_2006.pdf}\cite{Ververidis_2006}

\subsection{Emotion, age, and gender classification in children’s speech by humans and machines}

0. Emotional speech classification, age and gender recognition

1. first child emotional speech corpus in Russian, called “EmoChildRu”, 3-class

2. Features only. openSMILE feature set

3. least squares regression based classifiers, 

While in age estimation, the acoustics-based automatic systems show higher performance, they do not reach human perception levels in comfort state and gender classification

\url{Kaya_2017.pdf}\cite{Kaya_2017}

\subsection{Classification of speech under stress using harmonic peak to energy ratio}

0. Emotional speech classification

1. simulated stressed speech database (angry, Lombard, neutral, happy and sad)

2. Features only. new feature, harmonic peak to energy ratio

3. Support Vector Machine

\url{Deb_2016.pdf}\cite{Deb_2016}

\subsection{Hybrid approach for emotion classification of audio conversation based on text and speech mining}

0. Emotional speech classification

1. audio-visual emotion database named eNTERFACE'05 EMOTION

2. Sound seatures + word-level text features. sound:  formants, zero crossing rate, sound intensity, fundamental frequency (F0) and  energy; text: emotional tag of each word by WordNet Affect

3. Support Vector Machine

Звук и текст на вход вместе

\url{Bhaskar_2015.pdf}\cite{Bhaskar_2015}

\subsection{Inference of human beings’ emotional states from speech in human--robot interactions}

0. Emotional speech classification

1. Robots. JEMO, ARMEN, IDV-HR (Anger Joy Sad. Neut.)

2. Audio Features only

3. Support Vector Machines

\url{Devillers_2015.pdf}\cite{Devillers_2015}

\subsection{Recognition of emotional speech with convolutional neural networks by means of spectral estimates}

0. Розпізнання емоцій

1. emoDB, eNTERFACE, SUSAS

2. Спектрограма як картинка

3. CNN

\url{Weisskirchen_2017.pdf}\cite{Weisskirchen_2017}

\subsection{SPeech ACoustic (SPAC): A novel tool for speech feature extraction and classification}

0. Розпізнання емоцій

1. EMO-DB, EMOVA, eNTERFACE05 and SAVEE

2. Audio features

3.  SVM, k-NN and MLP

\url{Ozseven_2018.pdf}\cite{Ozseven_2018}

\section{мова/не-мова}

\subsection{SwishNet: A Fast Convolutional Neural Network for Speech, Music and Noise Classification and Segmentation}

0. Speech, Music and Noise classification/segmentation

1. MUSAN

2. Audio Features and Stectrogram. MFCC, (Mel Filter Banks) spectrum

3. CNN

мовлення/музика/шум

\url{Hussain_2018.pdf}\cite{Hussain_2018}

\subsection{Speech/music classification using speech-specific features}

0. Speech/Music classification

1. Корпус та/або предметна область

2. Audio Features. spectral flux, spectral centroid, spectral rolloff, zero crossing rate (ZCR), and percentage of low energy frames

3. non-linear mapping, Gaussian Mixture Models (GMM), Support Vector Machines (SVM)

speech/non-speech

\url{Khonglah_2016.pdf}\cite{Khonglah_2016}

\subsection{Cross-covariance-based features for speech classification in film audio}

0. Speech/non-speech classification

1. film audio

2. Audio Features.  Mel Frequency Cepstral Coefficient

3. random forest classifier

\url{Benatan_2015.pdf}\cite{Benatan_2015}

\subsection{Speech and non-speech identification and classification using KNN algorithm}

0. Speech/non-speech classification

1. 

2. Audio Features

3. K-Nearest Neighbor

\url{Priya_2012.pdf}\cite{Priya_2012}

\subsection{Classifying environmental sounds using image recognition networks}

0. Environmental sounds classification

1. ESC-50, ESC-10 and UrbanSound8K

2. Direct sound. Spectrogram, MFCC, and CRP

3. CNN (AlexNet and GoogLeNet)

\url{Boddapati_2017.pdf}\cite{Boddapati_2017}

\section{speech-act}

Класифікація між функціями мови в діалозі!

\subsection{New feature weighting approaches for speech-act classification}

0. speech-act classification

1. Korean. Reservations (RES) and schedule management (SM-11)

2. No audio. Text features only. Word level, TF/IDF+

3. SVM and k-NN

\url{Ko_2015.pdf}\cite{Ko_2015}

\subsection{Analysis system of speech acts and discourse structures using maximum entropy model}

0. speech-act classification

1. Korean. RES

2. No audio. Text features only. 

3. Maximum Entropy Model

\url{Choi_1999.pdf}\cite{Choi_1999}

\subsection{Discourse and dialogue}

\url{Grosz_1995.pdf#page=216}\cite{Grosz_1995}

\subsection{An analysis of Speech Acts for Korean Using Support Vector Machines}

0. speech-act classification

1. Korean

2. No audio. Text features only. лексична форма слова, відмітки частин мови у біграмах, контекст попередьнього висловлювання

3. Support Vector Machines

\url{En_2005.pdf}\cite{En_2005}

\subsection{Hierarchical speech-act classification for discourse analysis}

0. speech-act classification

1. Korean. Reservations (RES) and schedule management (SM-11)

2. No audio. Text features only. лексична форма слова, відмітки частин мови у біграмах, контекст попередьнього висловлювання

3. Support Vector Machine

4. Дворівнева (іерархічна) класифікація

\url{Kang_2013.pdf}\cite{Kang_2013}

\subsection{An effective application of contextual information using adjacency pairs and a discourse stack for speech-act classification}

0. speech-act classification

1. Korean. Reservations (RES) and schedule management (SM-11)

2. No audio. Text features only. лексична форма слова, відмітки частин мови у біграмах, контекст попередьнього висловлювання

3. Support Vector Machine

\url{Kim_2012.pdf}\cite{Kim_2012}

\subsection{Korean speech act analysis system using hidden markov model with decision trees}

0. speech-act classification

1. Korean

2. No audio. Text features only. 

3. HMM

\url{Lee_2002.pdf}\cite{Lee_2002}

\subsection{Modeling the intonation of discourse segments for improved online dialog act tagging}

0. speech-act classification

1. Switchboard-DAMSL

2. Audio features and ASR text features. Audio: pitch (f0) and the RMS energy

3. Метод класифікації

Використання акустично-просодичних ознак замысть контексту попередьного значення акту

\url{Sridhar_2008.pdf}\cite{Sridhar_2008}

\subsection{Error analysis of dialogue act classification}

0. speech-act classification

1. Switchboard-DAMSL

2. Text features only.

3. 

\url{Webb_2005.pdf}\cite{Webb_2005}

\subsection{A reliable multidomain model for speech act classification}

0. Задача

1. favorite foods, views on love, and favorite music

2. No audio. Text features only. 

3. 

Два етапу - спочатку класифікуємо предметну область, а потім класифікуємо мовленнєвий акт моделлю для предметної області

Без звуку взагалі, тільки текст.

\url{Kang_2010.pdf}\cite{Kang_2010}

\subsection{Local grammars of speech acts: An exploratory study}

Теоретична стаття про мовленеві акти та їх аналіз. Багато посилань на підґрунтя теорій. Можна взяти трохи опису.

\url{Su_2017.pdf}\cite{Su_2017}

\subsection{Prosody conveys speaker’s intentions: Acoustic cues for speech act perception}

0. Задача

1. однослівні та безслівні виразів

2. Audio features. mean intensity, harmonics-to-noise
ratio (HNR), mean fundamental frequency (f0) as well as
pitch rise

3. Discriminant analyses

The term prosody refers to variations in pitch, loudness, timing, or voice quality over the course of an utterance (Warren, 1999) that can modify the communicative content of a message, both linguistically and paralinguistically (Bolinger, 1986)

Опис чому просодія важлива з посиланнями

Дослідження: класифікація однакових однослівних та безслівних виразів з різною інтонацією що б визначити один з 6 намірів. 3 експерименти: моделювання з просодичних показників, опитування слухачей людей і моделювання реакції слухачей-людей.

\url{Hellbernd_2016.pdf}\cite{Hellbernd_2016}


\section{Speech enhancement}

\subsection{Speech enhancement in a smartphone-based hearing aid}

Усиление речи на смартфоне

\url{Speech_technology_2_2018.pdf#page=64}\cite{Vashkevich_2018}

\section{other}

\subsection{Метод ідентифікації людини на основі індивідуального мовного коду}

\url{Rabotyagov_2006.pdf}\cite{Rabotyagov_2006}

\subsection{Automatic Classification of Speech Overlaps: Feature Representation and Algorithms}

0. Класифікація одночасного мовлення на кооперативне та конкурентне.

1. 

2. Порівняння декількох типів ознак як акустичних так і лексичних (вкладення слів, мішок н-грам слів) та методів їх комбінації (лінійна та прихована)

3. Порівняння декількох методів (SVM, FFNN, CNN, RNN (LSTM))

\url{Chowdhury_2018.pdf}\cite{Chowdhury_2018}

\section{Control instruction}

\subsection{Метод розпізнавання команд голосового управління комп’ютерною системою}

Класифікація команд з голосу по узагальненим ознакам звукозипису. Насправді неправда.

Після первинної обробки сигналу формуються
набір векторів ознак, які містять в собі спектральні характеристики: мелчастотні
кепстральні коефіцієнти MFCC, а вони в свою чергу містять
інформацію на який ділянках сигналу присутня мова, та перетворюється в
коефіцієнти, які в подальшому надходять в блок розпізнавання. Далі
відбувається перетворення сигналу у вектори особливостей. На етапі
розпізнавання відбувається зіставлення вхідних з образів з уже наявними
шаблонами. Інши кажучи вектори ознак, порівнюються з моделями, в
результаті чого обчислюється правдоподібність для кожної моделі. 

0. Класифікація команд

1. 4 однослівні команди управління компьютером

2. MFCC зі звуку

3. Порівняння зі зразками методом динамічного програмування

\url{Shulika_2018.pdf}\cite{Shulika_2018}

\subsection{Research on multimodal human-robot interaction based on speech and gesture}

0. Класифікація команд

1. multimodal human-robot interaction

2. Speech->Text Microsoft speech SDK. TF-IDF

3. maximum entropy classification

Похоже на то что нужно

класификация через текст

можно взять теории про составление Control instruction corpus

взаимодействие(!) с роботом

дополнительно используются жесты

traditional natural language understanding used the keyword to match method

\url{Yongda_2018.pdf}\cite{Yongda_2018}

\subsection{Method and apparatus for multiple tiered matching of natural language queries to positions in a text corpus}

keyword to match method

\url{Shwe_2003.pdf}\cite{Shwe_2003}

\subsection{Tell me dave: Context-sensitive grounding of natural language to manipulation instructions}

0. Парсинг команд

1. human-robot interaction

2. Text only

3. Recently, learning methods based on combinatory categorial
grammar (CCG) (Steedman, 1996, 2000) have been
used with success for different applications (Zettlemoyer
and Collins, 2007). Matuszek et al. (2012b) use probabilistic
CCG parsers to convert natural language commands to
a Robot Control Language (subset of typed-lambda calculi)
such as given below.

Выделение простых известных глаголов и объектов из текста и превращение их в функциональные команды

\url{Misra_2016.pdf}\cite{Misra_2016}

\subsection{Modeling dynamic spatial relations with global properties for natural language-based human-robot interaction}

0. Парсинг команд

1. human-robot interaction

2. Text only



\url{Fasola_2013.pdf}\cite{Fasola_2013}

\subsection{Exploiting deep semantics and compositionality of natural language for Human-Robot-Interaction}

0. Парсинг команд

1. human-robot interaction

2. Text only

3. Embodied
Construction Grammar

\url{Eppe_2016.pdf}\cite{Eppe_2016}

\subsection{Моделювання процесу аналізу і класифікації голосових команд}

\url{Gryshchuk_2006.pdf}\cite{Gryshchuk_2006}

\chapter{Text processing}

\subsection{A recent overview of the state-of-the-art elements of text classification}

\url{Mironczuk_2018.pdf}\cite{Mironczuk_2018}

\subsection{Semantic text classification: A survey of past and recent advances}

\url{Altinel_2018.pdf}\cite{Altinel_2018} 

\subsection{Comparing automated text classification methods}

\url{Hartmann_2018.pdf}\cite{Hartmann_2018}

\subsection{Character-Aware neural language models}

0. Передбачення наступного слова

1. Порівняння 7 мов з різню морфологією

2. Символи одного слова

3. CNN + HighwayNetwork + LSTM + Softmax

\url{Kim_2016.pdf}\cite{Kim_2016}

\subsection{Text understanding from scratch}

0. topic classification and Sentiment Analysis

1. News Categorization, Yahoo! Answers Topic Classification, Amazon Review Sentiment Analysis, DBpedia Ontology Classification

2. Символи всього тексту (заголовок і аннотація)

3. Deep CNN

\url{Zhang_2015_2.pdf}\cite{Zhang_2015_2}

\textbf{Character-level Convolutional Networks for Text Classification}

\url{Zhang_2015.pdf}\cite{Zhang_2015}

\subsection{Learning character-level representations for part-of-speech tagging}

0. part-of-speech tagging

1. Wall Street Journal (WSJ) portion of the Penn Treebank

2. для кожного слова Word level embeddings and character level embeddings

3. CNN

\url{Santos_2014.pdf}\cite{Santos_2014}

\subsection{Convolutional Neural Networks for Sentence Classification}

Основа моєї роботи

0. Sentiment Analysis

1. IMDB Review

2. Ревю за словами

3. CNN

\url{Kim_2014.pdf}\cite{Kim_2014} 

\textbf{Implementing a CNN for Text Classification in TensorFlow}

Навчальна стаття за роботою

\url{http://www.wildml.com/2015/12/implementing-a-cnn-for-text-classification-in-tensorflow}\cite{Britz_2015_2}

\textbf{Understanding Convolutional Neural Networks for NLP}

Підготовча навчальна стаття

\url{http://www.wildml.com/2015/11/understanding-convolutional-neural-networks-for-nlp}\cite{Britz_2015}

\subsection{Алгоритми семантичного аналізу природномовних текстів}

\url{Marchenko_2005.pdf}\cite{Marchenko_2005}

\chapter{Phoneme usage}

\section{Document retrival}

\subsection{Experiments in spoken document retrieval using phoneme n-grams}

In this work, different methods for extracting overlapping phonesequence
indexing features for SCR are explored in detail. This article
arrives at the general conclusion that phone-based retrieval is not
as effective as word-based retrieval, but there are certain situations
where it is appropriate. Specifically, phone-based retrieval is effective
for addressing the OOV problem. Further, if speech recognition must
be performed on a platform with limited capacity (i.e., a hand-held
device), then a small language model, such as a phoneme bigram model,
makes the ASR system lightweight and compact. The authors
find that in terms of phone-sequence-based indexing features, a combination
of phone 3-grams and 4-grams proved most effective.
Further, this work shows that ignoring word boundaries when extracting
phone-based features does not affect retrieval performance significantly.

\url{Ng_C_2000.pdf}\cite{Ng_C_2000}

\subsection{Subword-based approaches for spoken document retrieval}

\url{Ng_K_2000.pdf}\cite{Ng_K_2000}

\subsection{Using Words and Phonetic Strings for Efficient Information Retrieval from Imperfectly Transcribed Spoken Documents}

phone-based features derived from
word-level transcripts are able to help compensate for word-level error

\url{Witbrock_1997.pdf}\cite{Witbrock_1997}

\chapter{Distribution}

\section{supply chain management}

\subsection{A systematic review of sustainable supply chain management in global supply chains}

\url{Koberg_2018.pdf}\cite{Koberg_2018}

\subsection{Multi-tier sustainable supply chain management: The role of supply chain leadership}

\url{Jia_2018.pdf}\cite{Jia_2018}

\subsection{Integrated quality and supply chain management business diagnostics for organizational sustainability improvement}

\url{Bastas_2019.pdf}\cite{Bastas_2019}

\subsection{Web-based platform for eco-sustainable supply chain management}

\url{Papetti_2019.pdf}\cite{Papetti_2019}

\subsection{A literature review on green supply chain management: Trends and future challenges}

\url{Tseng_2019.pdf}\cite{Tseng_2019}

\subsection{Fashion Retail Supply Chain Management: A Review of Operational Models}

\url{Wen_2018.pdf}\cite{Wen_2018}

\subsection{Sustainable Logistics and mobility in cities: Paris South-East Area}

\url{Sullet_2018.pdf}\cite{Sullet_2018}

\subsection{Trends in transportation and logistics}

\url{Speranza_2018.pdf}\cite{Speranza_2018}

\section{IoT, GPS}


\subsection{IoT-enabled Dynamic Optimisation for Sustainable Reverse Logistics}

\url{Liu_2018.pdf}\cite{Liu_2018}

\subsection{Big data analytics and application for logistics and supply chain management}

\url{Govindan_2018.pdf}\cite{Govindan_2018}

\subsection{Smartphone app versus GPS Logger: A comparative study}

\url{Stopher_2018.pdf}\cite{Stopher_2018}

\subsection{RFID GPS and GSM based logistics vehicle load balancing and tracking mechanism}

\url{Prasanna_2012.pdf}\cite{Prasanna_2012}

\section{last-mile}

\subsection{A Generalized Bin Packing Problem for parcel delivery in last-mile logistics}

\url{Baldi_2018.pdf}\cite{Baldi_2018}

\subsection{Stochastic last-mile delivery with crowdshipping}

\url{Gdowska_2018.pdf}\cite{Gdowska_2018}

\subsection{Scheduling last-mile deliveries with truck-based autonomous robots}

\url{Boysen_2018.pdf}\cite{Boysen_2018}

\subsection{Understanding the impact of e-commerce on last-mile light goods vehicle activity in urban areas: The case of London}

\url{Allen_2018.pdf}\cite{Allen_2018}

\subsection{E-commerce last-mile in Belgium: Developing an external cost delivery index}

\url{Cardenas_2017.pdf}\cite{Cardenas_2017}

\subsection{Greenhouse gas and air quality effects of auto first-last mile use with transit}

\url{Hoehne_2017.pdf}\cite{Hoehne_2017}

\subsection{Last mile freight distribution and transport operators’ needs: which targets and challenges?}

\url{Pronello_2017.pdf}\cite{Pronello_2017}

\subsection{Dispatching policies for last-mile distribution with stochastic supply and demand}

\url{Cook_2017.pdf}\cite{Cook_2017}


\chapter{spoken dialogue systems}

\subsection{Intelligent Conversation System Using Multiple Classification Ripple down Rules and Conversational Context}

\url{Herbert_2018.pdf}\cite{Herbert_2018}

\subsection{Speech understanding for spoken dialogue systems: From corpus harvesting to grammar rule induction}

\url{Iosif_2018.pdf}\cite{Iosif_2018}

\subsection{A methodology for turn-taking capabilities enhancement in Spoken Dialogue Systems using Reinforcement Learning}

\url{Khouzaimi_2018.pdf}\cite{Khouzaimi_2018} 

\subsection{Automatic creation of scenarios for evaluating spoken dialogue systems via user-simulation}

\url{Lopez_2016.pdf}\cite{Lopez_2016}

\subsection{From rule-based to data-driven lexical entrainment models in spoken dialog systems}

\url{Lopes_2015.pdf}\cite{Lopes_2015}

\subsection{XISL: a language for describing multimodal interaction scenarios}

Мова для опису мультимодальної діалогової комунікації. Передбачена для використання у веб додатках. Дозволяє синхронізувати етапи голосового вводу та текстовго/іншого. Голосовий ввод розпізнаєтся в текст.

\url{Katsurada_2003.pdf}\cite{Katsurada_2003} 

\subsection{Когнитивные исследования ассистивного многомодального интерфейса для бесконтактного человеко-машинного взаимодействия}

Система мультимодального управління компьютером голосом і жестами для інвалідів. Голос розпізнається в текст.

В ассистивном интерфейсе ICanDo, которому
посвящена данная статья, реализованы и применены
программные средства компьютерного зрения
для обнаружения лица человека в оптическом
потоке на основе характерных органов/черт лица
(нос, глаза, губы) без использования искусственных
маркеров/мишеней и специализированных, носимых
человеком, устройств, что выгодно отличает
его от имеющихся аналогов. Программная система
взаимодействия не накладывает дополнительных
ограничений на пользователя и обеспечивает
естественность и комфорт при бесконтактной работе
с компьютером. Применяемые в интерфейсе
голосовые команды, распознаваемые автоматической
системой, являются прекрасной альтернативой
стандартным органам ввода информации (клавиатура)
как для инвалидов без рук или пальцев
рук, так и для обычных пользователей.

\url{Karpov_2012.pdf}\cite{Karpov_2012}

\chapter{характеристики роботи}

0. Задача

1. Корпус та/або предметна область

2. Вхідні ознаки, методи їх отримання, звук/текст

3. Метод класифікації

4. Особливості/інновації

\printbibliography
