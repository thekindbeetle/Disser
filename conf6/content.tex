Дистрибуція — це діяльність, пов'язана з отриманням продукції, її зберіганням до моменту отримання замовлення і наступної доставки до клієнтів. Управління дистрибуцією включає в себе планування, організацію та контроль.

Інформаційні технології в управлінні дистрибуцією вже достатньо розроблені для забезпечення етапів отримання продукції та її збереження. Отже зараз найбільш інтенсивно йде розвиток етапу доставки продукції до кінцевих клієнтів, зокрема розробляються системи автоматизації побудови планових маршрутів руху автотранспорту \cite{art1}, системи керування транспортним парком (TMS) та моніторинг доставок в реальному часі.

Велику роль в управлінні дистрибуцією відіграють процеси голосової взаємодії, які сьогодні активно автоматизуються. Голосова взаємодія поділяється на безпосередню та взаємодію із залученням інформаційних технологій, які, у свою чергу, можуть слугувати лише засобом забезпечення зв’язку або вносити певні елементи автоматизації.

Передові автоконцерни світу, такі, як Ford, BMW, Mercedes, прагнуть підвищити безпеку та комфорт водія, тому створюють можливість керування бортовою електронікою (телефоном, музикою, навігацією) за допомогою голосу \cite{Jonsson_2009}. Крім того, за наявності достатньо потужної системи голосову взаємодію з водієм можна використовувати для підтримання діалогу при русі вночі, аби не дати водію заснути \cite{Kravchenko_2012}.

Для управління доставкою вантажів у дистрибуції необхідний моніторинг руху автомобілів у режимі реального часу. Це дозволяє аналізувати ефективність водія, а також передбачати певні небажані інциденти. Для такого моніторингу використовують GPS дані руху автомобіля. На жаль лише GPS треку автомобіля не достатньо для однозначного розуміння стану справ. Трек дає змогу бачити тільки те, що водій був біля точки доставки, але  з нього  не зрозуміло, чи виконана доставка, чи з якихось причин відмінена. З треку видно, що за поточної швидкості водій відстає від плану та не встигає на наступну точку, але не зрозумілими є причини відставання та чи має водій можливість надолужити втрачений час. Для отримання цієї інформації необхідна додаткова комунікація з диспетчером. Але дзвінок по телефону чи, ще гірше, комунікація через якийсь візуальний інтерфейс в смартфоні потребує зайвого часу та знижує концентрацію уваги водія на дорозі, що може спричинити ДТП. Тому потрібна система, що дасть змогу виявляти необхідну інформацію в голосових даних водія і відправляти її диспетчеру у формалізованому вигляді.

Найбільш подібна до цього система Pick-by-Voice \cite{Pick-to-Voice}, що використовується в іншій сфері управління дистрибуцією – управлінні складськими процесами. Pick-by-Voice дає змогу відбірнику по черзі отримувати голосові команди щодо того де, що і в якій кількості треба відібрати, а також у формі діалогу інформувати у відповідь про необхідність повторити завдання чи перейти до наступного та ін. Така система дає відбірнику можливість звільнити руки та очі та в цілому збільшити його ефективність.

Для управління ж транспортними доставками потрібна більш складна система, ніж Pick-by-Voice, адже вона повинна мати суттєво більший спектр необхідних для розпізнання команд. У передових системах управління міськими доставками вантажів (urban freight distribution) важливим параметром є часові вікна доставки \cite{Quak_2006}. Такий параметр одразу вводить цілу низку додаткової інформації, яку треба передати від водія до диспетчера. Більше того, система повинна забезпечувати взаємодію з диспетчером в режимі реального часу, а не відтворювати заданий заздалегідь перелік задач. 

Таким чином, аналіз наявних інструментів голосової взаємодії в задачах управління дистрибуцією та фіксація задач управління доставкою вантажів підтвердив необхідність розробки системи голосового управління доставкою продукції в дистрибуції.
