\chapter*{Заключение}                       % Заголовок
\addcontentsline{toc}{chapter}{Заключение}  % Добавляем его в оглавление

%% Согласно ГОСТ Р 7.0.11-2011:
%% 5.3.3 В заключении диссертации излагают итоги выполненного исследования, рекомендации, перспективы дальнейшей разработки темы.
%% 9.2.3 В заключении автореферата диссертации излагают итоги данного исследования, рекомендации и перспективы дальнейшей разработки темы.
%% Поэтому имеет смысл сделать эту часть общей и загрузить из одного файла в автореферат и в диссертацию:

Основные результаты работы заключаются в следующем.
%% Согласно ГОСТ Р 7.0.11-2011:
%% 5.3.3 В заключении диссертации излагают итоги выполненного исследования, рекомендации, перспективы дальнейшей разработки темы.
%% 9.2.3 В заключении автореферата диссертации излагают итоги данного исследования, рекомендации и перспективы дальнейшей разработки темы.

\begin{enumerate}
  \item Получены асимптотические формулы периодического решения уравнения Мэки--Гласса для достаточно больших значений коэффициента нелинейности. Показано, что из асимптотических соотношений следует сходимость решения уравнения Мэки--Гласса к решению соответствующего предельного релейного уравнения.
  \item Сформулированы и доказаны достаточные условия существования периодических режимов (а) в виде дискретной бегущей волны, (б) двухкластерной синхронизации в полносвязной цепи релейных генераторов Мэки--Гласса в виде ограничения на параметры соответствующей системы дифференциальных уравнений с запаздыванием. 
  \item Проведено численное моделирование, демонстрирующее устойчивость соответствующих режимов.
\end{enumerate}


В первой главе была построена асимптотика уравнения \eqref{eq:MG_norm} по степеням малого параметра $\gamma^{-\nu}$, где показатель нелинейности $\gamma$ считается большим параметром. Такое предположение оправдано тем, что в оригинальной статье \cite{Mackey1977} приводятся значения (после нормировки на запаздывание) $\alpha$, $\beta$ порядка единицы и $\gamma = 10$. Численное моделирование показывает, что для приведённых значений предельный цикл уравнения \eqref{eq:MG_norm} достаточно близок к циклу релейного уравнения \eqref{eq:MG_rele}. Подобное предположение использовалось в работах \cite{Bartha2021, Krisztin2020}, а также в исследованиях, посвящённых генным сетям (см., например, \cite{Volokitin2004}), где возникает аналогичная нелинейность.

Следует отметить, что схожий подход к построению асимптотики решения применялся в статьях \cite{Kolesov2010, Kolesov1997, Glyzin2013} и других работах тех же авторов.

В работах \cite{Bartha2021, Krisztin2020} также доказывается близость решения уравнения \eqref{eq:MG_norm} и решения его предельной версии при $\gamma \to +\infty$. В статье \cite{Krisztin2020} рассматривается случай близких значений параметров $\alpha$ и $\beta$, а в \cite{Bartha2021} случай большого $\alpha$ по сравнению с $\beta$, а именно случай 
\begin{equation*}
	\label{eq:cond_Bartha}
	\alpha > \max\{\beta e^\beta, e^\beta - e^{-\beta}\}.   
\end{equation*}
%Заметим, что ограничения \eqref{eq:cond_Bartha} и \eqref{} описывают схожие области, однако неравенство \eqref{eq:cond_Bartha} оказывается более сильным. 

Отличие результата, полученного в первой главе, от работы \cite{Bartha2021} заключается в точной асимптотической оценке разности решений уравнений \eqref{eq:MG_x} и \eqref{eq:MG_rele} (пара уравнений аналогична уравнениям, рассмотренным в \cite{Bartha2021, Krisztin2020}, но после экспоненциальной замены), а также в более широкой области параметров, для которой доказана сходимость решения к предельному.

Во второй главе была рассмотрена полносвязная система релейных осцилляторов Мэки--Гласса \eqref{eq:system_full_generators}. После нормировок и замен она преобразовалась в систему \eqref{eq:mg_full_renormed}, для которой была исследована предельная система \eqref{eq:system_relay}. Для этой предельной системы было установлено существование дискретных бегущих волн \eqref{eq:discrete_wave}, где функция $u(t)$ была найдена как периодическое решение уравнения \eqref{eq:mg_relay_1}, причём это решение содержит наименьшее возможное количество переключений (два на периоде). Результаты численного моделирования показали, что решения системы \eqref{eq:mg_full_renormed} близки к решениям предельной системы \eqref{eq:system_relay} при больших значениях $\gamma$, и для различных начальных условий решение системы стремится к режиму дискретной бегущей волны. Это позволяет предположить глобальную устойчивость найденного режима.

В третьей главе было доказано существование периодических режимов двухкластерной синхронизации в полносвязной системе релейных осцилляторов Мэки--Гласса, а также приведены явные формулы таких решений. Результаты численного моделирования показали, что решения системы \eqref{eq:system_main} близки к решениям предельной системы при больших значениях $\gamma$, что даёт основание предполагать устойчивость найденных решений.

% Автор благодарит \dots




