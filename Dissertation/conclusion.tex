\chapter*{Заключение}                       % Заголовок
\addcontentsline{toc}{chapter}{Заключение}  % Добавляем его в оглавление

%% Согласно ГОСТ Р 7.0.11-2011:
%% 5.3.3 В заключении диссертации излагают итоги выполненного исследования, рекомендации, перспективы дальнейшей разработки темы.
%% 9.2.3 В заключении автореферата диссертации излагают итоги данного исследования, рекомендации и перспективы дальнейшей разработки темы.
%% Поэтому имеет смысл сделать эту часть общей и загрузить из одного файла в автореферат и в диссертацию:

Основные результаты работы заключаются в следующем.
У дисертаційній роботі вирішено актуальне наукове завдання розробки моделей і методів формалізації голосової інформації в системах диспетчерського контролю за рухом автотранспорту. Загалом можна зробити наступні висновки.

1. Дослідження теоретико-методологічних засад формалізації голосової інформації в системах дистрибуції показало, що значну роль в їх управлінні відіграють процеси голосової взаємодії особливо стосовно своєчасного коригування планових маршрутів руху автотранспорту. Розроблення моделі голосової взаємодії без блоку переведення звуку голосу в текст може принципово покращити автоматизацію голосової взаємодії в системах контролю дистрибуції.

2. Розроблена система автоматичного розрахунку планових маршрутів та практика її використання забезпечили накопичення параметрів непередбачуваних ситуацій в процесі доставки, що впливають на створення сценаріїв голосової взаємодії, які представляються у вигляді орієнтованого графу та контекстів взаємодії. 
Принципи побудови рефлекторних систем на основі теорії несилової взаємодії адаптовано для формалізації голосової інформації в системах диспетчерського контролю за рухом автотранспорту.

3. Розроблено математичну модель голосової взаємодії водія та диспетчера в системах диспетчерського контролю за рухом автотранспорту, яка представлена у вигляді повного графу сценаріїв усіх етапів дистрибуції «склад – дорога – точка доставки». Виділено перелік унікальних контекстів голосової взаємодії, формалізація голосової інформації в яких може відбуватися незалежно, що дозволяє знизити кількість реакцій для автоматизованого розпізнання.

4. Розроблено метод формалізації голосової інформації в системах підтримки диспетчеризації автотранспорту з використанням інтелектуальних рефлекторних систем, що дозволяє автоматизувати процес передачі голосової інформації з уникненням переводу звукової інформації в лексичний текст за рахунок використання двох основних модулів (автоматичного фонетичного стенографа і ядра рефлекторної системи голосового управління). Для реалізації ядерного компонента запропоновано дуальну систему класифікації голосових команд, яка може використовувати метод інтелектуальних рефлекторних систем або метод згорткових нейронних мереж.

5. Метод структурної ідентифікації згорткових нейронних мереж для класифікації голосових команд адаптовано до розпізнавання фонемного тексту, що дозволяє класифікувати голосові команди без переведення голосу в лексичний текст. 

6. Метод інтелектуальних рефлекторних систем поєднано з теоретичним апаратом теорії нейронних мереж, шо дає можливість оптимізувати значення інформованості та визначеності шляхом навчання методом зворотного розповсюдження помилки.

7. Результати математичного моделювання формалізації голосової інформації показав підвищення ефективності розпізнавання повідомлень у голосовій взаємодії водія з диспетчером, а саме підвищення точності розпізнавання у середньому на 6.6 \% для згорткових нейронних мереж і на 19.1 \% для інтелектуальних рефлекторних систем за рахунок використання моделі голосової взаємодії водія та диспетчера. Крім того використання моделей на основі згорткових нейронних мереж показало підвищення швидкості розпізнавання на 15 \% порівняно з інтелектуальними рефлекторними системами.

8. Результати досліджень впроваджені в ТОВ «УІТ», м. Київ (довідка від 4 січня 2019) та використовувалися у трьох логістичних компаніях-клієнтах протягом року.

9. Мета досліджень щодо підвищення ефективності розпізнавання повідомлень у голосовій взаємодії водія з диспетчером досягнута іта всі часткові завдання вирішені повністю. Наукові результати досліджень є внеском у розвиток наукових і методологічних основ створення та застосування інформаційних технологій та інформаційних систем для автоматизованої переробки інформації й управління.

10. Перспективним шляхом подальших досліджень у зазначеному напрямку може бути широке коло питань щодо розробки та дослідження інших реалізацій фонемного стенографа, використання розроблених методів та моделей класифікації фонемного тексту для роботи з лексичним текстом, а також створення моделей голосової взаємодії у вигляді графу сценаріїв для інших предметних областей.


В первой главе была построена асимптотика уравнения \eqref{eq:MG_norm} по степеням малого параметра $\gamma^{-1}$, где показатель нелинейности $\gamma$ считается большим параметром. Такое предположение оправдано тем, что в оригинальной статье \cite{Mackey1977} приводятся значения (после нормировки на запаздывание) $\alpha$, $\beta$ порядка единицы и $\gamma = 10$. Численное моделирование показывает, что для таких значений предельный цикл уравнения \eqref{eq:MG_norm} достаточно близок к циклу предельного уравнения \eqref{eq:MG_rele} при $\gamma \to +\infty$. Подобное предположение использовалось в работах \cite{Bartha2021, Krisztin2020}, а также в исследованиях, посвящённых генным сетям (см., например, \cite{Volokitin2004}), где возникает аналогичная нелинейность.

Следует отметить, что схожий подход к построению асимптотики решения применялся в статьях \cite{Kolesov2010, Kolesov1997, Glyzin2013} и других работах тех же авторов.

В работах \cite{Bartha2021, Krisztin2020} также доказывается близость решения уравнения \eqref{eq:MG_norm} и решения его предельной версии при $\gamma \to +\infty$. В статье \cite{Krisztin2020} рассматривается случай близких значений параметров $\alpha$ и $\beta$, а в \cite{Bartha2021} случай большого $\alpha$ по сравнению с $\beta$, а именно случай 
\begin{equation*}
	\label{eq:cond_Bartha}
	\alpha > \max\{\beta e^\beta, e^\beta - e^{-\beta}\}.   
\end{equation*}
%Заметим, что ограничения \eqref{eq:cond_Bartha} и \eqref{} описывают схожие области, однако неравенство \eqref{eq:cond_Bartha} оказывается более сильным. 

В работах \cite{Bartha2021, Krisztin2020} доказывается сходимость решения уравнения \eqref{eq:MG_norm} к решению предельного уравнения при $\gamma \to +\infty$ с использованием общего результата, изложенного в \cite[гл. XIV]{Diekmann1995}.
Отличие результата, полученного в первой главе, от работы \cite{Bartha2021} заключается в точной асимптотической оценке разности решений уравнений \eqref{eq:MG_x} и \eqref{eq:MG_rele} (пара уравнений аналогична уравнениям, рассмотренным в \cite{Bartha2021, Krisztin2020}, но после экспоненциальной замены).

Во второй главе была рассмотрена полносвязная система генераторов Мэки--Гласса \eqref{eq:system_full_generators}. После нормировок и замен она преобразовалась в систему \eqref{eq:mg_full_renormed}, для которой была исследована предельная система \eqref{eq:system_relay}. Для этой предельной системы было установлено существование дискретных бегущих волн \eqref{eq:discrete_wave}, где функция $u(t)$ была найдена как периодическое решение уравнения \eqref{eq:mg_relay_1}, причём это решение содержит наименьшее возможное количество переключений. Результаты численного моделирования показали, что решения системы \eqref{eq:mg_full_renormed} близки к решениям предельной системы \eqref{eq:system_relay} при больших значениях $\gamma$, и для различных начальных условий решение системы стремится к режиму дискретной бегущей волны. Это позволяет предположить глобальную устойчивость найденного режима.

В третьей главе было доказано существование периодических режимов двухкластерной синхронизации в полносвязной системе генераторов Мэки--Гласса, а также приведены явные формулы таких решений. Результаты численного моделирования показали, что решения системы \eqref{eq:system_main} близки к решениям предельной системы при больших значениях $\gamma$, что даёт основание предполагать устойчивость найденных решений.

% Автор благодарит \dots




