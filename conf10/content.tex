Аналіз отриманих результатів попередніх етапів дослідження \cite{conf8} дозволив нам побудувати рефлекторну модель \cite{Teslia_2010} голосової взаємодії в задачах управління дистрибуцією. В основу моделі покладено логічні сценарії взаємодії на тему управління дистрибуцією, які мають враховувати параметри основних причин невідповідності реальної ситуації запланованому маршруту, наприклад, запізнення або відмови обслуговування на точці доставки тощо. Це дає змогу отримати інформацію для прийняття рішення про повернення вантажу на склад, про відміну чи відкладення обслуговування однієї точки доставки, щоб мати можливість встигнути на іншу, про зміну маршруту для об’їзду затору або про утворення нового маршруту з резервною машиною тощо.

Звичайно, абсолютно всі причини та параметри не можуть бути враховані заздалегідь, але проробка і врахування основної типології дозволить приймати базові рішення та вдаватися до безпосереднього зв`язку з диспетчером лише у складних випадках, що розвантажить водія та канали комунікації і дасть змогу підвищити загальну ефективність дистрибуції.

Виходячи з наявної логіки побудови маршрутів було розроблено принципові блоки сценаріїв. Першим етапом, на якому можуть виникнути проблеми розбіжності плану та факту, є етап завантаження на складі (якщо, наприклад, буде виявлений неврахований перевантаження або недовантаження машини, будуть відсутні необхідні товари чи працівники складу не встигнуть їх вчасно відібрати, або навіть виявиться, що машина не здатна вийти на маршрут).

Другий етап сценаріїв голосової взаємодії визначають проблеми, які можуть виникнути в дорозі до певної точки доставки, як, наприклад, ремонт дороги по маршруту руху або зміни в правилах руху на деяких вулицях, які ще не відбиті в алгоритмах прокладення маршруту (нові заборони поворотів чи односторонній рух), проблеми з автомобілем на дорозі, які призводять до зниження швидкості або відмови в подальшому русі по маршруту, або найбільш розповсюджена проблема заторів на дорогах. 

Третій етап сценаріїв голосової взаємодії викликаний можливими невідповідностями між планом та фактом в обслуговуванні точки доставки. Це можуть бути як проблеми зі сторони клієнта («нікого немає дома», клієнт не має грошей, клієнт відмовляється від замовлення чи стверджує, що він замовляв щось інше), так і проблеми зі сторони водія (запізнення на точку доставки, пошкодження товару тощо). Найбільш поширеною є ситуація, коли водій проводить в точці доставки більше часу, ніж заплановано, що призводить до проблем на всьому подальшому маршруті \cite{art2}.

Ці та інші інциденти на всіх зазначених етапах потребують вирішення із залученням диспетчера для вибору найкращої стратегії і мінімізації втрат. Відповідно розроблене дерево сценаріїв голосової взаємодії включає всі три етапи та враховує типові відомі проблеми та способи їх розв’язання.

Для представлення дерева сценаріїв найкраще підходить орієнтовний граф, в якому вершини позначають стан системи та діалогові фрази які буде озвучувати система, а ребра --- репліки (стимули) які можуть бути сприйняті системою в кожній конкретній вершині. Реакція на стимул може привести до переходу між станами, отже орієнтоване ребро проводиться від тієї вершини в якій стимул може буди сприйнятий, до тієї, в який стан система перейде в якості реакції на стимул. Отже множина всіх ребер, що виходять з вершини, позначають перелік стимулів, між якими треба проводити розпізнання для стану, що відповідає цей вершині.

Назва «дерево» сценаріїв використовується як сталий вираз, але реально представити всю необхідну інформацію у вигляді дерева неможливо, адже переходи між станами неминуче приводять до утворення циклів, а отже граф для представлення такої інформації підходить краще.

Нажаль, такій схемі недостатньо повноти для представлення всіх можливих варіантів перебігу подій. По перше, реакції системи не обмежуються переключенням станів (контекстів) що позначають доступний перелік стимулів та відтворенням діалогових фраз. По друге, стимули які можуть викликати певні реакції системи і відповідно переходи між станами не обмежуються голосовими репліками сказаними водієм. Це можуть бути певні події які надійшли з інших джерел інформації, наприклад, команда від диспетчера або інформація з внутрішніх джерел даних,  використовуючи яку  можливо автоматизувати перехід між деякими станами, що підвищить зручність користування системою і зменшить кількість необхідних альтернатив для розпізнавання. 

\textbf{Висновки}. Були пророблені можливі сценарії взаємодії диспетчера та водія у випадку різних позапланових ситуацій. Запровадження автоматизованої системи голосової комунікації між водієм та диспетчером дозволяє спростити для водія повідомлення диспетчеру інформації про позапланові події, які потребують втручання, коректування маршруту тощо. Таким чином підвищується швидкість та ефективність отримання диспетчером інформації. Крім того робота водія спрощується за рахунок голосових порад чи вказівок,  що надійшли від диспетчера, або отриманих та обрахованих на самому пристрої у водія.