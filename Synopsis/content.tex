\pdfbookmark{Общая характеристика работы}{characteristic}             % Закладка pdf
\section*{Общая характеристика работы}

\newcommand{\actuality}{\pdfbookmark[1]{Актуальность}{actuality}\textbf{\actualityTXT}}
\newcommand{\progress}{\pdfbookmark[1]{Разработанность темы}{progress}\textbf{\progressTXT}}
\newcommand{\aim}{\pdfbookmark[1]{Цели}{aim}{\textbf\aimTXT}}
\newcommand{\tasks}{\pdfbookmark[1]{Задачи}{tasks}\textbf{\tasksTXT}}
\newcommand{\aimtasks}{\pdfbookmark[1]{Цели и задачи}{aimtasks}\aimtasksTXT}
\newcommand{\novelty}{\pdfbookmark[1]{Научная новизна}{novelty}\textbf{\noveltyTXT}}
\newcommand{\influence}{\pdfbookmark[1]{Практическая значимость}{influence}\textbf{\influenceTXT}}
\newcommand{\methods}{\pdfbookmark[1]{Методология и методы исследования}{methods}\textbf{\methodsTXT}}
\newcommand{\defpositions}{\pdfbookmark[1]{Положения, выносимые на защиту}{defpositions}\textbf{\defpositionsTXT}}
\newcommand{\reliability}{\pdfbookmark[1]{Достоверность}{reliability}\textbf{\reliabilityTXT}}
\newcommand{\probation}{\pdfbookmark[1]{Апробация}{probation}\textbf{\probationTXT}}
\newcommand{\contribution}{\pdfbookmark[1]{Личный вклад}{contribution}\textbf{\contributionTXT}}
\newcommand{\publications}{\pdfbookmark[1]{Публикации}{publications}\textbf{\publicationsTXT}}

\textbf{Актуальність теми дослідження.} 
Системи диспетчерського контролю за рухом автотранспорту, призначені ефективно коригувати відхилення від запланованих маршрутів при зіткненні з непередбачуваними обставинами, потребують ефективного обміну повідомленнями між водієм і диспетчером. Різні форми автоматизації диспетчерського контролю (GPS, додатки з сенсорним інтерфейсом, мобільний інтернет) на сьогодні не здатні замінити голосову взаємодію, в якій диспетчер отримує необхідну для прийняття рішень інформацію зокрема про характер і причини відхилень від плану. 

Таким чином підвищення ефективності передачі повідомлень за рахунок формалізації голосової взаємодій між водієм та диспетчером є одним із перспективних напрямів вдосконалення системи диспетчерського контролю, що робить тему дисертаційного дослідження інформаційних технологій формалізації голосової інформації в системах диспетчерського контролю за рухом автотранспорту \textbf{актуальною}.

%
%.
%
%.
%
%.
%
%Системи диспетчерського контролю за рухом автотранспорту призвані (у тому числі) ефективно коригувати відхилення від запланованих маршрутів при зіткненні з непередбачуваними обставинами. Інформація про обставини має повідомлятися диспетчеру в найкоротші терміни для забезпечення можливості прийняття ефективних рішень.
%
%В існуючих системах диспетчерського контролю, параметри, такі як GPS трек, повідомляюся диспетчеру у формалізованому вигляді із застосуванням мобільного інтернету або супутникового звʼязку. З цих даних диспетчер має можливість бачити, що характеристика руху не відповідає запланованій, але не має інформації щодо причин такої невідповідності.
%
%Ця інформація може бути отримана за рахунок безпосередньої голосової взаємодії за допомогою мобільного телефону, або у формалізованому вигляді через мобільний додаток водія з сенсорним інтерфейсом.
%
%Водії часто уникають використовувати сенсорний додаток для передачі інформації про причини невідповідності руху та плану, через те що це відволікає від безпосередніх завдань керування автомобілем.
%
%Тобто єдиним каналом отримання цієї інформації залишається безпосередня, неформалізована голосова взаємодія. Проте така голосова взаємодія є витратною по часу як для водіїв так і для диспетчера і використання мобільного телефону є проблемою бо також відволікає від водійських функцій.
% 
%Підвищення ефективності передачі повідомлень є актуальним, оскільки неформалізована голосова взаємодія відбувається часто неефективно, а автоматизована часто уникання.
%
%Все це робить тему дисертаційного дослідження інформаційних технологій формалізації голосової інформації в системах диспетчерського контролю за рухом автотранспорту \textbf{актуальною}.
%
%.
%
%.
%
%.
%
%Системи диспетччерського контрою за рухом автотранспорту призначені для підвищення ефективності своєчасного реагування на незаплановані події.
%
%Вдосконалення систем диспеттчерського контрою призваного коригувати рух автомобіля актуальне в звязку з розвитком автоматизації і ....
%
%.
%
%Системи диспетчерського контролю за рухом автотранспорту призвані ефективно коригувати відхилення від запланованих маршрутів при зіткненні з непередбачуваними обставинами. Інформація про обставини має повідомлятися диспетчеру в найкоротщі терміни і з найменшими витратами часу для ефективних рішень. Голосові повідомлення часто потрибують більше часу і ...
%
%.
%
%.
%
%.
%
%В існуючих системах диспетчерського контролю не використовується голосовий звʼязок «водій --- обладнання автомобіля --- сервер». Всі параметри, такі як GPS трек, передаються диспетчеру у формалізованому вигляді із застосуванням мобільного інтернету або супутникового звʼязку.
%
%Диспетчер бачить що характеристика руху не відповідає запланованій і може її отримати або за рахунок безпосередньої голосової взаємодії за допомогою мобільного телефону, або у формалізованому вигляді через мобільний додаток водія з сенсорним інтерфейсом. 
%
%Водії часто уникають використовувати сенсорний додаток для передачі інформації про причини невідповідності руху та плану, через те що це відволікає від безпосередніх завдань керування автомобілем.
%
%Тобто єдиним каналом отримання цієї інформації залишається безпосередня, неформалізована голосова взаємодія. Проте така голосова взаємодія є витратною по часу як для водіїв так і для диспетчера і використання мобільного телефону є проблемою бо також відволікає від водійських функцій.
%
%.
%
%Формалізація голосової взаємодії потрібна, щоб забезпечити диспетчера швидкою і якісною інформацією про причини збоїв не відволікаючи водія від виконання його основних функцій.
%
%.
%
%
%Під час руху автотранспорту завжди відбуваюся ті чи інші відхилення від плану, які в кожному випадку потребують коригування плану через комунікацію з диспетчером.
%
%.
%
%.
%
%Навіть при отриманні диспетчером всіх параметрі руху у формалізованому вигляді із застосуванням мобільного інтернету або супутникового звʼязку залишається невідомою причина відхилення від плану, що має суттево значення для прийняття диспетчером рішення, щодо подальшого руху автотранспорту.
%
%.
%
%.
%
%.
%
%
%Розвиток прикладних інформаційних технологій на ринку транспортних послуг зумовлений посиленням жорсткої економічної конкуренції та запитом на підвищення екологічності, комфорту та ефективності роботи персоналу.
%
%\ifsynopsis
%\else
%Сьогодні, світові виробники автомобілів, електроніки та телекомунікаційних технологій створюють та використовують компʼютерні інформаційні системи у спроектованих та діючих транспортних засобах. За останні десятиліття, більшість автомобілів набуло оснащення інтерактивними інформаційними системами, що включають аудіо та відео системи, супутникові навігаційні системи, гарнітури телефонії і контроль над кліматом та технічним станом автомобіля. Не дивлячись на те, що такі системи обладнанні дисплеєм, голосова взаємодія з водієм стає все більш широко використовуваною в автомобілях, що допомагає збільшити кількість контрольованих функцій і систем, кнопки яких не можуть бути встановлені на рульовому колесі та приладовій панелі, оскільки обмежено простір. Голосова технологія також дозволяє водіям не відволікатись від управління, знижуючи ймовірність виникнення небезпечних ситуацій на дорозі та підвищуючи безпеку руху. Власними назвами систем голосового управління володіють такі бренди, як Mercedes-Benz, Ford, Cadillac. На марках автомобілів Audi, BMW, Kia, Lexus установлені системи голосового управління для зручності і забезпечення комфорту водіїв. Для систем голосового управління характерна різниця, що полягає у кількості підтримуваних мов, належному рівню при розпізнаванні команд, кількості реалізації функцій управління. Найбільшою кількістю мов володіє система «Ford Sync», крім того арсенал включає і російську мову, але української – немає.
%\fi
%
%У звʼязку з дедалі активнішим використанням природного інтерфейсу і зокрема голосу для спілкування водія з технічними засобами зросло і значення систем голосового управління в самому автомобілі як носія інформації у системах диспетчерського контролю за рухом автотранспорту при здійсненні етапів дистрибуції «склад – дорога – точка доставки».
%
%Не дивлячись на інтенсивний розвиток систем диспетчерського контролю за рухом автотранспорту при взаємодії із водієм, саме голосова інформація потребує формалізації у випадку проведення автоматизації таких систем. Проте існуючі розробки в сфері формалізації голосової інформації поки не пристосовані для аналізу мовлення водіїв, з метою покращення та полегшення їх взаємодії з диспетчерською системою. Саме модель голосової взаємодії водія в системах диспетчерського контролю потребує автоматизації для підвищення ефективності процесу дистрибуції.
%
%Фінальна доставка до дверей клієнта, відома як «остання миля», є одним з найдорожчих та найскладніших у організації дистрибуції. Під час виконання доставки завжди відбуваються ті чи інші відхилення від плану, яким би оптимальним він не був, подібні відхилення в кожному випадку потребують коригування плану через комунікацію з диспетчером. Водії-експедитори та кур'єри скоріше починають виконувати доставки поза планом, якщо процеси комунікації з диспетчером та коригування планів недостатньо прості та ефективні. Для задачі дистрибуції може бути важко забезпечити постійний доступ до мережі інтернет, оскільки доставка може відбуватися до місць/регіонів, де навіть мобільний GPRS інтернет відсутній, або має надто низьку швидкість передачі даних для роботи зі звуком.
%
%Інформаційних технологій які забезпечують автоматизацію голосової взаємодії в системах дистрибуції розроблено не достатньо.
%
%Все це робить тему дисертаційного дослідження інформаційних технологій формалізації голосової інформації в системах диспетчерського контролю за рухом автотранспорту \textbf{актуальною}.

\begin{refsection}
Питаннями формалізації голосової взаємодії, побудови діалогових систем та сценаріїв голосової взаємодії займалися такі вчені, як \citea{Ishiguro_2016,Iosif_2018,Herbert_2018,Lopes_2015,Khouzaimi_2018}, але їх дослідження не були спрямовані на сферу систем диспетчерського контролю за рухом автотранспорту. 

Досліджували голосове управління у транспортних системах \citea{Kravchenko_2009,Korsun_2013,Heisterkamp_2001,Jonsson_2009}, але більша частина результатів спрямована на автоматизацію голосового управління бортовим обладнанням.

Загалом різні способи покращення диспетчерського контролю при виконання доставок автотранспортом розглядали \citea{Prasanna_2012,Stopher_2018,Prasanna_2012,Liu_2018,Govindan_2018,Stopher_2018,Papetti_2019,,Gonzalez_2013,Comendador_2012,Baumann_2012,Quak_2006}, але їхні роботи не були спрямовані на формалізацію голосової взаємодії.

Питаннями формалізації голосової інформації за рахунок переведення її у текст розглядає велика кількість учених, зокрема \citea{Pylypenko_2008,Lydovyk_2011,Vasilyeva_2012,Womack_1999,Zirneeva_2008,Gladunov_2005,Robeyko_2012,Abdel_2012,Zhang_2017,Sharma_2018,Yermolenko_2008,He_2019}, на сам перед шляхом переведення переведення голосу у текст, який, за рахунок необхідності великих словників та доступу до інтернету, має певні обмеження для використання на мобільних пристроях.

Вирішення зазначених суперечностей теорії та практики формалізації голосової інформації може спиратися на дослідження голосового управління, заснованих на теорії несилової взаємодії та рефлекторні системи голосового управління, що належать такими вченим, як Тесля~Ю.~М., Пилипенко~В.~В., Чорний~О.~Ю., Єгорченков~А.~В.
\end{refsection}

У звʼязку з цим, на даний час існує необхідність вирішення актуального наукового завдання розробки моделей і методів формалізації голосової інформації в системах диспетчерського контролю за рухом автотранспорту.

%Питаннями автоматизації систем голосового управління займалися такі вчені, як: Бондарос Ю.Г., Волков А.В., Кравченко А.П., Козлов О.С., Корсун О.М., Любімов А.М, Пилипенко В.В., Робейко В.В., Тесля Ю.М., Чорний О.Ю., Чучупал В.Я., Фінаєв І.М., Яцко А.А., Britz D., Deng L., Heisterkamp P., Hinton G., Jonsson I.-M., Kim Y., LeCun Y., Saini P., Yu D., Zhang X., Zhao J.J. та багато інших. Зокрема, результати досліджень голосового управління, заснованих на теорії несилової взаємодії та рефлекторної системи голосового управління належать таким вченим як: Тесля Ю.М., Пилипенко В.В., Чорний О.Ю., Єгорченков А.В.

\textbf{Звʼязок роботи з науковими програмами і планами.}

Дисертаційна робота виконана відповідно до пріоритетного напряму розвитку інформаційних та комунікаційних технологій, що визначені в Законі України «Про пріоритетні напрями розвитку науки і техніки» на період до 2020 року та тематичного плану науково-дослідних робіт Київського національного університету імені Тараса Шевченка в рамках науково-дослідної роботи «Розробка теоретико-методологічних основ впровадження систем управління проектами для розвитку підприємств і організацій» (№ держреєстрації 0117U002694), у якій автор брав участь як виконавець, запропонувавши впровадження графу сценаріїв для систем формалізації голосового управління.

\textbf{Обʼєктом дослідження} є процеси голосової взаємодії в системах диспетчерського контролю за рухом автотранспорту.

\textbf{Предмет дослідження} – моделі і методи формалізації голосової взаємодії в системах диспетчерського контролю за рухом автотранспорту.

\textbf{Метою дослідження} підвищення ефективності розпізнавання повідомлень у голосовій взаємодії водія з диспетчером на основі розробки та використання інформаційної технології формалізації голосової інформації в системах диспетчерського контролю за рухом автотранспорту.

Для досягнення сформульованої мети поставлено ряд часткових \textbf{завдань досліджень}:

\begin{itemize}
	\item здійснити аналіз сучасних інформаційних систем обробки та формалізації голосової інформації;
	\item розробити метод формалізації голосової інформації в допоміжних системах диспетчеризації автотранспорту;
	\item розробити математичну модель голосової взаємодії водія та диспетчера в системах диспетчерського контролю за рухом автотранспорту у вигляді повного графу сценаріїв усіх етапів процесу доставки «склад – дорога – точка доставки»;
	\item адаптувати метод структурної ідентифікації згорткових нейронних мереж для класифікації голосових команд для розпізнання фонемного тексту;
	\item поєднати метод інтелектуальних рефлекторних систем з теоретичним апаратом теорії нейронних мереж;
	\item провести експериментальні дослідження на основі математичного моделювання формалізації голосової інформації, отриманої від водіїв, для оптимізації диспетчерського контролю за рухом автотранспорту.
\end{itemize}

\textbf{Методи дослідження}. Для досягнення поставленої мети в роботі використано: теорію графів --- для опису моделі голосової взаємодії, теорію інформації та теорію несилової взаємодії --- для вдосконалення методу інтелектуальних рефлекторних систем, теорію штучних нейронних мереж та методи обробки природної мови --- для вдосконалення методу згорткових нейронних мереж.

\textbf{Наукова новизна отриманих результатів} полягає в тому, що в дисертаційній роботі:

\begin{itemize}
	\item вперше розроблено метод формалізації голосової інформації в системах підтримки диспетчеризації автотранспорту, який на відміну від аналогів поєднує використання інтелектуальних рефлекторних систем та згорткових нейронних мереж, що дозволяє автоматизувати процес передачі голосової інформації;
	\item удосконалено математичну модель голосової взаємодії водія та диспетчера в системах диспетчерського контролю за рухом автотранспорту, яка на відміну від існуючих представлена у вигляді повного графу сценаріїв усіх етапів процесу доставки «склад – дорога – точка доставки», що дозволяє виділити контексти голосової взаємодії для підвищення точності подальшої формалізації;
	\item набув подальшого розвитку метод структурної ідентифікації згорткових нейронних мереж для класифікації голосових команд, в якому на відміну від існуючих ведеться розпізнання фонемного тексту, що дозволяє класифікувати голосові команди без переведення голосу в лексичний текст;
	\item отримав подальший розвиток метод інтелектуальних рефлекторних систем, який відрізняється від існуюих поєднанням з теоретичним апаратом теорії нейронних мереж, шо дає можливість оптимізувати значення інформованості та визначеності шляхом навчання методом зворотного розповсюдження помилки.
\end{itemize}

\textbf{Практичне значення} отриманих результатів полягає в тому, що з використанням наукових результатів, закладається можливість підвищення точності та швидкості розпізнавання голосових повідомлень безпосередньо на мобільному пристрої, що покращує можливості диспетчерського контролю за рухом автотранспорту. Розроблені на базі запропонованих особисто автором моделей і методів програмні засоби становлять практичний результат, який впроваджений на підприємстві ТОВ «УІТ», м. Київ.

\textbf{Особистий внесок здобувача.} Наукові положення, розробки та висновки дисертаційної роботи є результатом самостійно проведеного дослідження здобувача. Основні наукові результати, представлені в дисертації, отримані здобувачем особисто.

\textbf{Апробація результатів досліджень.} Основні положення дисертаційної роботи були апробовані на 5-х міжнародних науково-технічних конференціях, в тому числі:

\begin{itemize}
	\item XVII Міжнародна науково-технічна конференція «Системний аналіз та інформаційні технології» (м. Київ, 22--25 червня 2015 р.)
	\item XII Міжнародна конференція «Управління проектами у розвитку суспільства», тема: «Комплексне управління проектами розвитку в умовах нестабільного оточення» (м. Київ, 21-23 травня 2015 р.)
	\item ІІІ Міжнародна науково-практична конференція «Інформаційні технології та взаємодії» (м. Київ, 8--10 листопада 2016 р.)
	\item 16th EAGE International Conference on Geoinformatics - Theoretical and Applied Aspects (м. Київ, 15--17 травня 2017 р.)
	\item ІV Міжнародна науково-практична конференція «Інформаційні технології та взаємодії» (м. Київ, 8--10 листопада 2017 р.)
\end{itemize}

\printbibliography[heading=countauthor, env=countauthor, keyword=biblioauthor, section=1]%
\printbibliography[heading=countauthorvak, env=countauthorvak, keyword=biblioauthorvak, section=1]%
\printbibliography[heading=countauthorconf, env=countauthorconf, keyword=biblioauthorconf, section=1]%

\textbf{Публікації.} 
За результатами дисертаційних досліджень опубліковано
\formbytotal{citeauthor}{науков}{у працю}{і праці}{их праць}. 
Основні наукові положення викладено у 
\formbytotal{citeauthorvak}{науков}{ій статті}{их статтях}{их статтях} \cite{art2,art3,art4,art5,art8},
серед яких \cite{art2,art3,art4,art5} опубліковані у спеціалізованих фахових виданнях України, 
\cite{art8} опубліковано у закордонному науковому виданні. 
За матеріалами виступів на науково-технічних конференціях опубліковано 
\formbytotal{citeauthorconf}{тез}{а}{и}{} доповідей \cite{conf5,conf6,conf8,conf9,conf10}.
Додатково результати досліджень відображені в науковій статті \cite{art1}.

\printbibliography[heading=countauthor, env=countauthor, keyword=biblioauthor]%

\ifsynopsis

%\todo{
\textbf{Структура і обсяг роботи.} Дисертаційна робота представлена на 246 сторінках друкованого тексту, включає 51 рисунок, 20 таблиць, які розташовані на 31 повній сторінці тексту. Робота складається з вступу, чотирьох розділів, висновків і списку використаних джерел із 142 найменуваннями, який розміщений на 21 сторінці. Основний текст викладений на 124 сторінках роботи.
%}

\else

\newtotcounter{MainPages}
\setcounter{MainPages}{\pagedifference{introduction}{conclusion_end}}
\newtotcounter{RefPages}
\setcounter{RefPages}{\pagedifference{references}{references_end}}

\textbf{Структура і обсяг роботи.} Дисертаційна робота представлена на \formbytotal{TotPages}{сторін}{ці}{ках}{ках} друкованого тексту, включає \formbytotal{totalcount@figure}{рисун}{ок}{ки}{ків}, \formbytotal{totalcount@table}{таблиц}{ю}{і}{ь}, які розташовані на 31 повній сторінці тексту. Робота складається з вступу, чотирьох розділів, загальних висновків і списку використаних джерел із \formbytotal{citenum}{найменуван}{ням}{нями}{ь}, який розміщений на \formbytotal{RefPages}{сторін}{ці}{ках}{ках}. Основний текст викладений на \formbytotal{MainPages}{сторін}{ці}{ках}{ках} роботи.

\fi

 % Характеристика работы по структуре во введении и в автореферате не отличается (ГОСТ Р 7.0.11, пункты 5.3.1 и 9.2.1), потому её загружаем из одного и того же внешнего файла, предварительно задав форму выделения некоторым параметрам

%Диссертационная работа была выполнена при поддержке грантов \dots

%\underline{\textbf{Объем и структура работы.}} Диссертация состоит из~введения,
%четырех глав, заключения и~приложения. Полный объем диссертации
%\textbf{ХХХ}~страниц текста с~\textbf{ХХ}~рисунками и~5~таблицами. Список
%литературы содержит \textbf{ХХX}~наименование.

\pdfbookmark{Содержание работы}{description}                          % Закладка pdf
\section*{Содержание работы}

\pdfbookmark{Содержание первой главы}{chfirst}
\paragraph{Содержание первой главы.} В первой главе исследуется уравнение Мэки--Гласса в предположении, что показатель степени в знаменателе нелинейности --- большой параметр. Рассматривается случай, в котором релейное уравнение, возникающее при устремлении большого параметра к бесконечности, имеет периодическое решение с наименьшим числом переключений релейной части на периоде. Для данного случая доказывается существование периодического решения уравнения Мэки--Гласса, асимптотически близкого периодическому решению предельного уравнения.

Уравнение Мэки--Гласса \eqref{eq:mg_equation_1:intro} после нормировки параметров и времени принимает вид
\begin{equation}
	\label{eq:intro:mg_norm}
	\dot{u}=-\beta u+\frac{\alpha u(t-1)}{1+(u(t-1))^\gamma}, \text{ где } \alpha > 0, \beta > 0, \gamma > 0.
\end{equation}

Для положительных начальных функций решение уравнения \eqref{eq:intro:mg_norm} также положительно, поэтому корректна замена $u = e^x$, после которой это уравнение примет вид
\begin{equation}
	\label{eq:intro:MG_x}
	\dot{x}=-\beta+\alpha\frac{e^{x(t-1)-x}}{1+e^{\gamma x(t-1)}}.
\end{equation}

Будем считать $\gamma$ большим параметром. При $\gamma \to +\infty$ получаем релейное уравнение
\begin{equation}
	\label{eq:intro:MG_rele}
	\dot{x}=-\beta + \alpha e^{-x} F(\exp({x(t-1)})),
\end{equation}
где
\begin{equation}
	\label{eq:intro:F}
	F(u)=\lim\limits_{\gamma\to +\infty}\frac{u}{1+u^{\gamma}}=
	\begin{cases}
		0, & u > 1,\\
		\frac{1}{2}, & u = 1,\\
		u, & 0 \leq u < 1.
	\end{cases}
\end{equation}

Определим множество начальных функций, для которых затем явно будет построено решение предельного уравнения методом шагов. Зафиксируем положительные параметры $\sigma_0 < 1/2$, $p$, $q$ такие, что 
%
\[0 < p < \beta \sigma_0 < q.\]
%
В качестве множества начальных функций для уравнений \eqref{eq:intro:MG_x} и \eqref{eq:intro:MG_rele} зададим множество
\begin{multline}
	\label{eq:intro:init_set}
	S=\{\varphi\in C[-1 - \sigma_0, -\sigma_0]: 0 < p \leqslant \varphi(t)\leqslant q \text{ при } t \in [-1 - \sigma_0, -\sigma_0],\\ \varphi(-\sigma_0) = \beta \sigma_0 \}.
\end{multline}

\begin{figure}
	\centering
	\includegraphics[width=0.7\textwidth]{initial_func_S.eps}
	\caption{Представитель множества \eqref{eq:intro:init_set} начальных функций уравнений \eqref{eq:intro:MG_x} и \eqref{eq:intro:MG_rele}.}
	\label{fig:intro:initial_funcs:ch1}
\end{figure}

Введём обозначения:
\begin{equation}
	\label{eq:intro:T}
	T = \frac{1}{\beta} \ln\left(\frac{1}{2}\alpha^2e^{2\beta}(t_1 - 2)^2 + \alpha e^{\beta}(t_1 - 1) + 1\right),
\end{equation}
где $t = t_1$ --- корень уравнения 
\begin{equation}
	\label{eq:intro:t1_cond_exp}
	-\beta(t - 1) + \ln(\alpha e^{\beta}(t - 2) + 1) = 0,
\end{equation}
\begin{equation}
	\label{eq:intro:t2_period}
	t_2 = 1 + T.
\end{equation}

Верна следующая теорема \cite{wosbib1}.

\textbf{Теорема} (О решении релейного уравнения). \textit{
	Для произвольного $\beta > 0$ и достаточно большого $\alpha$ уравнение \eqref{eq:intro:MG_rele} с начальной функцией из множества \eqref{eq:intro:init_set} имеет $T$-периодическое решение
	\small
	\begin{equation}
		\label{eq:intro:sol_x_star}
		x^*(t)= 
		\begin{cases}
			-\beta t, & t\in[-\sigma_0, 1],\\
			-\beta t +\ln(\alpha e^{\beta}(t - 1)+1), & t\in[1, 2],\\
			-\beta t + \ln(\frac{\alpha^2}{2}e^{2\beta}(t - 2)^2+\alpha e^{\beta}(t - 1)+1), & t\in[2, t_1],\\
			-\beta t + \ln(\frac{\alpha^2}{2}e^{2\beta}(t_1 - 2)^2+\alpha e^{\beta}(t_1 - 1) + 1), & t\in[t_1, t_2].
		\end{cases}
	\end{equation}
	\normalsize
}

Изображение цикла \eqref{eq:intro:sol_x_star} приведено на рисунке \ref{fig:intro:x_star:ch1}.

% 2024-01-21-mackey-glass-asymptotics.ipynb
\begin{figure}
	\centering
	\includegraphics[width=0.7\textwidth]{x_star.eps}
	\captionof{figure}{Периодическое решение $x^{*}(t)$ уравнения \eqref{eq:intro:MG_rele}.}
	\label{fig:intro:x_star:ch1}
\end{figure}

\medskip

Введём вспомогательные функции $w_i(\tau)$, $i = 0, 1$ для описания решения в окрестностях точек излома релейного уравнения.
\[
w_i(\tau) = -\beta \tau - \dfrac{\alpha e^{-x^*(t_i)}}{\dot{x}^*(t_i - 1)} \ln\left(e^{-\dot{x}^*(t_i - 1)\tau} + 1\right) \quad \text{при} \quad \dot{x}^*(t_i - 1) < 0,
\]
\[
w_i(\tau) = (-\beta + \alpha e^{-x^*(t_i)})\tau - \dfrac{\alpha e^{-x^*(t_i)}}{\dot{x}^*(t_i - 1)} \ln\left(e^{\dot{x}^*(t_i - 1)\tau} + 1\right) \quad \text{при} \quad \dot{x}^*(t_i - 1) > 0.
\]
Соответствующие значения функции $x^*(t)$ находится из формул \eqref{eq:intro:sol_x_star}.

Данные функции удовлетворяют асимптотическим свойствам.

Если $\dot{x}^*(t_i - 1) < 0$,
\begin{equation*}
	w_i(\tau) = -\beta \tau + O(\exp(-\dot{x}^*(t_i - 1) \tau)) \text{ при } \tau \to -\infty,
\end{equation*}
\begin{equation*}
	w_i(\tau) = (-\beta + \alpha e^{-x^*(t_i)})\tau + O(\exp(\dot{x}^*(t_i - 1) \tau)) \text{ при } \tau \to +\infty,
\end{equation*}

Если $\dot{x}^*(t_i - 1) > 0$,
\begin{equation*}
	w_i(\tau) = (-\beta + \alpha e^{-x^*(t_i)})\tau + O(\exp(\dot{x}^*(t_i - 1) \tau)) \text{ при } \tau \to -\infty,
\end{equation*}
\begin{equation*}
	w_i(\tau) = -\beta \tau + O(\exp(-\dot{x}^*(t_i - 1) \tau)) \text{ при } \tau \to +\infty.
\end{equation*}

Коэффициенты при $\tau$ в приведённых формулах совпадают с односторонними производными в точках излома решения релейного уравнения $t_i$ при $i = 0, 1$.

Основной результат главы формулируется следующим образом.

\bigskip

\textbf{Теорема.} \textit{Для произвольного $\beta > 0$ и достаточно больших $\alpha$ существуют такие значения параметров $\sigma_0, p, q$ и такое достаточно большое $\gamma_0$, что при всех $\gamma > \gamma_0$ уравнение \eqref{eq:intro:MG_x} с начальной функцией $\varphi$ из множества \eqref{eq:intro:init_set} обладает периодическим решением $x^*_\gamma(t, \varphi)$ периода $T_{\gamma, \varphi}$ с асимптотикой}
\footnotesize
\begin{equation}
	\label{eq:intro:sol_x*gamma}
	x^*_\gamma(t, \varphi)= 
	\begin{cases}
		- \beta t + O(\gamma^{-1} e^{-\beta \delta \gamma}), & t\in[-\sigma_0, 1 - \delta],\\
		-\beta + \frac{1}{\gamma} w_0(\tau)|_{\tau=(t - 1)\gamma} + O(\gamma^{-2\nu}), & t \in [1 - \delta,1 + \delta],\\
		- \beta t + \ln(\alpha e^{\beta}(t - 1) + 1) + O(\gamma^{-2\nu}) & t\in[1 + \delta, 2]\\
		- \beta t + \ln(\frac{\alpha^2}{2}e^{2 \beta}(t - 2)^2 + \alpha e^{\beta}(t - 1) + 1) + O(\gamma^{-2\nu}), & t \in [2, t_1 - \delta],\\
		- \beta t_1 + \ln(\eta)+\frac{1}{\gamma} w_1(\tau)|_{\tau=(t - t_1)\gamma} + O(\gamma^{-2\nu}), & t\in[t_1 - \delta, t_1  +\delta],\\
		- \beta t + \ln(\eta) + O(\gamma^{-2\nu}), & t \in [t_1 + \delta, t_2 - \delta],
	\end{cases}
\end{equation}
\normalsize
где $\nu \in (\frac{1}{2}, 1)$, $\delta = \gamma^{-\nu}$, $\eta=\frac{\alpha^2}{2}e^{2\beta}(t_1 - 2)^2 + \alpha e^{\beta}(t_1 - 1) + 1$.
%
\textit{Данное решение удовлетворяет предельным равенствам}
%
\begin{equation}
	\label{eq:intro:lim_x*}
	\lim_{\gamma\to+\infty}\max_{0\leqslant t\leqslant T_{\gamma, \varphi}}|x_{\gamma}^*(t, \phi)-x^*(t)|=0,\quad \lim_{\gamma\to+\infty}T_{\gamma, \varphi} = T.
\end{equation}
\textit{Все остатки и пределы равномерны по $\varphi \in S$ и $t$ из соответствующих промежутков.}

\pdfbookmark{Содержание второй главы}{chsecond}
\textbf{Содержание второй главы.} Во второй главе рассматривается полносвязная сеть релейных осцилляторов Мэки--Гласса, описываемая системой \eqref{eq:intro:mg_full_renormed}. Будем искать решение в виде дискретной бегущей волны. После подстановки $u_j(t) = u(t + j\Delta)$ получим вспомогательное уравнение \eqref{eq:intro:mg_auxiliary}.

Нормируем время так, чтобы наименьшее из запаздываний вспомогательного уравнения стало равно 1. Пусть $1 = \tau_0 \leq \tau_1 \leq \ldots \leq \tau_N$ --- множество запаздываний после нормировки. Получим уравнение 
\begin{equation}
	\label{eq:intro:mg_relay_w}
	\dot{u}=-\beta u+\alpha F(w(t)), \text{ где } w(t) = \sum\limits_{s = 0}^N u(t - \tau_s).
\end{equation}
%
Определим множество начальных функций на промежутке длины наибольшего запаздывания: 
%
\begin{equation}
	\label{eq:intro:mg_init_set}
	\{\varphi\in C[-\tau_{N},0]:\  \varphi(t)>1 \text{ при } t\in[-\tau_{N},0),\ \varphi(0)=u_0 > 1\}.
\end{equation}
%
Введём обозначения:
%
\begin{equation*}
	A = \sum_{i=0}^{m}e^{\beta \tau_{i}}=e^\beta+e^{\beta \tau_1}+\ldots+e^{\beta \tau_{N}},
\end{equation*}
\begin{equation*}
	\tau_* = \min\{2,\tau_1\}=\left\lbrace\begin{array}{cl}
		\min\{2,1/\Delta\}, & \text{ если } \Delta < 1,
		\\
		\min\{2,\Delta\}, & \text{ если } \Delta > 1,
	\end{array}\right.
\end{equation*}
\begin{equation*}
	s_* = t_1-t_0.
\end{equation*}
%
Для вспомогательного уравнения \eqref{eq:intro:mg_relay_w} доказана следующая теорема.

\textbf{Теорема.} \textit{Для произвольного $\beta$ и достаточно больших $\alpha$ решение уравнения \eqref{eq:intro:mg_relay_w} при любой начальной функции из множества \eqref{eq:intro:mg_init_set} совпадает с одной и той же периодической функцией $u_*$, обладающей минимальным возможным числом точек излома на периоде.}

Схематичный график функции $u_*$ изображён на рисунке \ref{fig:intro:u_star}.

\begin{figure}[h]
	\centering
	\includegraphics[width=0.7\textwidth]{u_star.eps}
	\caption{Периодическая функция $u_*(t)$, являющаяся решением вспомогательного уравнения \eqref{eq:intro:mg_relay_w}.}
	\label{fig:intro:u_star}
\end{figure}

Для существования дискретной бегущей волны нужно проверить, что при некотором $\Delta$ период решения вспомогательного уравнения кратен $\Delta (N + 1)$, где $(N + 1)$ --- число уравнений в системе, т.е. для некоторого $p \in \mathbb{N}$ выполнено
\begin{equation}
	\label{eq:intro:period_eq}
	pT = \Delta (N + 1).
\end{equation}

Во второй главе доказывается следующая теорема.

\textbf{Теорема.} \textit{Для произвольного $\beta > 0$ и достаточно больших $\alpha$ найдётся $\Delta > 1$, при котором у системы \eqref{eq:intro:mg_full_renormed} существовует решение в виде дискретной бегущей волны.}
	
Численные эксперименты показывают, что решение является устойчивым; более того, результаты экспериментов позволяют предположить глобальную притягиваемость решений данного вида.

\bigskip

\pdfbookmark{Содержание третьей главы}{third}
\textbf{Содержание третьей главы.} В третьей главе рассматривается полносвязная сеть, состоящая из $N = m + n$ осцилляторов Мэки--Гласса, описываемая системой \eqref{eq:intro:mg_full_renormed_delta}. Будем искать решение, соответствующее режиму двухкластерной синхронизации. После подстановки \eqref{eq:intro:cluster} получим вспомогательную систему \eqref{eq:intro:system_uv}. После замен
%
\begin{equation}
	\label{eq:intro:tilde_change}
	\tilde{u}(t) = u(t - 1) + \delta (m - 1) u + \delta n v, \quad \tilde{v}(t) = v(t - 1) + \delta m u + \delta (n - 1) v
\end{equation}
%
и последующей экспоненциальной подстановки
\begin{equation}
	\label{eq:intro:exp_change}
	\tilde{u} = e^x, \quad \tilde{v} = e^y
\end{equation}
%
система \eqref{eq:intro:system_uv} примет вид
%
\begin{equation}
	\label{eq:intro:system_cluster_main}
	\begin{cases}
		\dot{x} = -\beta + \alpha \left(e^{x(t - 1) - x} G_{\gamma} (x(t - 1)) + \delta (m - 1) G_{\gamma} (x) + \delta n e^{y - x} G_{\gamma} (y)\right),\\
		\dot{y} = -\beta + \alpha \left(e^{y(t - 1) - y} G_{\gamma} (y(t - 1)) + \delta m e^{x - y} G_{\gamma} (x) + \delta (n - 1) G_{\gamma} (y)\right),
	\end{cases}
\end{equation}
где $G_{\gamma} (x) = e^{-x} \, F_{\gamma} (e^x)$.

Замена \eqref{eq:intro:tilde_change} корректна в смысле следующей теоремы.

\textbf{Теорема.} \textit{Пусть $(x, y)$ --- $T$-периодическое решение системы \eqref{eq:intro:system_cluster_main}. Тогда существуют $T$-периодические функции $u, v$, однозначно определяемые соотношениями \eqref{eq:intro:tilde_change}, \eqref{eq:intro:exp_change}, которые являются решением системы \eqref{eq:intro:system_uv}.}

Как и в предыдущей части диссертации, будем исследовать предельный при $\gamma \to +\infty$ объект. Заменим функцию $G_{\gamma}$ её предельной версией $G$:
\begin{equation}
	\label{eq:intro:relay_G_tilde}
	G(x) = \lim\limits_{\gamma \to +\infty} G_{\gamma}(x) = 
	\begin{cases}
		1, & x < 0,\\
		1/2, & x = 0,\\
		0, & x > 0.
	\end{cases}
\end{equation}
%
В этом случае правая часть релейной системы \eqref{eq:intro:system_cluster_main} терпит разрыв при $x = 0$ и при $y = 0$. Анализ системы показывает, что её решение при достижении одной из прямых разрыва может продолжаться только вдоль этой прямой (трансверсальное пересечение прямой разрыва оказывается невозможным).

Обобщённое решение строится методом эквивалентного управления \cite[\S 4, с. 54]{Filippov1988}.

Рассмотрим систему
%
\small
\begin{equation}
	\label{eq:intro:system_main_relay}
	\begin{cases}
		\dot{x} = -\beta + \alpha \left(e^{x(t - 1) - x} g_x(x(t - 1), t - 1) + \delta (m - 1) g_x(x, t) + \delta n e^{y - x} g_y(y, t)\right),\\
		\dot{y} = -\beta + \alpha \left(e^{y(t - 1) - y} g_y(y(t - 1), t - 1) + \delta m e^{x - y} g_x(x, t) + \delta (n - 1) g_y(y, t)\right),\\
		g_x(x, t) = G(x) \quad\text{при } x \neq 0,\\
		g_y(y, t) = G(y) \quad\text{при } y \neq 0,
	\end{cases}
\end{equation}
\normalsize
%
где функции $g_x(x, t)$ и $g_y(y, t)$ принимают при $x = 0$ (соответственно, $y = 0$) значения из интервала $(0, 1)$, позволяющее продолжить решение вдоль прямой $x = 0$. Как будет показано далее, значения $g_x(0, t)$ и $g_y(0, t)$ на каждом шаге определяются однозначно из уравнения $\dot{x} = 0$ или, соответственно, $\dot{y} = 0$. Функции $g_x$ и $g_y$ будем рассматривать как часть системы.
%
Систему \eqref{eq:intro:system_main_relay} будем называть \emph{релейной}.

Определим на множестве $[-1, 0]$ семейство пар начальных функций. Фиксируем $x_0 > y_0 > 0$.
\begin{equation}
	\label{eq:intro:initial_set}
	S = \left\{(\phi, \psi) \in (C[-1, 0])^2 \,|\, \phi(t) > 0, \psi(t) > 0, x_0 = \phi(0), y_0 = \psi(0)\right\}.
\end{equation}

Доказана следующая теорема.

\bigskip

\textbf{Теорема.}
\textit{В пространстве параметров $x_0$, $y_0$, $\alpha$, $\beta$, $\delta$ существует открытое множество, для любого набора параметров из которого и любой пары начальных функций из множества \eqref{eq:intro:initial_set} релейная система \eqref{eq:intro:system_cluster_main} имеет обобщённое решение $(x, y)$. График решения показан на рисунке \ref{fig:intro:cluster_step_by_step}.}

\begin{figure}[!ht]
	\centering
	\includegraphics[width=\textwidth]{cluster_step_by_step.eps}
	\caption{Обобщённое решение релейной системы \eqref{eq:intro:system_cluster_main}, построенное методом шагов. Чёрная линия --- компонента решения $x$, красная линия --- компонента решения $y$. Числа в кругах --- номер шага построения.}
	\label{fig:intro:cluster_step_by_step}
\end{figure}

Основным результатом третьей главы является следующая теорема.

\textbf{Теорема.} \textit{В пространстве параметров $x_0$, $y_0$, $\alpha$, $\beta$, $\delta$ существует открытое множество, для любого набора параметров из которого релейная система~\eqref{eq:intro:system_cluster_main} имеет периодическое решение.}

\FloatBarrier
\pdfbookmark{Заключение}{conclusion}                                  % Закладка pdf
В \textbf{заключении} приведены основные результаты работы, которые заключаются в следующем:
У дисертаційній роботі вирішено актуальне наукове завдання розробки моделей і методів формалізації голосової інформації в системах диспетчерського контролю за рухом автотранспорту. Загалом можна зробити наступні висновки.

1. Дослідження теоретико-методологічних засад формалізації голосової інформації в системах дистрибуції показало, що значну роль в їх управлінні відіграють процеси голосової взаємодії особливо стосовно своєчасного коригування планових маршрутів руху автотранспорту. Розроблення моделі голосової взаємодії без блоку переведення звуку голосу в текст може принципово покращити автоматизацію голосової взаємодії в системах контролю дистрибуції.

2. Розроблена система автоматичного розрахунку планових маршрутів та практика її використання забезпечили накопичення параметрів непередбачуваних ситуацій в процесі доставки, що впливають на створення сценаріїв голосової взаємодії, які представляються у вигляді орієнтованого графу та контекстів взаємодії. 
Принципи побудови рефлекторних систем на основі теорії несилової взаємодії адаптовано для формалізації голосової інформації в системах диспетчерського контролю за рухом автотранспорту.

3. Розроблено математичну модель голосової взаємодії водія та диспетчера в системах диспетчерського контролю за рухом автотранспорту, яка представлена у вигляді повного графу сценаріїв усіх етапів дистрибуції «склад – дорога – точка доставки». Виділено перелік унікальних контекстів голосової взаємодії, формалізація голосової інформації в яких може відбуватися незалежно, що дозволяє знизити кількість реакцій для автоматизованого розпізнання.

4. Розроблено метод формалізації голосової інформації в системах підтримки диспетчеризації автотранспорту з використанням інтелектуальних рефлекторних систем, що дозволяє автоматизувати процес передачі голосової інформації з уникненням переводу звукової інформації в лексичний текст за рахунок використання двох основних модулів (автоматичного фонетичного стенографа і ядра рефлекторної системи голосового управління). Для реалізації ядерного компонента запропоновано дуальну систему класифікації голосових команд, яка може використовувати метод інтелектуальних рефлекторних систем або метод згорткових нейронних мереж.

5. Метод структурної ідентифікації згорткових нейронних мереж для класифікації голосових команд адаптовано до розпізнавання фонемного тексту, що дозволяє класифікувати голосові команди без переведення голосу в лексичний текст. 

6. Метод інтелектуальних рефлекторних систем поєднано з теоретичним апаратом теорії нейронних мереж, шо дає можливість оптимізувати значення інформованості та визначеності шляхом навчання методом зворотного розповсюдження помилки.

7. Результати математичного моделювання формалізації голосової інформації показав підвищення ефективності розпізнавання повідомлень у голосовій взаємодії водія з диспетчером, а саме підвищення точності розпізнавання у середньому на 6.6 \% для згорткових нейронних мереж і на 19.1 \% для інтелектуальних рефлекторних систем за рахунок використання моделі голосової взаємодії водія та диспетчера. Крім того використання моделей на основі згорткових нейронних мереж показало підвищення швидкості розпізнавання на 15 \% порівняно з інтелектуальними рефлекторними системами.

8. Результати досліджень впроваджені в ТОВ «УІТ», м. Київ (довідка від 4 січня 2019) та використовувалися у трьох логістичних компаніях-клієнтах протягом року.

9. Мета досліджень щодо підвищення ефективності розпізнавання повідомлень у голосовій взаємодії водія з диспетчером досягнута іта всі часткові завдання вирішені повністю. Наукові результати досліджень є внеском у розвиток наукових і методологічних основ створення та застосування інформаційних технологій та інформаційних систем для автоматизованої переробки інформації й управління.

10. Перспективним шляхом подальших досліджень у зазначеному напрямку може бути широке коло питань щодо розробки та дослідження інших реалізацій фонемного стенографа, використання розроблених методів та моделей класифікації фонемного тексту для роботи з лексичним текстом, а також створення моделей голосової взаємодії у вигляді графу сценаріїв для інших предметних областей.


\pdfbookmark{Литература}{bibliography}                               % Закладка pdf

\ifdefmacro{\microtypesetup}{\microtypesetup{protrusion=false}}{} % не рекомендуется применять пакет микротипографики к автоматически генерируемому списку литературы
\urlstyle{rm}                               % ссылки URL обычным шрифтом
\ifnumequal{\value{bibliosel}}{0}{% Встроенная реализация с загрузкой файла через движок bibtex8
    \renewcommand{\bibname}{\large \bibtitleauthor}
    \nocite{*}
    \insertbiblioauthor           % Подключаем Bib-базы
    %\insertbiblioexternal   % !!! bibtex не умеет работать с несколькими библиографиями !!!
}{% Реализация пакетом biblatex через движок biber
    % Цитирования.
    %  * Порядок перечисления определяет порядок в библиографии (только внутри подраздела, если `\insertbiblioauthorgrouped`).
    %  * Если не соблюдать порядок "как для \printbibliography", нумерация в `\insertbiblioauthor` будет кривой.
    %  * Если цитировать каждый источник отдельной командой --- найти некоторые ошибки будет проще.
    %
    %% authorvak
    \nocite{vakbib1}%
    \nocite{vakbib2}%
    %
    %% authorwos
    \nocite{wosbib1}%
    %
    %% authorscopus
    \nocite{scbib1}%
    %
    %% authorpathent
    \nocite{patbib1}%
    %
    %% authorprogram
    \nocite{progbib1}%
    %
    %% authorconf
    \nocite{confbib1}%
    \nocite{confbib2}%
    %
    %% authorother
    \nocite{bib1}%
    \nocite{bib2}%

    \ifnumgreater{\value{usefootcite}}{0}{
        \begin{refcontext}[labelprefix={}]
            \ifnum \value{bibgrouped}>0
                \insertbiblioauthorgrouped    % Вывод всех работ автора, сгруппированных по источникам
            \else
                \insertbiblioauthor      % Вывод всех работ автора
            \fi
        \end{refcontext}
    }{
        \ifnum \totvalue{citeexternal}>0
            \begin{refcontext}[labelprefix=A]
                \ifnum \value{bibgrouped}>0
                    \insertbiblioauthorgrouped    % Вывод всех работ автора, сгруппированных по источникам
                \else
                    \insertbiblioauthor      % Вывод всех работ автора
                \fi
            \end{refcontext}
        \else
            \ifnum \value{bibgrouped}>0
                \insertbiblioauthorgrouped    % Вывод всех работ автора, сгруппированных по источникам
            \else
                \insertbiblioauthor      % Вывод всех работ автора
            \fi
        \fi
        %  \insertbiblioauthorimportant  % Вывод наиболее значимых работ автора (определяется в файле characteristic во второй section)
        \newpage
        \begin{refcontext}[labelprefix={}]
            \insertbiblioexternal            % Вывод списка литературы, на которую ссылались в тексте автореферата
        \end{refcontext}
        % Невидимый библиографический список для подсчёта количества внешних публикаций
        % Используется, чтобы убрать приставку "А" у работ автора, если в автореферате нет
        % цитирований внешних источников.
        \printbibliography[heading=nobibheading, section=0, env=countexternal, keyword=biblioexternal, resetnumbers=true]%
    }
}
\ifdefmacro{\microtypesetup}{\microtypesetup{protrusion=true}}{}
\urlstyle{tt}                               % возвращаем установки шрифта ссылок URL
