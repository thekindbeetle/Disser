\thispagestyle{empty}
\begin{center}%
	\thesisOrganizationDone\\
	\thesisOrganizationMinistryDone\\
\end{center}%

\vspace{0pt plus1fill}
\begin{flushright}
	На правах рукопису
\end{flushright}

\vspace{0pt plus0.25fill}
\begin{center}%
\textbf{\MakeUppercase{\thesisAuthor}}
\ifnumequal{\value{showperssign}}{0}{}{%
\begin{picture}(0,0)
\put(25,-20){\includegraphics[height=1.5cm]{personal-signature.png}}
\end{picture}
}
\end{center}%


\vspace{0pt plus0.25fill}
\begin{flushright}%
	УДК \thesisUdk
\end{flushright}%



\vspace{0pt plus2fill} %число перед fill = кратность относительно некоторого расстояния fill, кусками которого заполнены пустые места
\begin{center}
\textbf{\MakeUppercase{\thesisTitle}}

\vspace{0pt plus2fill}
Спеціальність \thesisSpecialtyNumber\ "--- \thesisSpecialtyTitle

\vspace{0pt plus2fill}
\textbf{Автореферат}

дисертації на здобуття наукового ступеня

\thesisDegree
\end{center}

\vspace{0pt plus6fill} %число перед fill = кратность относительно некоторого расстояния fill, кусками которого заполнены пустые места
{\centering\thesisCity~--- \thesisYear\par}

\newpage
% оборотная сторона обложки
\thispagestyle{empty}
Дисертацією є рукопис.

Роботу виконано  на {\thesisInOrganization}.

\vspace{0.008\paperheight plus1fill}
\noindent%
\begin{tabularx}{\textwidth}{@{}lX@{}}
    \textbf{Науковий керівник} --   & \supervisorRegalia\par
                              \textbf{\supervisorFio},\par
                              \supervisorJobPlace,\par
                              \supervisorJobPost\ (\supervisorJobCity)\par
    \vspace{0.013\paperheight}\\
    \textbf{Офіційні опоненти}:  &
    \ifnumequal{\value{showopplead}}{0}{\vspace{13\onelineskip plus1fill}}{%
        \textbf{\opponentOneFio},\par
        \opponentOneRegalia,\par
        \opponentOneJobPlace,\par
        \opponentOneJobPost\ (\opponentOneJobCity)\par
            \vspace{0.01\paperheight}
        \textbf{\opponentTwoFio},\par
        \opponentTwoRegalia,\par
        \opponentTwoJobPlace,\par
        \opponentTwoJobPost\ (\opponentTwoJobCity)
    }%
%    \vspace{0.013\paperheight} \\
%    Ведущая организация:    &
%    \ifnumequal{\value{showopplead}}{0}{\vspace{6\onelineskip plus1fill}}{%
%        \leadingOrganizationTitle
%    }%
\end{tabularx}
\vspace{0.008\paperheight plus1fill}

Захист відбудеться \defenseDate~на~засіданні спеціалізованої вченої ради \defenseCouncilNumber~в \defenseCouncilTitle~за адресою: \defenseCouncilAddress.

\vspace{0.008\paperheight plus1fill}
З дисертацією можна ознайомитись у бібліотеці \synopsisLibraryTitle~за адресою: \synopsisLibraryAddress.

%\vspace{0.008\paperheight plus1fill}
%\noindent Отзывы на автореферат в двух экземплярах, заверенные печатью учреждения, просьба направлять по адресу: \defenseCouncilAddress, ученому секретарю диссертационного совета~\defenseCouncilNumber.

\vspace{0.008\paperheight plus1fill}
Автореферат розіслано \synopsisDate.

%\noindent Телефон для справок: \defenseCouncilPhone.

\makeatletter
\vspace{0.008\paperheight plus1fill}
\noindent%
\begin{tabularx}{\textwidth}{@{}%
>{\raggedright\arraybackslash}b{21em}@{}
>{\centering\arraybackslash}X
r
@{}}
    Вчений секретар спеціалізованої вченої ради\par
    \defenseSecretaryRegalia
    &
    \ifnumequal{\value{showsecrsign}}{0}{}{%
        \parbox[c]{\hsize}{\includegraphics[width=2cm]{secretary-signature.png}}%
    }%
    &
    \defenseSecretaryFio
\end{tabularx}
\makeatother
