\begin{frame}
    \frametitle{Научная новизна}
    \begin{itemize}
    	\item Впервые получены асимптотические формулы решения уравнения Мэки--Гласса по параметру $\gamma \gg 1$ и доказано существование периодических решений для достаточно больших значений отношения $a / b$.
    	\item Впервые доказано существование периодических режимов в виде дискретной бегущей волны в полносвязной цепи релейных генераторов Мэки--Гласса, а также сформулированы и доказаны условия их существования в виде ограничения на параметры соответствующей системы дифференциальных уравнений с запаздыванием.
    	\item Впервые доказано существование периодических режимов двухкластерной синхронизации в полносвязной цепи релейных генераторов Мэки--Гласса, а также сформулированы и доказаны условия их существования в виде ограничения на параметры соответствующей системы дифференциальных уравнений с запаздыванием.
    	%TODO: сказать про скользящие траектории.
    \end{itemize}
\end{frame}
\note{
    Проговаривается вслух научная новизна
}

\begin{frame} % публикации на одной странице
	\frametitle{Список публикаций}
	\begin{itemize}
		\item Two-cluster synchronization on a fully coupled network of Mackey--Glass generators // V.~Alekseev // Partial Differential Equations in Applied Mathematics. --- 2024. --- Vol. 12. --- P. 100930.
		\item Existence of Discrete Traveling Waves in Fully Coupled Network of Mackey--Glass Relay Generators / V.~Alekseev, M.~Preobrazhenskaia, V.~Vorontsova // Differential Equations. --- 2024. --- Vol. 60, No 9.
		\item Анализ асимптотической сходимости периодического решения уравнения Мэки--Гласса к решению предельного релейного уравнения / В.~В.~Алексеев, М.~М.~Преображенская // Теоретическая и математическая физика. --- 2024. --- Т. 220, № 2. --- С. 213--236.
    \end{itemize}
\end{frame}
\note{
    Результаты работы опубликованы в N печатных изданиях,
    в~т.\:ч. M реферируемых изданиях.
}

\begin{frame}
    \frametitle{Выступления на семинарах}
    \begin{itemize}
        \item Семинар по качественной теории дифференциальных уравнений в московском государственном университете имени М.В.~Ломоносова, 29 ноября 2024 года. \cite{Sergeev2024},\\\texttt{https://www.elibrary.ru/item.asp?id=75144298}
        \item Научный семинар лаборатории динамических систем и приложений НИУ ВШЭ в Нижнем Новгороде, 25 сентября 2024 г.,\\\texttt{https://nnov.hse.ru/bipm/dsa/semtmd}.
        \item Семинар по нелинейной динамике Ярославского государственного университета, 19 сентября 2024 г.,\\\texttt{https://cis.uniyar.ac.ru/index.php/event/460}.
    \end{itemize}
\end{frame}
\note{
    Работа была представлена на ряде конференций.
}

\begin{frame}
	\frametitle{Участие в конференциях}
	\begin{itemize}
		\item Конференция <<Integrable Systems and Nonlinear Dynamics>> (ISND – 2024), Ярославль, 2024.
		\item Конференция <<Topological Methods in Dynamics and Related Topics VII>>, Нижний Новгород, 2024.
		\item Международная конференция по дифференциальным уравнениям и динамическим системам DIFF-2024, Суздаль, 2024.
		\item Конференция <<Нелинейные дни в Саратове для молодых>>, Саратов, 2023.
		\item Конференция <<Satellite International Conference on Nonlinear Dynamics {\&} Integrability>>, Ярославль, 2022.
		\item Международная конференция по дифференциальным уравнениям и динамическим системам DIFF--2022, Суздаль, 2022.
	\end{itemize}
\end{frame}
\note{
	Работа была представлена на ряде конференций.
}

\begin{frame}[plain, noframenumbering] % последний слайд без оформления
    \begin{center}
        \Huge
        Спасибо за внимание!
    \end{center}
\end{frame}
