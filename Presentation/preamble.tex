\begin{frame}[noframenumbering,plain]
    \setcounter{framenumber}{1}
    \maketitle
\end{frame}

\section{Введение}

\begin{frame}
	\frametitle{Уравнение Мэки--Гласса}
	
	\begin{empheq}[box=\myeq]{equation*}
		\dot{u}=-\beta u + \frac{\alpha u(t - 1)}{1 + u^{\gamma}(t - 1)}, 
		\quad \alpha, \beta, \gamma > 0.
	\end{empheq}
	
	Если $u(t) > 0$ при $t \in [-1; 0]$, то $u(t) > 0$ при $t \in (0; + \infty)$.
	
	\pause
	\bigskip
	
	После экспоненциальной замены $u = e^x$:
	
	\begin{empheq}[box=\myeq]{equation*}
		\dot{x}=-\beta+\alpha\frac{e^{x(t-1)-x}}{1 + e^{\gamma x(t-1)}}.
	\end{empheq}
\end{frame}

\begin{frame}
	\frametitle{Релейное уравнение Мэки--Гласса}
	
	\begin{empheq}[box=\myeq]{equation*}
		\dot{x}=-\beta+\alpha e^{-x} F\left(\exp(x(t-1))\right),
	\end{empheq}
	
	\begin{empheq}[box=\myeq]{equation*}
		\text{где } F(u)=\lim\limits_{\gamma\to +\infty}\frac{u}{1+u^{\gamma}} = 
		\begin{cases}
			u, & u < 1;\\
			1/2, & u = 1;\\
			0, & u > 1.
		\end{cases}
	\end{empheq}
	
	\begin{figure}[ht]
		\centering
		\includegraphics[width=0.7\textwidth]{F_relay_plot_intro.eps}
	\end{figure}
\end{frame}

\begin{frame}
	\frametitle{Полносвязная сеть генераторов}
	
	Полносвязная сеть релейных генераторов Мэки--Гласса:
	
	\begin{equation}
		\dot{u}_j(t) = -\beta u_j(t) + \alpha F \left(u_j(t - 1) + \sum\limits_{k = 0, k\neq j}^N u_k(t)\right),
	\end{equation}
	где $j = 0, 1, \dots, N.$
	
	\begin{figure}
		\centering
		\includegraphics[width=0.5\textwidth]{full_mesh_u.eps}
	\end{figure}
	
\end{frame}

\begin{frame}
	\frametitle{Методы: переход к предельному объекту}
	
	До предельного перехода:
	\begin{equation*}
		\dot{u}=-\beta u + \frac{\alpha u(t - 1)}{1 + u^{\gamma}(t - 1)}, 
		\quad \alpha, \beta, \gamma > 0.
	\end{equation*}
	
	\pause
	
	После предельного перехода при $\gamma \to +\infty$:
	\begin{equation*}
		\dot{u}=-\beta u + \alpha u(t - 1) F(u(t - 1)), \text{ где}
	\end{equation*}
	
	\[
	F(u) = \begin{cases}
				1, & u < 1,\\
				1/2, & u = 1,\\
				0, & u > 1.
			\end{cases}
	\]
	
\end{frame}

\begin{frame}
	\frametitle{Методы: поиск дискретных бегущих волн.}
	
	В кольце:
	\begin{equation*}
		\dot{x}_j=\Phi(x_j, x_{j-1}), \quad j=1, \ldots, n, \quad x_{0} = x_{n},
	\end{equation*}
	где $\Phi:\mathbb{R}^2\to\mathbb{R}$. 
	
	\begin{equation*}
		x_j(t) = x(t + j\Delta).
	\end{equation*}
	
	\pause
	
	Вспомогательное уравнение:
	
	\begin{equation}
		\label{eq:intro:Phi_circ}
		\dot{x}=\Phi(x, x(t-\Delta)).
	\end{equation}
	
	Уравнение периодов:
	
	\begin{equation}
	\label{eq_period_Delta}
	p T(\Delta) = n\Delta, \quad p \in \mathbb{N}.
	\end{equation}
\end{frame}

\begin{frame}
	\frametitle{Методы: поиск дискретных бегущих волн}
	
	В полносвязной системе:
	\begin{equation*}
		\dot{x}_j= \Psi(x_j, x_{j-1}, \ldots, x_{j-n+1}), \quad j=1, \ldots, n, \quad x_j = x_{j+n}.
	\end{equation*}
	где $\Psi:\mathbb{R}^{n}\to\mathbb{R}$.
	
	Для некоторой нумерации осцилляторов:
	\begin{equation*}
		x_j(t) = x(t + j\Delta).
	\end{equation*}
	
	Вспомогательное уравнение:
	
	\begin{equation}
		\dot{x}= \Psi(x, x(t-\Delta), \ldots, x(t-n\Delta)).
	\end{equation}
	
	Уравнение периодов:
	
	\begin{equation}
		p T(\Delta) = n\Delta, \quad p \in \mathbb{N}.
	\end{equation}
\end{frame}

\begin{frame}
	\frametitle{Методы: поиск режимов кластерной синхронизации}
	
	В полносвязной системе:
	\begin{equation*}
		\dot{x}_j = \Xi(x_j;x_1,\ldots,\hat{x}_j,\ldots,x_{N}),\quad j=1,\ldots,N,
	\end{equation*}
	где $\Xi:\mathbb{R}^{N}\to\mathbb{R}$. Функция $\Xi$ симметрична относительно аргументов со 2-го по $N$-й.
	
	\pause
	\bigskip
	
	Пусть $N = m + n$,
	\begin{equation}
		x_1(t)=\ldots=x_n(t)=x(t),\quad x_{n+1}(t)=\ldots=x_{n+m}(t)=y(t).
	\end{equation}
	
	При этом $x(t)$ и $y(t)$ удовлетворяют системе
	\[
	\begin{cases}
		\dot{x}=\xi(x,y),\\
		\dot{y}=\eta(x,y),
	\end{cases}
	\]
	где 
	\[
	\xi(x,y)=\Xi(x; \underbrace{x, \ldots, x}_{n-1},\underbrace{y \ldots, y}_{m}),\quad
	\eta(x,y)=\Xi(y; \underbrace{x, \ldots, x}_{n},\underbrace{y \ldots, y}_{m-1}).
	\]
\end{frame}

\begin{frame}
	\frametitle{Методы: построение решений д.у. с разрывной правой частью}
	
	\begin{equation*}
		\dot{x} = f(x, t), \quad x \in \mathbb{R}^n.
	\end{equation*}
	
	Пусть $F(x, t) \subset \mathbb{R}^n$, причём если $f$ непр. в $(x, t)$, то $F(x, t) = \{f(x, t)\}$.
	
	Тогда $x(t)$ решение, если п. в. $\dot{x}(t) \in F(t, x)$.
\end{frame}


\begin{frame}
    \frametitle{Положения, выносимые на защиту}

    \begin{itemize}
    	\item Доказана теорема о существовании периодического решения уравнения Мэки--Гласса для достаточно больших значений (порядка $e^b$) отношения $a / b$.
    	\item Получены асимптотические формулы периодического решения уравнения Мэки--Гласса по параметру $\gamma \gg 1$ для достаточно больших значений (порядка $e^b$) отношения $a / b$.
    	\item Доказано существование периодических режимов в виде дискретной бегущей волны в полносвязной сети релейных генераторов Мэки--Гласса, сформулированы и доказаны условия их существования в виде ограничения на параметры соответствующей системы дифференциальных уравнений с запаздыванием.
    \end{itemize}
\end{frame}

\begin{frame}
	\frametitle{Положения, выносимые на защиту}
	
	\begin{itemize}
		\item Доказана теорема о эквивалентности системы с нелинейностью, включающей обе неизвестных функции, описывающей двухкластерную синхронизацию в полносвязной цепи осцилляторов Мэки--Гласса, и системы, нелинейные слагаемые которой содержат только одну неизвестную функцию.
		\item Доказана теорема о существовании (в смысле обобщённого решения системы дифференциальных уравнений с разрывной правой частью) периодических режимов двухкластерной синхронизации в полносвязной цепи релейных осцилляторов Мэки--Гласса, сформулированы и доказаны условия их существования в виде ограничения на параметры соответствующей системы дифференциальных уравнений с запаздыванием.
	\end{itemize}
\end{frame}
\note{
    Проговариваются вслух положения, выносимые на защиту
}

\begin{frame}
    \frametitle{Содержание}
    \tableofcontents
\end{frame}
\note{
    Работа состоит из трёх частей.

    \medskip
    В первой части рассматривается уравнение Мэки--Гласса. При некоторых ограничениях на параметры доказывается существование периодического решения. Приводятся асимптотические формулы периодического решения уравнения Мэки--Гласса по параметру $\gamma \gg 1$ для этого решения.
    
    \pause

    Во второй главе рассматривается полносвязная цепь осциллляторов Мэки--Гласса в предельном (релейном) случае. При некоторых ограничениях на параметры доказывается существование периодических режимов в виде дискретной бегущей волны.
    
    \pause

	В третьей главе рассматривается полносвязная цепь осциллляторов Мэки--Гласса. Доказывается, что возникающая при отыскивании периодических режимов двухкластерной синхронизации система, содержащая нелинейность, включающая обе компоненты решения, эквивалентна системе, нелинейные слагаемые которой содержат только одну компоненту решения. Для релейного варианта системы при некоторых ограничениях на параметры доказывается существование (в смысле обобщённого решения системы дифференциальных уравнений с разрывной правой частью) периодических режимов двухкластерной синхронизации.
}
