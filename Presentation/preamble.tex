\begin{frame}[noframenumbering,plain]
    \setcounter{framenumber}{1}
    \maketitle
\end{frame}

\section{Введение}

\begin{frame}
	\frametitle{Историческая справка}
	
	Mackey M., Glass L.\\
	\textbf{Oscillation and Chaos in Physiological Control Systems}\\
	Science, 1977.
	
	\begin{empheq}[box=\myeq]{equation*}
		\dot{v}=-b v+\frac{a \theta^{\gamma} v(t-\tau)}{\theta^{\gamma}+(v(t-\tau))^{\gamma}}, \quad a, b, \theta, \tau, \gamma > 0.
	\end{empheq}
	
	Модель описывает функцию кроветворения:\\
	$v(t)$ --- плотность нейтрофилов (вид лейкоцитов) в крови,\\
	$b$ --- скорость случайного распада нейтрофилов,\\
	$\tau$ --- запаздывание реакции организма в ответ на изменение плотности нейтрофилов.
\end{frame}

\begin{frame}
	\frametitle{Историческая справка}
	
	Krisztin T. (2020); Bartha F., Krisztin T., Vigh A. (2021) --- исследование устойчивости периодических решений уравнения Мэки--Гласса.
	
	\smallskip
	
	Кубышкин Е. П., Морякова А. Р. (2016) --- исследование периодических решений, бифурцирующих из состояния равновесия при изменении параметров уравнения.
	
	\smallskip
	
	Tateno M., Uchida A. (2012); Namajūnas et al. (1995) --- моделирование функционирования электрогенераторов, описываемых уравнением Мэки--Гласса.
	
	\smallskip
	
	Grassberger P., Procaccia I. (1983); Amil P. et al. (2015), Shahverdiev~E.~M.~et~al.~(2006) --- моделирование хаотического сигнала в системах уравнений типа Мэки--Гласса.	
\end{frame}

\begin{frame}
	\frametitle{Историческая справка}
	
	Обобщения уравнения Мэки--Гласса:
	
	\smallskip
	
	Liz E. et al. (2002) --- исследование сходимости решений к нулю для уравнения, полученного заменой нелинейности на функцию с более общими свойствами.
	
	\smallskip
	
	Berezansky L., Braverman E. (2006); Wu X.-M. et al. (2007) --- исследование периодических решений уравнения Мэки--Гласса с переменными коэффициентами и запаздыванием.
	
	\smallskip
	
	Huang X., Li Y. (2024) --- исследование стохастической модели Мэки--Гласса с несколькими запаздываниями.
	
	\smallskip
	
	Sano S. et al. (2007), Wan A., Wen J. (2009), Преображенская М.~М. (2020, 2021) --- исследование цепей электрических генераторов Мэки--Гласса.
\end{frame}

\begin{frame}
	\frametitle{Объекты исследования}
	
	Нормированное уравнение Мэки--Гласса (после замен $v(t) = \theta u\Big(\frac{t}{\tau}\Big)$, $\beta = b\tau$, $\alpha=a\tau$, $\frac{t}{\tau} \mapsto t$):
	
	\begin{empheq}[box=\myeq]{equation*}
		\dot{u}=-\beta u + \frac{\alpha u(t - 1)}{1 + u^{\gamma}(t - 1)}, 
		\quad \alpha, \beta, \gamma > 0.
	\end{empheq}
	
	\bigskip
	
	Если $u(t) > 0$ при $t \in [-1; 0]$, то $u(t) > 0$ при $t \in (0; + \infty)$.
	
%	\pause
%	\bigskip
%	
%	После экспоненциальной замены $u = e^x$:
%	
%	\begin{empheq}[box=\myeq]{equation*}
%		\dot{x}=-\beta+\alpha\frac{e^{x(t-1)-x}}{1 + e^{\gamma x(t-1)}}.
%	\end{empheq}
\end{frame}

\begin{frame}
	\frametitle{Объекты исследования}
	
	Релейное уравнение Мэки--Гласса:
	
	\begin{empheq}[box=\myeq]{equation*}
		\dot{u}=-\beta u + \alpha F(u),
	\end{empheq}
	
	\begin{empheq}[box=\myeq]{equation*}
		\text{где } F(u)=\lim\limits_{\gamma\to +\infty}\frac{u}{1+u^{\gamma}} = 
		\begin{cases}
			u, & 0 \leqslant u < 1;\\
			1/2, & u = 1;\\
			0, & u > 1.
		\end{cases}
	\end{empheq}
	
	\begin{figure}[ht]
		\centering
		\includegraphics[width=0.7\textwidth]{F_relay_plot_intro.eps}
	\end{figure}
\end{frame}

\begin{frame}
	\frametitle{Объекты исследования}
	
	Пусть параметры $\alpha, \beta > 0$. Полносвязная сеть из $N + 1$ релейных генераторов Мэки--Гласса задаётся системой:
	
	\small
	\begin{empheq}[box=\myeq]{equation*}
		\dot{u}_j(t) = -\beta u_j(t) + \alpha F \bigg(u_j(t - 1) + \sum\limits_{k = 0, k\neq j}^N u_k(t)\bigg), \quad j = 0, 1, \dots, N.
	\end{empheq}
	\normalsize
	
	\begin{figure}
		\centering
		\includegraphics[width=0.5\textwidth]{full_mesh_u.eps}
	\end{figure}
	
 \end{frame}

\begin{frame}
	\frametitle{Основные определения: дискретные бегущие волны}
	
	Пусть $u(t)$ --- $T$-периодическая функция. \emph{Дискретная бегущая волна} --- решение системы, которое имеет следующий вид:
	\begin{empheq}[box=\myeq]{equation*}
		u_j(t) = u(t - j \Delta), \quad j = 0, 1, \dots N
	\end{empheq}
	
	и удовлетворяет условию на период: при некотором натуральном $p$ выполнено
	\begin{empheq}[box=\myeq]{equation*}
		pT(\Delta) = \Delta (N + 1).
	\end{empheq}
	
	\begin{figure}
		\centering
		\includegraphics[width=\textwidth]{discrete_waves_plot.eps}
	\end{figure}
\end{frame}

\begin{frame}
	\frametitle{Основные определения: двухкластерная синхронизация}
	
	Пусть параметры $\alpha, \beta, \delta > 0$. Полносвязная сеть из $N = m + n$ релейных генераторов Мэки--Гласса описывается системой
	\small	
	\begin{empheq}[box=\myeq]{equation*}
		\dot{u}_j(t) = -\beta u_j(t) + \alpha F \bigg(u_j(t - 1) + \delta \sum\limits_{k = 0, k\neq j}^N u_k(t)\bigg), \ j = 1, \dots, N.
	\end{empheq}
	\normalsize
	
	\begin{wrapfigure}{r}{0.4\textwidth} 
		\vspace{-22pt} 
		\centering
		\includegraphics[width=0.5\textwidth]{two_cluster_uv.eps}
	\end{wrapfigure}
	
	Периодическое решение системы, \\которое имеет вид 
	\small	
	\begin{empheq}[box=\myeq]{multline*}
		u_1(t)=\ldots=u_m(t) = u(t),\\u_{m+1}(t)=\ldots=u_{m+n}(t) = v(t)
	\end{empheq}
	\normalsize
	называется \emph{режимом двухкластерной синхронизации}.

\end{frame}

\begin{frame}
    \frametitle{Положения, выносимые на защиту}

	\small
    \begin{itemize}
    	\item Получены асимптотические формулы периодического решения уравнения Мэки--Гласса по параметру $\gamma \gg 1$ при ограничении на параметры $\alpha > \exp\left(\beta(1 + e^{-\beta})\right)$ [1, Теорема 5.6]. На основе полученных формул доказана теорема о существовании периодического решения уравнения Мэки--Гласса при соответствующих ограничениях на параметры [1, Теорема 3.2].
    	\item Доказано существование периодических режимов в виде дискретной бегущей волны в полносвязной цепи релейных генераторов Мэки--Гласса, сформулированы и доказаны достаточные условия их существования в виде ограничения на параметры соответствующей системы дифференциальных уравнений с запаздыванием [2, Теорема 16].
    	\item Доказана теорема о существовании (в смысле обобщённого решения системы дифференциальных уравнений с разрывной правой частью) периодических режимов двухкластерной синхронизации в полносвязной цепи релейных осцилляторов Мэки--Гласса, сформулированы и доказаны достаточные условия их существования в виде ограничения на параметры соответствующей системы дифференциальных уравнений с запаздыванием [3, Теорема 5.2].
    \end{itemize}
    \normalsize
\end{frame}

\note{
    Проговариваются вслух положения, выносимые на защиту
}

\begin{frame}
    \frametitle{Содержание}
    \tableofcontents
\end{frame}
\note{
    Работа состоит из трёх частей.

    \medskip
    В первой части рассматривается уравнение Мэки--Гласса. При некоторых ограничениях на параметры доказывается существование периодического решения. Приводятся асимптотические формулы периодического решения уравнения Мэки--Гласса по параметру $\gamma \gg 1$ для этого решения.
    
    \pause

    Во второй главе рассматривается полносвязная сеть осциллляторов Мэки--Гласса в предельном (релейном) случае. При некоторых ограничениях на параметры доказывается существование периодических режимов в виде дискретной бегущей волны.
    
    \pause

	В третьей главе рассматривается полносвязная сеть осциллляторов Мэки--Гласса. Для релейного варианта системы при некоторых ограничениях на параметры доказывается существование (в смысле обобщённого решения системы дифференциальных уравнений с разрывной правой частью) периодических режимов двухкластерной синхронизации.
}
