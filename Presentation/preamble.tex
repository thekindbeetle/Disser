\begin{frame}[noframenumbering,plain]
    \setcounter{framenumber}{1}
    \maketitle
\end{frame}

\begin{frame}
    \frametitle{Положения, выносимые на защиту}
    \begin{itemize}
        \item Построение асимптотики для периодического решения уравнения Мэки--Гласса при некоторых ограничениях на параметры уравнения.
        \item Доказательство существования и явное аналитическое описание периодических режимов в орме дискретной бегущей волны в полносвязной цепи осцилляторов Мэки--Гласса.
        \item Доказательство существования периодических режимов двухкластерной синхронизации в полносвязной цепи осцилляторов Мэки--Гласса.
    \end{itemize}
\end{frame}
\note{
    Проговариваются вслух положения, выносимые на защиту
}

\begin{frame}
    \frametitle{Содержание}
    \tableofcontents
\end{frame}
\note{
    Работа состоит из трёх частей.

    \medskip
    В первой части рассматривается уравнение Мэки--Гласса. При некоторых ограничениях на параметры доказывается существование периодического решения определённого вида.  \dots

    Во второй главе рассматривается полносвязная цепь осциллляторов Мэки--Гласса в предельном (релейном) случае. При некоторых ограничениях на параметры доказывается существование периодических режимов в виде дискретной бегущей волны.

	В третьей главе также рассматривается полносвязная цепь осциллляторов Мэки--Гласса в предельном (релейном) случае. При некоторых ограничениях на параметры доказывается существование периодических режимов двухкластерной синхронизации --- режимов, для которых осцилляторы разбиваются на две группы, функционирование которых определяется одной и той же (для каждой группы) периодической функцией.
}
