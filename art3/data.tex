
\newcommand{\articleUDK}{656.073, 004.934}

\newcommand{\articleTitleUkr}{
Модель голосової взаємодії водія в системах диспетчерського контролю за рухом автотранспорту}

\newcommand{\articleTitleRus}{
Модель голосового взаимодействия водителя в системах диспетчерского контроля за движением автотранспорта}

\newcommand{\articleTitleEng}{
Model of voice interaction of the driver in the systems of dispatching control over the movement of vehicles}

\newcommand{\annotationUkr}{
У роботі запропоновано та розроблено модель голосової взаємодії водія в системах диспетчерського контролю за рухом автотранспорту. Рефлекторна модель розпізнання побудована за аналогією зі структурою умовного рефлексу, в якому виділяються афектори, центральний компонент та ефектори і поєднана з ідеєю використання дерева сценаріїв, оскільки сценарії також складаються із реакцій, і одиницею моделювання стає не лінгвістична особливість мовлення, а реакція (або команда), яка може бути врахована автоматизованою системою підрахунку маршрутів. У результаті запропоновано повне дерево сценаріїв усіх етапів дистрибуції «склад – дорога – точка доставки» з вказівкою контекстів і з включенням можливих реакцій в них, тобто на кожний позначений контекст існує реакція.
}

\newcommand{\annotationRus}{
В работе предложена и разработана модель голосового взаимодействия водителя в системах диспетчерского контроля за движением автотранспорта. Рефлекторная модель распознавания построена по аналогии со структурой условного рефлекса, в котором выделяются афекторы, центральный компонент и эффекторы и сопряжена с идеей использования дерева сценариев, поскольку сценарии также состоят из реакций, и единицей моделирования становится не лингвистическая особенность речи, а реакция (или команда), которая может быть учтена автоматизированной системой подсчета маршрутов. В результате предложено полное дерево сценариев всех этапов дистрибуции «склад – дорога – точка доставки» с указанием контекстов и с включением возможных реакций в них, то есть на каждый обозначенный контекст существует реакция.
}

\newcommand{\annotationEng}{
In the work, a model of driver voice interaction in the systems of dispatching control over the movement of vehicles has been proposed and developed. The reflex pattern of recognition is built by analogy with the structure of the conditioned reflex, in which there are selectors, the central component and effectors, and is coupled with the idea of using the scenario tree, because the scenarios also consist of reactions, and the reaction (or command) becomes the unit of modeling which can be taken into account by an automated route calculation system. As a result, a complete scenario tree of all distribution stages “depot - road - delivery point” was proposed, indicating contexts and including possible reactions to them, that is, there is a reaction to each designated context.
}

\newcommand{\keywordsUkr}{
модель голосової взаємодії, дерево сценаріїв, диспетчерський контроль, дистрибуція, склад, дорога, точка доставки}

\newcommand{\keywordsRus}{
модель голосового взаимодействия, дерево сценариев, диспетчерский контроль, дистрибуция, склад, дорога, точка доставки}

\newcommand{\keywordsEng}{
voice interaction model, scenario tree, dispatch control, distribution, depot, road, point of delivery}
