\textbf{Объект исследования.} Объектом исследования диссертационной работы являются дифференциальные уравнения с запаздывающим аргументом, или дифференциально-разностные уравнения, --- дифференциальные уравнения, в которые неизвестная функция и её производные входят при разных значениях аргумента.

Простейшее уравнение с запаздыванием имеет вид\footnote{В работе приняты следующие соглашения: явно указывается только запаздывающий аргумент функции (так, например, вместо $x(t)$ пишем $x$). Дифференцирование по времени обозначается точкой над функцией.}
\begin{equation}
	\label{eq:delay_equation}
	\dot{x} = f(t, x, x(t - \tau)),
\end{equation}
где $x$ --- скалярная функция или вектор-функция, $\tau > 0$ --- запаздывание.

Основная начальная задача для уравнения \eqref{eq:delay_equation} заключается в определении непрерывного решения при $t > t_0$ при условии, что $x = \phi(t)$ при $t_0 - \tau \leq t \leq t_0$, где $\phi(t)$ --- заданная непрерывная функция, называемая начальной.

В случае уравнения с несколькими запаздываниями
\begin{equation}
	\label{eq:multiple_delay_equation}
	\dot{x} = f(t, x, x(t - \tau_1), \dots, x(t - \tau_m)), \quad \tau_i > 0, \ i = 1, \ldots, m.
\end{equation}
начальная функция задаётся на промежутке длины наибольшего запаздывания.

Специфика исследования дифференциальных уравнений с запаздыванием заключается в том, что для корректной постановки основной начальной задачи нужно знать значение функции не в одной точке (как для ОДУ), а на промежутке длины запаздывания. Таким образом, элементом фазового пространства является не точка в евклидовом пространстве, а функция, и для исследования свойств решений используются методы функционального анализа. Объемлющее введение в теорию дифференциальных уравнений с запаздывающим аргументом дано в книге \cite{Elsgoltz}.

При исследовании свойств конкретного уравнения с запаздыванием, его полное описание, т.~е. аналитическое описание \emph{всех} решений (для произвольных начальных функций и значений параметров модели) не представляется возможным, и исследование может пойти по пути поиска решений какого-то конкретного вида. Это может быть задача поиска периодических решений, или решений, близких к периодическим. С другой стороны, это может быть задача поиска хаотических решений и условий, при которых они возникают. В диссертационной работе избран первый путь --- производится описание периодических решений и условий (ограничений на начальные функции и параметры модели), при которых эти решения возникают.

В первой части диссертации проводится исследование методом большого параметра для модели Мэки--Гласса \cite{Mackey1977, Glass1988}:
\begin{equation}
	\label{eq:mg_equation_1:intro}
	\dot{v}=-b v+\frac{a \theta^{\gamma} v(t-\tau)}{\theta^{\gamma}+(v(t-\tau))^{\gamma}},
\end{equation}
где параметры $a, b, \gamma, \theta$ --- положительные вещественные числа.

Вторая и третья части посвящены поиску периодических решений для систем дифференциальных уравнений с запаздыванием, являющихся обобщением модели Мэки--Гласса.

{\methods} В работе использованы следующие методы исследования уравнений с запаздыванием.

\textit{Метод шагов.} Рассмотрим основную начальную задачу для уравнения~\eqref{eq:delay_equation}, где $\tau > 0$, $x = \phi_0(t)$ при $t_0 - \tau \leq t \leq t_0$.

Решение $x(t)$ рассматриваемой задачи на промежутке $t_0 \leq t \leq t_0 + \tau$ определяется из начальной задачи Коши для дифференциального уравнения без запаздывания
%
\[
\dot{x} = f(t, x, \phi_0(t - \tau)),\quad x(t_0) = \phi_0(t_0),
\]
%
так как при $t_0 \leq t \leq t_0 + \tau$ аргумент $t - \tau$ изменяется на множестве $[t_0 - \tau, t_0]$ и, следовательно, $x(t - \tau) = \phi_0(t - \tau)$.

Метод шагов, или метод последовательного интегрирования, заключается в следующем. Предположим, что решение начальной задачи $x = \phi_1(t)$ существует на всём отрезке $[t_0, t_0 + \tau]$. Можно этот отрезок рассмотреть в качестве начального множества, а решение $\phi_1(t)$ как начальную функцию, и аналогичным образом продлить решение на отрезок $[t_0 + \tau, t_0 + 2\tau]$. Аналогичным образом, <<шагами>> длины $\tau$ можно продлить решение на произвольный промежуток \cite{Elsgoltz}.

Для уравнения \eqref{eq:multiple_delay_equation} метод шагов работает аналогично, и решение восстанавливается шагами длины наименьшего запаздывания.

Для сложных моделей в случае, когда не удается отыскать решение непосредственным интегрированием, уместно использование специальных методов поиска решения. Одна из идей упрощения исследования --- это переход к предельному объекту.

\textit{Метод большого параметра.} При наличии в модели большого параметра $\lambda \gg 1$ часто оказывается удобным рассмотреть предельный объект при $\lambda \to +\infty$. Этот объект может обладать свойствами исходной модели, но быть проще в исследовании. В таком случае исследование исходной модели сводится к исследованию предельной модели и последующего доказательства асимптотической близости её решения к решению исходной модели. Такой подход называется \emph{методом большого параметра}, и был развит в работах С.~А.~Кащенко, Ю.~С.~Колесова и др. \cite{Kashchenko1982, Kashchenko1983, KolesovKolesov1993, Kolesov2010}. Изложим коротко его суть.

Рассмотрим дифференциальное уравнение с параметром $\gamma \gg 1$
\[
\dot{x} = f_{\gamma}(t, x, x(t - \tau))
\]
Для упрощения исследования необходимо подобрать такую замену, чтобы при стремлении большого параметра $\gamma \to +\infty$ получилось предельное уравнение, у которого, с одной стороны, имеется достаточно сложная динамика (периодические решения различной структуры), а с другой стороны, при стремлении $\gamma \to +\infty$ можно было доказать сходимость решения исходного уравнения к периодическому решению предельной задачи. В частности, это получается сделать, если нелинейность в правой части уравнения близка к сигмоидальной функции, например, имеет вид $S_\gamma(x)=\frac{1}{1 + x^\gamma}$, $x > 0$ \cite{Preobrazhenskaya2020, Glyzin2017, Krisztin2020, Bartha2021}. Такая функция в пределе при $\gamma\to+\infty$ устремляется к кусочно-постоянной функции, меняющей значение в 1.

В этом случае (и вообще, когда правая часть уравнения имеет разрыв первого рода) уравнение, получаемое при предельном переходе, называют \emph{релейным}. Тогда исходное уравнение можно заменить релейным, которое, как правило, существенно проще для анализа. После построения решения релейного уравнения доказывается существование асимптотически близкого решения исходной задачи.

Методология подобного исследования описана в работах А.~Ю.~Колесова, Ю.~С.~Колесова, Е.~Ф.~Мищенко и Н.~Х.~Розова \cite{KolesovKolesov1993, Kolesov2010}.

Так, для исследуемого в первой части уравнения \eqref{eq:mg_equation_1:intro} после перенормировки времени, замены параметров и неизвестной функции 
\begin{equation}
	\label{eq:intro_substitutions_v}
	v(t) = \theta u\Big(\frac{t}{\tau}\Big),\ \beta = b\tau,\ \alpha=a\tau, \ \frac{t}{\tau} \mapsto t,
\end{equation}
а также последующей экспоненциальной замены $u = e^x$, получаем уравнение
\begin{equation}
	\label{eq:intro:MG_norm1}
	\dot{x}=-\beta+\alpha\frac{e^{x(t-1)-x}}{1+e^{\gamma x(t-1)}}.
\end{equation}
Здесь мы полагаем $\gamma$ большим параметром. Соответствующее (при $\gamma \to +\infty$) релейное уравнение имеет вид
\begin{equation}
	\label{eq:intro:MG_norm_relay}
	\dot{x}=-\beta + \alpha e^{-x} F(\exp({x(t-1)})),
\end{equation}
где функция $F$ (см. рис. \ref{fig:F_relay_plot:intro}) задаётся формулой
\begin{equation}
	\label{eq:intro:F_relay}
	F(u)=\lim\limits_{\gamma\to +\infty}\frac{u}{1+u^{\gamma}} = 
	\begin{cases}
		u, & 0 \leq u < 1,\\
		\frac{1}{2}, & u = 1,\\
		0, & u > 1.
	\end{cases}
\end{equation}

\begin{figure}[ht]
	\centering
	\includegraphics[width=0.7\textwidth]{F_relay_plot_intro.eps}
	\caption{Релейная функция $F(u)$, задаваемая формулой \eqref{eq:intro:F_relay}.}
	\label{fig:F_relay_plot:intro}
\end{figure}

Другая идея состоит в том, чтобы найти решение в специальном виде. Остановимся на описании этой процедуры подробнее.

\textit{Поиск решений специального вида.} В настоящей работе используются два подхода к построению специальных решений системы дифференциально-разностных уравнений, разработанные в работах С.~Д.~Глызина и др. \cite{GlyKol2013, GlyKol2013a, Glyzin2014}: это построение дискретных бегущих волн (вторая глава диссертации) и поиск режимов кластерной синхронизации (третья глава диссертации). Опишем метод построения решений того и другого сорта в общем виде.

\textit{Дискретные бегущие волны.} Рассмотрим симметричную систему осцилляторов, связанных либо в кольцо, либо в полносвязную сеть. Дискретной бегущей волной называют периодическое решение, все компоненты которого представлены одной и той же периодической функцией $u(t)$ со сдвигом по времени, кратным некоторому параметру $\Delta$.

Для полносвязной сети релейных осцилляторов Мэки--Гласса, рассматриваемой во второй главе диссертации, соответствующая система имеет вид
%
\begin{equation}
	\label{eq:intro:mg_full_renormed}
	\dot{u}_j(t) = -\beta u_j(t) + \alpha F \left(u_j(t - 1) + \sum\limits_{k = 0, k\neq j}^N u_k(t)\right), \text{ где } j = 0, 1, \dots, N,
\end{equation}
функция $F$ (как и в первой части) задаётся формулой \eqref{eq:intro:F_relay}.

После подстановки $u_j(t) = u(t + j\Delta)$ в систему \eqref{eq:intro:mg_full_renormed} получаем вспомогательное уравнение

\begin{equation}
	\label{eq:intro:mg_auxiliary}
	\dot{u}(t) =-\beta u(t) + \alpha F\left(u(t - 1) + \sum_{s=1}^{N}u(t-s\Delta)\right).
\end{equation}

Дискретной бегущей волне соответствует периодическое решение уравнения \eqref{eq:intro:mg_auxiliary}, период (не обязательно главный) которого кратен параметру $\Delta$. Отметим, что из существования одного решения в виде дискретной бегущей волны следует одновременное существование сразу $N!$ таких режимов, получаемых перестановкой компонент исходного решения, где $N$ --- количество уравнений в системе.

\textit{Режимы кластерной синхронизации.} 
В третьей главе диссертации рассматривается система из $N = n + m$ уравнений
\begin{equation}
	\label{eq:intro:mg_full_renormed_delta}
	\dot{u}_j(t) = -\beta u_j(t) + \alpha F_{\gamma} \left(u_j(t - 1) + \delta\sum\limits_{k = 1, k\neq j}^N u_k(t)\right), \text{ где } j = 1, \dots, N,
\end{equation}
где
\[
F_{\gamma}(u) = \dfrac{u}{1 + u^{\gamma}},
\]
с коэффициентом $\delta > 1$ в обратной связи. Решение вида 
\begin{equation}
	\label{eq:intro:cluster}
	u_1(t)=\ldots=u_m(t) = u(t),\quad u_{m+1}(t)=\ldots=u_{m+n}(t) = v(t),
\end{equation}
при котором часть осцилляторов описывается одной функцией, а остальные --- другой, называется режимом двухкластерной синхронизации. После подстановки \eqref{eq:intro:cluster} в систему \eqref{eq:intro:mg_full_renormed_delta} получается система из двух уравнений
%
\begin{equation}
	\label{eq:intro:system_uv}
	\begin{cases}
		\dot{u} = -\beta u + \alpha \, F_{\gamma} \big(u(t - 1) + \delta (m - 1) u + \delta n v\big),\\
		\dot{v} = -\beta v + \alpha \, F_{\gamma} \big(v(t - 1) + \delta m u + \delta (n - 1) v\big),
	\end{cases}
\end{equation}
%
для которой ищется периодическое решение.

Режимы двухкластерной синхронизации строятся в работах \cite{Glyzin2016a, Glyzin2022}.

\textit{Дифференциальные уравнения с разрывной правой частью.} Рассмотрим дифференциальное уравнение 
\[
\dot{x} = f_{\gamma}(t, x),
\]
где $\gamma$ --- вещественный параметр. При переходе к предельному при $\gamma \to +\infty$ уравнению
\begin{equation}
	\label{eq:intro:equiv_equation_initial}
	\dot{x} = \lim\limits_{\gamma \to +\infty}f_{\gamma}(t, x) = f(t, x),
\end{equation}
правая часть может стать разрывной функцией. При этом возникает необходимость обобщить понятие решения уравнения так, чтобы оно удовлетворяло следующим естественным требованиям.
\begin{enumerate}
	\item Для дифференциальных уравнений с непрерывной правой частью определение решения должно быть равносильно обычному.
	\item Для уравнения $\dot{x} = f(t)$ решениями (в обобщённом смысле) должны быть функции $x(t) = \int f(t)\, dt + c$ (и только они).
\end{enumerate}

Наиболее известные определения обобщённого решения уравнения \eqref{eq:intro:equiv_equation_initial} излагаются в книге \cite[\S 4]{Filippov1988}. Приведём два из них.

Пусть дано уравнение \eqref{eq:intro:equiv_equation_initial}, где функция $f$ кусочно непрерывна в области $G \subset \mathbb{R} \times \mathbb{R}^n$, $x \in \mathbb{R}^n$, $M$ --- множество точек разрыва функции $f$, состоящее из конечного числа гиперповерхностей в $\mathbb{R}^n$.

\emph{Простейшее выпуклое доопределение \cite{Filippov1988}.} Для каждой точки $(t, x) \in G$ укажем множество $\mathcal{F}(t, x) \subset \mathbb{R}^n$. Если в точке $(t, x)$ функция $f$ непрерывна, то $\mathcal{F}(t, x)$ состоит в точности из этой точки. Если же $f$ разрывна в точке $(t, x)$, то множество $\mathcal{F}(t, x)$ определяется как выпуклая оболочка всех предельных значений функции $f(t, x^*)$, где $x^* \not\in M$, при $x^* \to x$. Решением уравнения \eqref{eq:intro:equiv_equation_initial} называется абсолютно непрерывная функция $x(t)$, определённая на промежутке $I$, для которой почти всюду на $I$ верно включение $\dot{x} \in \mathcal{F}(t, x)$.

\emph{Метод эквивалентного управления \cite{Utkin1981}.} Пусть дано уравнение 
\begin{equation}
	\label{eq:intro:equiv_equation_initial_2}
	\dot{x} = f(t, x, u_1(t, x), \ldots, u_r(t, x)),
\end{equation}
где $x \in \mathbb{R}^n$, функция $f$ непрерывна, а функции $u_i(t, x)$ разрывны на множествах $M_i$, $i = 1, \ldots, r$. В каждой точке $(t, x)$ разрыва функции $u_i$ укажем множество $U_i(t, x)$ --- множество возможных значений аргумента $u_i$ функции $f$. В точках, где $u_i(t, x)$ непрерывна, множество $U_i(t, x)$ состоит из одного значения $u_i(t, x)$. В точках разрыва функции $u_i(t, x)$ множество $U_i(t, x)$ определяется как некоторое замкнутое множество, содержащее все предельные точки функции $u_i(t, x^*)$ при $x^* \to x$. Для каждой точки $t, x$ определим множество
\[
\mathcal{F}(t, x) = f(t, x, U_1(t, x), \ldots, U_r(t, x))
\]
--- множество значений функции $f(t, x, u_1, \ldots, u_r)$, когда $t$, $x$ постоянны, а $u_1, \ldots, u_r$ независимо друг от друга пробегают соответственно множества $U_1(t, x), \ldots, U_r(t, x)$. Функция $x(t)$ называется решением уравнения \eqref{eq:intro:equiv_equation_initial_2}, если почти при всех $t$ верно включение $\dot{x}(t) \in \mathcal{F}(t, x)$. В случае, когда $U_i(t, x)$ определяется как выпуклая оболочка предельных точек, данный метод эквивалентен методу выпуклого доопределения \cite{Filippov1988}.

В третьей части диссертации используется доопределение методом эквивалентного управления.

% common/newnames.tex & common/renames.tex
% ВЫСОКО ВЫСОКО В ГОРАХ ЖИЛ МАЛЕНЬКИЙ ПТИЧЬКА
{\actuality} Биологические модели занимают важное место в теории динамических систем, охватывая широкий спектр явлений, происходящих в биологических процессах. Эти модели включают как те, которые описывают концентрацию различных веществ в биологических системах, таких как химические реакции в клетках или процессы обмена веществ в организмах, так и популяционные уравнения, моделирующие изменение численности и структуру популяций живых существ. Обычно модели популяционной динамики описываются дифференциальными или дифференциально-разностными уравнениями. 
Эти модели с течением времени претерпели значительные изменения и усложнения.

Одной из первых популяционных моделей была модель Томаса Мальтуса, который предположил, что рост популяции, не сдерживаемой ограничениями на ресурсы, будет экспоненциальным. % Массовые вымирания, такие как эпидемии, голод и войны, в модели Мальтуса объясняются тем, что ресурсы, необходимые для выживания, не могут воспроизводиться с такой же скоростью, как и популяция, что приводит к перенаселению \cite{Malthus1798}.

Пьер Ферхюльст в работе \cite{Verhulst1838} предложил добавить в модель Мальтуса квадратичное слагаемое, ограничивающее скорость роста популяции для больших значений её размера. Таким образом, он получил модель, называемую логистическим уравнением
\begin{equation}
	\label{eq:intro:logistic}
	\dot{x}=\lambda x\left(1-\frac{x}{K}\right),
\end{equation}
где функция $x(t)$ показывает текущую плотность (или численность) популяции, параметр $\lambda$ характеризует скорость роста популяции, а параметр $K$ --- ёмкость среды.

Позже рассматривались различные усовершенствования приведённой модели. Одна из идей усложнения правой части уравнения \eqref{eq:intro:logistic} --- добавление запаздывания\footnote{В работе приняты следующие соглашения: явно указывается только запаздывающий аргумент функции: так, например, вместо $x(t)$ пишем $x$. Дифференцирование по времени обозначается точкой над функцией.}. Так в 1948 году Джордж Хатчинсон предложил модификацию логистического уравнения, обладающую запаздыванием по времени и простейшим образом учитывающую возрастную структуру популяции:
%
\begin{equation}
	\label{eq:intro:hutch}
	\dot{x}=\lambda x\left(1 - \frac{x(t-\tau)}{K}\right),
\end{equation}
%
где $x(t)$ --- плотность популяции, $\lambda$ --- скорость роста популяции, $K$ --- ёмкость среды, запаздывание $\tau$ --- возраст половозрелости \cite{Hutchinson1948}. %https://encyclopediaofmath.org/wiki/Hutchinson_equation

Помимо уравнения Хатчинсона, рассматривались различные его обобщения с заменой квадратичной зависимости в правой части на более сложную. Например, в работе \cite{Glyzin2007} предлагается модель, более тонко учитывающая возрастную структуру
\begin{equation}
	\label{eq:intro:glyzin2007}
	\dot{x}=\lambda \left(1 - \frac{1}{K}\sum\limits_{i = 1}^{m} a_i x(t-\tau_i)\right) x.
\end{equation}
%
В работе \cite{Kaschenko2012} исследуется обобщение модели \eqref{eq:intro:hutch}
\begin{equation}
	\dot{x} = \lambda \left(1 - \int\limits_{h_1}^{h_2}dr(\tau)x(t - \tau)\right) x,
\end{equation}
где $r(\tau)$ --- монотонная неотрицательная функция, $\lambda, h_1, h_2$ --- положительные параметры.

В работе \cite{Kolesov2010} методом большого параметра исследуется обобщённое уравнение Хатчинсона
\begin{equation}
	\label{eq:intro:hutch_modified}
	\dot{x} = \lambda f(x(t - 1)) x,
\end{equation}
%
где функция $f$ обладает свойствами
\[
f(0) = 1, \quad f(x) = -a_0 + \sum\limits_{k = 1}^{\infty} \frac{a_k}{x^k}, \quad x \to +\infty, \quad a_0 > 0.
\]
После экспоненциальной замены $x = e^{\lambda y}$ уравнение \eqref{eq:intro:hutch_modified} принимает вид
\begin{equation}
	\label{eq:intro:hutch_modified_exp}
	\dot{y} = f(e^{\lambda y(t - 1)}).
\end{equation}
В качестве множества начальных функций рассматривается множество
\begin{multline}
	\label{eq:intro:hutch_init_func}
	\phi \in C[-1 - \sigma_0, -\sigma_0],\quad \phi(t) < 0 \quad \forall t \in [-1 - \sigma_0, -\sigma_0], \\ \phi(-\sigma_0) = -\sigma_0, \ 0 < \sigma_0 < a_0.
\end{multline}
При достаточно большом $\lambda$ правая часть становится близкой к кусочно постоянной (<<релейной>>) функции $R(y)$, меняющей значение при смене знака аргумента:
\[
R(y) = \begin{cases}
	-a_0, & y > 0,\\
	1, & y < 0.
\end{cases}
\]
Предельное для \eqref{eq:intro:hutch_modified_exp} уравнение имеет вид
\begin{equation}
	\label{eq:hutch_relay}
	\dot{y} = R(y(t - 1)),
\end{equation}
решением которого при $t > \sigma_0$ является кусочно линейная периодическая функция (см. рис. \ref{fig:hutch_relax})
\begin{equation}
	\label{eq:hutch_relay_solution}
	x_0(t)=\left\{\begin{array}{l}
		t \text { при } 0 \leq t \leq 1, \\
		1-a_0(t-1) \quad \text { при } \quad 1 \leq t \leq t_0 + 1, \\
		-a_0 + t - t_0 - 1 \quad \text { при } t_0 + 1 \leq t \leq T_0,
	\end{array} \quad x_0\left(t+T_0\right) \equiv x_0(t) .\right.
\end{equation}
Затем для достаточно больших $\lambda$ доказывается существование периодического решения уравнения \eqref{eq:intro:hutch_modified_exp}, близкого к решению \eqref{eq:hutch_relay_solution} \cite{Kolesov2010}.

\begin{figure}
	\centering
	\includegraphics[width=0.7\textwidth]{hutch_relay_sol.png}
	\caption{Решение уравнения \eqref{eq:hutch_relay}. Рисунок взят из статьи \cite{Kolesov2010}.}
	\label{fig:hutch_relax}
\end{figure}

Доказано, что при всех достаточно больших $\lambda > 0$ уравнение \eqref{eq:intro:hutch_modified_exp} имеет асимптотически орбитально  устойчивый цикл\footnote{Предельный цикл $\xi$ называется орбитально устойчивым, если для всякого $\varepsilon > 0$ найдётся $\delta > 0$ такое, что всякая положительная полутраектория, начинающаяся в $\delta$-окрестности цикла $\xi$, содержится в $\varepsilon$-окрестности $\xi$. Предельный цикл $\xi$ называется асимптотически орбитально устойчивым, если он является орбитально устойчивым и, кроме того, найдётся $\delta_0 > 0$ такое, что траектория всякого решения $x(t)$, начинающегося в $\delta_0$-окрестности цикла $\xi$, стремится при $t \to +\infty$ к $\xi$, то есть
\[
\lim\limits_{t\to +\infty} d(x(t), \xi) = 0, \quad \text{где} \quad d(x, \xi) = \inf_{y\in \xi} \Vert x - y \Vert.
\]
--- расстояние от точки $x$ до множества $\xi$ \cite{MathEncyclopedy1984}.} $x_*(t, \lambda)$, $x_*(-\sigma_0, \lambda) = -\sigma_0$ периода $T_*(\lambda)$, удовлетворяющий предельным равенствам
%
$$
\lim _{\lambda \rightarrow +\infty} \max_t \left|x_*(t, \lambda) - x_0(t)\right|=0, \quad \lim _{\lambda \rightarrow 0} T_*(\lambda) = T_0.
$$

В диссертационной работе исследуется уравнение Мэки--Гласса \eqref{eq:mg_equation_1:intro}. Эта модель была предложена в работе \cite{Mackey1977} для описания регуляторных функций в процессах кроветворения. На рисунке \ref{fig:mg_delay_form} показан вид нелинейного слагаемого уравнения.

\begin{figure}
	\centering
	\includegraphics[width=0.5\textwidth]{mg_delay_form.eps}
	\caption{График нелинейного слагаемого в уравнении \eqref{eq:mg_equation_1:intro} (как функции от переменной $v(t - \tau)$).
	}
	\label{fig:mg_delay_form}
\end{figure}

Биологический смысл данной модели следующий: функция $v(t)$ --- плотность циркулирующих в крови человека нейтрофилов (вид лейкоцитов) в клетках на кг массы тела, $b$ --- скорость случайного распада нейтрофилов, положительное слагаемое означает текущий приток клеток в кровь, возникающий в ответ на запрос, создавшийся в некоторый момент $\tau$ времени назад в прошлом.

В широком диапазоне изменений уровня циркулирующих нейтрофилов скорость образования нейтрофилов падает с увеличением их плотности. Однако благодаря действию различных факторов можно ожидать, что при очень низких уровнях нейтрофилов скорость их образования будет падать, приближаясь к нулю \cite[с. 85]{Mackey1977}. Форма нелинейности выбирается из приведённых соображений.

Уравнение Мэки--Гласса исследовалось во множестве работ: например, см.~ \cite{Junges2012, Su2011, Wu2007, Kubyshkin2016, Krisztin2020, Bartha2021}, а также статью \cite{Berezansky2012}, в которой представлен обширный обзор различных известных (к 2012 г.) результатов, связанных с исследованием уравнения Мэки--Гласса и его обобщений, со ссылками на соответствующие работы. 

В оригинальной работе \cite{Mackey1977} приведены численные решения, как в случае периодических, так и непериодических колебаний. В работах \cite{Krisztin2020, Bartha2021} аналитически и численно исследуются периодические решения уравнения Мэки--Гласса. В частности, в \cite{Bartha2021} доказано (при некоторых ограничениях на параметры и множество начальных функций) существование и единственность орбитально устойчивого предельного цикла. В работе \cite{Kubyshkin2016} изучаются периодические решения уравнения Мэки--Гласса, бифурцирующие из его единственного состояния равновесия при изменении параметров уравнения.

Уравнение Мэки--Гласса и его различные модификации использовались для моделирования функционирования электрогенераторов \cite{Tateno2012, Namajunas1995, Glyzin2018, Glyzin2018a}, а также для симуляции хаотического сигнала \cite{Grassberger1983, Amil2015, Amil2015a, Shahverdiev2006}.

Помимо моделей, описываемых одним уравнением, представляют интерес модели, получаемые объединением элементов, функционирующих некоторым известным образом, в сеть \cite{Glyzin2022}. Можно выделить две естественные структуры для связи элементов сети: кольцо, где элемент связан с соседними элементами, и полносвязную сеть (рис. \ref{fig:full_mesh:intro}), где каждый элемент сети связан со всеми остальными.

\begin{figure}[ht]
	\centering
	\includegraphics[width=0.5\textwidth]{mg_generator_full.eps}
	\caption{Полносвязная сеть осцилляторов. Каждый осциллятор является передающим и принимающим для всех остальных осцилляторов в~сети.}
	\label{fig:full_mesh:intro}
\end{figure}

Примером такой системы являются искусственные генные сети. Интерес к искусственным генным осцилляторам вызван тем обстоятельством, что они являются упрощёнными моделями таких ключевых биологических процессов, как клеточный цикл и циркадные ритмы. Простейший генетический осциллятор, предложенный в \cite{Elowitz2000} и названный репрессилятором, состоит как минимум из трёх элементов, соединённых в кольцо \cite{Glyzin2017, GlyzinBook2018}. Функционирование такой сети описывается системой
\begin{equation}
	\label{eq:intro:repressilator}
	\dot{u}_j = -u_j + \dfrac{\alpha}{1 + u^{\gamma}_{j - 1}}, \quad j = 1, 2, 3,
\end{equation}
где $u_0 = u_3$, $\alpha, \gamma > 0$. Исследование генных сетей проводится в работах \cite{Likhoshvaj2003, Volokitin2004, Golubyatnikov2006, Buse2009, Buse2010}.

Аналогичным образом можно соединить в сеть элементы, функционирование которых описывается уравнением \eqref{eq:mg_equation_1:intro}. Такие системы были исследованы в работах \cite{Preobrazhenskaia2021, Tateno2012, Sano2007, Wan2009}. Так, в \cite{Sano2007} численно и экспериментально изучалась система из четырёх генераторов Мэки--Гласса, два из которых были вещательными, а два --- принимающими. В работе \cite{Wan2009} исследовалась потеря устойчивости состояния равновесия в этой системе, а также условия, при которых в результате бифуркации рождается устойчивый предельный цикл.

В диссертационной работе исследуются периодические режимы, возникающие в полносвязной сети осцилляторов, функционирование которых описывается уравнениями Мэки--Гласса. Доказывается существование периодических режимов специального вида: дискретных бегущих волн и двухкластерной синхронизации.

\bigskip

{\aim} Целью данной работы является исследование периодических режимов в полносвязной сети релейных осцилляторов Мэки--Гласса.

Для~достижения поставленной цели необходимо было решить следующие задачи.
\begin{enumerate}[beginpenalty=10000] % https://tex.stackexchange.com/a/476052/104425
	\item Описать достаточные условия, при которых уравнение Мэки--Гласса \eqref{eq:mg_equation_1:intro} имеет периодическое решение.
	\item Исследовать полносвязную систему релейных осцилляторов Мэки--Гласса, описать условия и ограничения на параметры системы, при которых она имеет решение:
	\begin{enumerate}
		\item[а)]в виде дискретной бегущей волны,
		\item[б)]в виде, соответствующем режиму двухкластерной синхронизации.
	\end{enumerate}
\end{enumerate}

\bigskip

{\novelty} Все полученные в работе результаты являются новыми. 
\begin{enumerate}[beginpenalty=10000] % https://tex.stackexchange.com/a/476052/104425
	\item Впервые получены асимптотические формулы решения уравнения Мэки--Гласса \eqref{eq:intro:MG_norm1} по параметру $\gamma \gg 1$ и доказано существование периодических решений при ограничении на параметры $\alpha > \exp\left(\beta(1 + e^{-\beta})\right)$.
	\item Впервые доказано существование периодических режимов в виде дискретной бегущей волны в полносвязной сети релейных осцилляторов Мэки--Гласса, а также сформулированы и доказаны условия их существования в виде ограничения на параметры соответствующей системы дифференциальных уравнений с запаздыванием.
	\item Впервые доказано существование периодических режимов двухкластерной синхронизации в полносвязной сети релейных осцилляторов Мэки--Гласса, а также сформулированы и доказаны условия их существования в виде ограничения на параметры соответствующей системы дифференциальных уравнений с запаздыванием.
	%TODO: сказать про скользящие траектории.
\end{enumerate}

\bigskip

{\influence} Научная работа носит теоретический характер, в ней были применены методы нелинейного анализа динамических систем в бесконечномерном фазовом пространстве. Теоретическая ценность работы определяется тем, что асимптотический метод большого параметра, а также подходы к построению дискретных бегущих волн и режимов кластерной синхронизации были адаптированы для применения к уравнению Мэки--Гласса и составленной на его основе полносвязной релейной системы из $N$ дифференциальных уравнений с запаздыванием (соответственно).

Полученные в диссертации результаты могут стать основой для дальнейших исследований в области нелинейной динамики и нелинейного функционального анализа, быть использованы специалистами для решения широкого спектра научных и прикладных задач. В частности, техника, представленная в диссертации, может быть перенесена на класс уравнений, обобщающих уравнение Мэки--Гласса, где вместо рациональной нелинейности в правой части уравнения рассматривается класс функций, имеющих в качестве предельного объекта релейную функцию. 

Также полученные результаты могут применяться при разработке и чтении курсов и спецкурсов по теории дифференциальных уравнений с запаздыванием и методам их исследования.

Работа выполнена в рамках программы развития Регионального научно-образовательного математического центра Ярославского государственного университета им.~П.~Г.~Демидова при финансовой поддержке Министерства науки и высшего образования Российской Федерации (Соглашение о предоставлении субсидии из федерального бюджета № 075-02-2025-1636).

\bigskip

{\defpositions} На защиту выносятся следующие результаты диссертации:
\begin{enumerate}[beginpenalty=10000] % https://tex.stackexchange.com/a/476052/104425
	\item Получены асимптотические формулы периодического решения уравнения Мэки--Гласса \eqref{eq:intro:MG_norm1} по параметру $\gamma \gg 1$ при ограничении на параметры $\alpha > \exp\left(\beta(1 + e^{-\beta})\right)$ (Теорема~1.3.8). На основе полученных формул доказана теорема о существовании периодического решения уравнения Мэки--Гласса  (Теорема~1.3.1).
	\item Доказано существование периодических режимов в виде дискретной бегущей волны в полносвязной сети релейных осцилляторов Мэки--Гласса, сформулированы и доказаны достаточные условия их существования в виде ограничения на параметры соответствующей системы дифференциальных уравнений с запаздыванием (Теорема 2.3.7).
	\item Доказана теорема о существовании (в смысле обобщённого решения системы дифференциальных уравнений с разрывной правой частью) периодических режимов двухкластерной синхронизации в полносвязной  релейных осцилляторов Мэки--Гласса, сформулированы и доказаны достаточные условия их существования в виде ограничения на параметры соответствующей системы дифференциальных уравнений с запаздыванием (Теорема 3.4.2).
\end{enumerate}

\bigskip

\textbf{Основные публикации по теме исследования и степень достоверности результатов.} The following publications are presented for defense:
%
\begin{enumerate}[beginpenalty=10000] % https://tex.stackexchange.com/a/476052/104425
	\item \emph{В.~В.~Алексеев, М.~М.~Преображенская}. Анализ асимптотической сходимости периодического решения уравнения Мэки–-Гласса к решению предельного релейного уравнения. \emph{Теоретическая и математическая физика}. --- 2024. --- Т. 220, № 2. --- С. 213--236. \cite{wosbib1}
	\item \emph{V.~Alekseev, M.~Preobrazhenskaia, V.~Vorontsova}. Existence of Discrete Traveling Waves in Fully Coupled Network of Mackey--Glass Relay Generators. \emph{Differential Equations}. --- 2024. --- Vol. 60, No 9. --- P.~1217--1231 \cite{wosbib2}
	\item \emph{V.~Alekseev}. Two-cluster synchronization on a fully coupled network of Mackey--Glass generators. \emph{Partial Differential Equations in Applied Mathematics}. --- 2024. --- Vol. 12. --- P. 100930. \cite{scbib1}
\end{enumerate}

Основные результаты по теме диссертации изложены в 10 печатных изданиях, 3 из которых \cite{wosbib1,wosbib2,scbib1} изданы в журналах, рекомендованных ВАК, 3 "--- в~периодических научных журналах, индексируемых Web of~Science или Scopus \cite{wosbib1,wosbib2,scbib1}, 7 "--- в~тезисах докладов \cite{Sergeev2024,confbib1,confbib2,confbib3,confbib4,confbib5,confbib6}.

Из работ, написанных в соавторстве, в диссертацию включены только результаты, полученные автором лично. Научным руководителем М.~М.~Преображенской осуществлялась постановка задач.

Достоверность полученных результатов обеспечивается строгими математическими доказательствами, приведёнными в работе.

\nocite{scbib1, wosbib1, wosbib2}

\bigskip

{\probation}
Основные результаты работы докладывались~на следующих конференциях и семинарах.
\begin{enumerate}
	\item Семинар кафедры <<Функциональный анализ и его приложения>> Владимирского государственного университета им.~А.~Г.~и~Н.~Г.~Столетовых, 13 февраля 2025 года.
	\item Семинар по качественной теории дифференциальных уравнений в Московском государственном университете им.~М.~В.~Ломоносова, 29~ноября 2024~года. \cite{Sergeev2024},\\\texttt{https://www.elibrary.ru/item.asp?id=75144298}
	\item Научный семинар лаборатории динамических систем и приложений НИУ ВШЭ в Нижнем Новгороде, 25 сентября 2024 г.,\\\texttt{https://nnov.hse.ru/bipm/dsa/semtmd}.
	\item Семинар по нелинейной динамике Ярославского государственного университета им.~П.~Г.~Демидова, 19 сентября 2024 г.,\\\texttt{https://cis.uniyar.ac.ru/index.php/event/460}.
	\item Конференция <<Integrable Systems and Nonlinear Dynamics>> (ISND – 2024), Ярославль, 2024 \cite{confbib5}.
	\item Конференция <<Topological Methods in Dynamics and Related Topics VII>>, Нижний Новгород, 2024 \cite{confbib6}.
	\item Международная конференция по дифференциальным уравнениям и динамическим системам DIFF-2024, Суздаль, 2024 \cite{confbib3}.
	\item Конференция <<Нелинейные дни в Саратове для молодых>>, Саратов, 2023 \cite{confbib2}.
	\item Конференция <<Satellite International Conference on Nonlinear Dynamics {\&} Integrability>>, Ярославль, 2022 \cite{confbib4}.
	\item Международная конференция по дифференциальным уравнениям и динамическим системам DIFF--2022, Суздаль, 2022 \cite{confbib1}.
\end{enumerate}

%{\contribution} Автор принимал активное участие \ldots

% \vspace{-11em}

\begin{refsection}[bl-author, bl-registered]
	% Это refsection=2.
	% Процитированные здесь работы:
	%  * попадают в авторскую библиографию, при usefootcite==0 и стиле `\insertbiblioauthorimportant`.
	%  * ни на что не влияют в противном случае
	\nocite{vakbib2}%vak
	\nocite{patbib1}%patent
	\nocite{progbib1}%program
	\nocite{bib1}%other
	\nocite{confbib1}%conf
\end{refsection}%
%
% Всё, что вне этих двух refsection, это refsection=0,
%  * для диссертации - это нормальные ссылки, попадающие в обычную библиографию
%  * для автореферата:
%     * при usefootcite==0, ссылка корректно сработает только для источника из `external.bib`. Для своих работ --- напечатает "[0]" (и даже Warning не вылезет).
%     * при usefootcite==1, ссылка сработает нормально. В авторской библиографии будут только процитированные в refsection=0 работы.
