
% common/newnames.tex & common/renames.tex
{\actuality} Биологические модели занимают важное место в теории динамических систем, охватывая широкий спектр явлений, происходящих в биологических процессах. Эти модели включают как те, которые описывают концентрацию различных веществ в биологических системах, таких как химические реакции в клетках или процессы обмена веществ в организмах, так и популяционные уравнения, моделирующие изменение численности и структуру популяций живых существ. Обычно модели популяционной динамики описываются дифференциальными или дифференциально-разностными уравнениями. 
Эти модели с течением времени претерпели значительные изменения и усложнения.
%TODO Надо связать популяционные модели и моделирование концентрации в общий биологический контекст (экологический)

Одной из первых популяционных моделей была модель Томаса Мальтуса, который предположил, что рост популяции, не сдерживаемой ограничениями на ресурсы, будет экспоненциальным. Массовые вымирания, такие как эпидемии, голод и войны, в модели Мальтуса объясняются тем, что ресурсы, необходимые для выживания, не могут воспроизводиться с такой же скоростью, как и популяция, что приводит к перенаселению \cite{Malthus1798}.

Пьер Ферхюльст в работе \cite{Verhulst1838} предложил добавить в модель Мальтуса квадратичное слагаемое, ограничивающее скорость роста популяции для больших значений её размера. Таким образом, он получил модель, называемую логистическим уравнением
\begin{equation}
	\dot{x}=rx\left(1-\frac{x}{K}\right),
\end{equation}
где функция $x(t)$ показывает текущий размер популяции, параметр $r$ характеризует скорость роста популяции, а параметр $K$ --- поддерживающую ёмкость среды (то есть, максимально возможную численность популяции). 

Позже рассматривались различные усложнения модели. В частности, для описания динамики численности двух конкурирующих видов была предложена модель Лотки -- Вольтерры, также с квадратичной зависимостью от неизвестной функции в правой части \cite{Lotka1925, Volterra1926}.
%
\begin{equation}
	\begin{cases}
		\dot{x}=(\alpha-\beta y)x,\\
		\dot{y}=(-\gamma+\delta x)y,
	\end{cases}
\end{equation}
%
где функция $x(t)$ --- плотность популяции добычи (жертвы), $y(t)$ --- плотность популяции хищника, параметр $\alpha$ описывает естественную скорость роста популяции жертвы, $\beta$ --- влияние численности популяции хищника на уменьшение популяции жертвы, $\gamma$ --- коэффициент смертности хищника, $\delta$ --- влияние численности популяции жертвы на увеличение популяции хищника. Все параметры --- положительные вещественные числа.

% Усложнение модели --- экспоненциальная зависимость в правой части %TODO Пока не нашла подробности, что это за модель

Одна из идей усложнения правой части уравнения --- добавление запаздывания. Так в 1948 году Джордж Хатчинсон предложил модификацию логистического уравнения, обладающую запаздыванием по времени
%
\begin{equation}
	\dot{x}=rx\left(1 - \frac{x(t-\tau)}{K}\right),
\end{equation}
%
где параметр $r$ характеризует скорость роста популяции, параметр $K$ соответствует средней численности популяции, запаздывание $\tau$ --- возраст половозрелости \cite{Hutchinson1948}. %https://encyclopediaofmath.org/wiki/Hutchinson_equation

Помимо уравнения Хатчинсона, рассматривались различные его обобщения с заменой квадратичной зависимости в правой части на более сложную. Например, в работе \cite{Glyzin2007} предлагается модель, более тонко учитывающая возрастную структуру
\begin{equation}
	\dot{x}=rx\left(1 - \frac{1}{K}\sum\limits_{i = 1}^{m} a_i x(t-\tau_i)\right).
\end{equation}
%
В работе \cite{Kaschenko2012} исследуется обобщение модели с непрерывным запаздыванием
\begin{equation}
	\dot{x} = \lambda x \left(1 - \int\limits_{h_1}^{h_2}dr(\tau)x(t - \tau)\right),
\end{equation}
где $r(\tau)$ --- монотонная неотрицательная функция, $\lambda, h_1, h_2$ --- положительные параметры.
%
В работе \cite{Kolesov2010} предлагается модель вида
\begin{equation}
	\dot{x} = \lambda x f(x(t - 1)).
\end{equation}

%
%Введение запаздывания позволяет построить более точную модель, учитывающую предыдущие состояния моделируемого объекта. Запаздывание дает более богатую динамику... новые эффекты... %TODO Написать чем лучше модель с запаздыванием. Примеры, ссылки
%
%Следующий вариант развития модели --- тенденция к насыщению. Как следствие, обретает смысл устремление некоторого параметра к бесконечности, рассмотрение предельного для уравнения объекта.  Сигмоидальная функция в правой части....
%% TODO: что-нибудь надушить

В диссертационной работе исследуется уравнение Мэки--Гласса. Уравнениями Мэки--Гласса называют две модели с запаздыванием \cite{Mackey1977, Glass1988}:
\begin{equation}
	\label{eq:mg_equation_1:intro}
	\dot{v}=-b v+\frac{a \theta^{\gamma} v(t-\tau)}{\theta^{\gamma}+(v(t-\tau))^{\gamma}},
\end{equation}
\begin{equation}
	\label{eq:mg_equation_2:intro}
	\dot{v}=-b v+\frac{a \theta^{\gamma}}{\theta^{\gamma}+(v(t-\tau))^{\gamma}},
\end{equation}
где параметры $a, b, \gamma, \theta$ --- положительные вещественные числа.

Эти модели были предложены в работе \cite{Mackey1977} для описания регуляторных функций в процессах кроветворения. Уравнения \eqref{eq:mg_equation_1:intro} и \eqref{eq:mg_equation_2:intro} различаются формой нелинейности в слагаемом с запаздыванием: в уравнении \eqref{eq:mg_equation_2:intro} она монотонно убывает, в то время как в уравнении \eqref{eq:mg_equation_1:intro} имеет форму <<горба>>. В данной работе под уравнением Мэки--Гласса будет пониматься уравнение \eqref{eq:mg_equation_1:intro}.

Биологический смысл данной модели следующий: функция $v(t)$ --- плотность циркулирующих в крови человека нейтрофилов (вид лейкоцитов) в клетках на кг массы тела, $b$ --- скорость случайного распада нейтрофилов, положительное слагаемое означает текущий приток клеток в кровь, возникающий в ответ на запрос, создавшийся в некоторый момент $\tau$ в прошлом.

В широком диапазоне изменений уровня циркулирующих нейтрофилов скорость образования нейтрофилов и падает с увеличением их плотности. Однако благодаря действию различных факторов можно ожидать, что при очень низких уровнях нейтрофилов скорость их образования будет падать, приближаясь к нулю \cite[с. 85]{Mackey1977}.

Форма нелинейности выбирается из приведённых соображений.

Уравнение Мэки--Гласса исследовалось во множестве работ \cite{Junges2012, Berezansky2006, Su2011, Liz2002, Wu2007, Kubyshkin2016, Krisztin2020, Bartha2021}. В оригинальной работе \cite{Mackey1977} приведены численные решения, как в форме периодических, так и непериодических колебаний. В работах \cite{Krisztin2020, Bartha2021} исследуются периодические решения уравнения Мэки--Гласса. В частности, в \cite{Bartha2021} доказано (при некоторых ограничениях на параметры и множество начальных функций) существование и единственность орбитально устойчивого предельного цикла. 

Уравнение допускает различные обобщения. Так, в работе \cite{Berezansky2006} исследуется аналог уравнения Мэки--Гласса с переменным запаздыванием. В работе \cite{Liz2002} изучаются асимптотические свойства решений уравнений <<типа Мэки--Гласса>> с нелинейностью похожей формы. В \cite{Huang2024} изучается стохастическая версия уравнения с несколькими запаздываниями.

Уравнение Мэки--Гласса и его различные модификации широко использовались для моделирования функционирования электрогенераторов \cite{Tateno2012, Namajunas1995, Glyzin2018, Glyzin2018a}, а также для симуляции хаотического сигнала \cite{Grassberger1983, Amil2015, Amil2015a, Shahverdiev2006}.

Помимо биологических моделей, описываемых одним уравнением, представляют интерес модели, получаемые объединением элементов, функционирующих некоторым известным образом, в цепь. Следует отметить, что цепочки идентичных нелинейных осцилляторов используются в качестве математических моделей в различных областях естествознания: биофизике, экологии, оптике, химической кинетике, нейродинамике, генной инженерии и др. \cite{Goodwin1963}. Можно выделить две естественные структуры для связи элементов цепи: кольцо (рис. \ref{fig:ring:intro}), где элемент связан с соседними элементами, и полносвязную цепь (рис. \ref{fig:full_mesh:intro}), где каждый генератор связан с каждым.

\begin{figure}[ht]
	\begin{minipage}[b]{0.45\linewidth}
		\centering
		\includegraphics[width=\textwidth]{mg_generator_ring.eps}
		\caption{Кольцо осцилляторов. В однонаправленной кольцевой цепи каждый осциллятор является принимающим для предыдущего, и передающим для следующего в кольце генератора.}
		\label{fig:ring:intro}
	\end{minipage}
	\hspace{0.5cm}
	\begin{minipage}[b]{0.45\linewidth}
		\centering
		\includegraphics[width=\textwidth]{mg_generator_full.eps}
		\caption{Полносвязная цепь осцилляторов. Каждый осциллятор является передающим и принимающим для всех остальных осцилляторов в цепи.}
		\label{fig:full_mesh:intro}
	\end{minipage}
\end{figure}

Примером такой модели являются искусственные генные сети. Интерес к искусственным генным осцилляторам вызван тем обстоятельством, что они являются упрощёнными моделями таких ключевых биологических процессов, как клеточный цикл и циркадные ритмы. Простейший генетический осциллятор, предложенный в \cite{Elowitz2000} и названный репрессилятором, состоит как минимум из трёх элементов, соединённых в кольцо \cite{GlyzinBook2018}. Функционирование такой сети описывается системой
\begin{equation}
	\label{eq:intro:repressilator}
	\dot{u}_j = -u_j + \dfrac{\alpha}{1 + u_{j - 1}}, \quad j = 1, 2, 3,
\end{equation}
где $u_0 = u_3$, $\alpha, \gamma > 0$.

Вопрос существования и свойств предельных циклов в простейшей генной сети изучался в работах \cite{Volokitin2004, Buse2009, Buse2010}.

Аналогичным образом можно соединить в цепь элементы, функционирование которых описывается уравнением \eqref{eq:mg_equation_1:intro}. Такие системы были исследованы во множестве работ \cite{Preobrazhenskaia2021, Tateno2012, Sano2007, Wan2009}. Так, в \cite{Sano2007} численно и экспериментально изучалась система из четырёх генераторов Мэки--Гласса, два из которых были вещательными, а два --- принимающими. В работе \cite{Wan2009} исследовалась потеря устойчивости состояния равновесия в этой системе.

В диссертационной работе исследуются периодические режимы, возникающие в цепи осцилляторов, функционирование которых описывается уравнением Мэки--Гласса. 

%При исследовании задач диссертационной работы используются метод большого параметра, метод эквивалентного управления, методы функционального анализа, метод шагов. Кроме того, в дополнение проведённых исследований выполнен численный эксперимент.

{\methods} Для сложных моделей, в случае, когда не удается отыскать решение непосредственным интегрированием, уместно использование новых специальных методов поиска решения. В частности, это относится к дифференциальным уравнениям с запаздывающим аргументом. Одна из идей упрощения исследования --- это переход к предельному объекту. Другая идея состоит в том, чтобы найти решение в специальном виде. Остановимся на описании этих идей подробнее.

\textit{Переход к предельному объекту.} В ряде случаев при исследовании сложного уравнения удается определить содержательный предельный объект при устремлении одного из параметров к бесконечности (см., например, \cite{Kolesov1997}). В частности, это получается сделать, если нелинейность в правой части уравнения близка к сигмоидальной функции, например, имеет вид $F_\gamma(u)=\frac{1}{1 + u^\gamma}$, $u > 0$ (см., например, \cite{Preobrazhenskaya2020, Glyzin2017, Krisztin2020, Bartha2021}). Такая функция в пределе при $\gamma\to+\infty$ устремляется к кусочно-постоянной функции, меняющей значение в 1. Тогда исходное уравнение можно подменить релейным, которое, как правило, существенно проще для анализа. После построения решения предельного уравнения можно попытаться доказать существование асимптотически близкого решения исходной задачи.

\textit{Поиск решений специального вида.} В настоящей работе используются два подхода к построению специальных решений системы дифференциально-разностных уравнений, разработанные в работах С. Д. Глызина и др. \cite{GlyKol2013, GlyKol2013a, Glyzin2014}: это построение дискретных бегущих волн (вторая глава диссертации) и поиск режимов кластерной синхронизации (третья глава диссертации). Опишем метод построения решений того и другого сорта в общем виде.

\textit{Дискретные бегущие волны.} Это специальные периодические решения в симметричной системе осцилляторов, которые могут быть связаны в кольцо, либо в полносвязную систему. Дискретные бегущие волны в кольцевых системах описаны в работах \cite{GlyKol2013a, Glyzin2016, Kolesov2016}, а в полносвязных --- в работах \cite{Glyzin2022, Glyzin2022a, Preobrazhenskii2024}.

Пусть $n$ осцилляторов $x_1, \ldots, x_n$ описываются симметричной кольцевой системой дифференциальных уравнений
\begin{equation}
\label{eq:intro:sys_Phi_circ}
	\dot{x}_j=\Phi(x_j, x_{j-1}), \quad j=1, \ldots, n, \quad x_{0} = x_{n},
\end{equation}
где $\Phi:\mathbb{R}^2\to\mathbb{R}$. 

Дискретной бегущей волной называют периодическое решение, все компоненты которого представлены одной и той же периодической функцией $x(t)$ со сдвигом по времени, кратным некоторому параметру $\Delta$. Таким образом, решение представляется в следующем виде:
%
\begin{equation}
\label{eq:intro:wave}
	x_j(t) = x(t + j\Delta).
\end{equation}
%
При подстановке \eqref{eq:intro:wave} в систему \eqref{eq:intro:sys_Phi_circ} получаем, что $x(t)$ удовлетворяет уравнению
%
\begin{equation}
	\label{eq:intro:Phi_circ}
	\dot{x}=\Phi(x, x(t-\Delta)).
\end{equation}

Условие $x_0 \equiv x_n$ требует выполнения тождества
$x(t + n\Delta) \equiv x(t)$. Это значит, что величина $n\Delta$ кратна периоду $T = T(\Delta)$ функции $x(t)$. Следовательно, параметр $\Delta$ обязан удовлетворять уравнению периодов
\begin{equation}
	\label{eq_period_Delta}
	p T(\Delta) = n\Delta
\end{equation}
при некотором целом $p \neq 0$.

Таким образом, задача построения дискретных бегущих волн системы \eqref{eq:intro:sys_Phi_circ} сводится к поиску периодического решения вспомогательного дифференциально-разностного уравнения \eqref{eq:intro:Phi_circ} и подбору подходящего параметра $\Delta$, удовлетворяющего уравнению периодов (\ref{eq_period_Delta}). 

В случае полносвязной системы осцилляторов соответствующая система дифференциальных уравнений имеет следующую форму:
%
\begin{equation}\label{eq:intro:sys_Psi_full}
	\dot{x}_j= \Psi(x_j, x_{j-1}, \ldots, x_{j-n+1}), \quad j=1, \ldots, n, \quad x_j = x_{j+n}.
\end{equation}
%
Здесь $\Psi:\mathbb{R}^{n}\to\mathbb{R}$. 
Соответствующее вспомогательное уравнение с запаздываниями для поиска функции $x(t)$ принимает вид
%
\begin{equation}
	\label{eq:intro:Psi_full}
	\dot{x}= \Psi(x, x(t-\Delta), \ldots, x(t-n\Delta)).
\end{equation}

Для полносвязной системы \eqref{eq:intro:sys_Psi_full} последовательный сдвиг функции $x(t)$ по времени можно организовать для любой перестановки $(j_1,\ldots,j_n)$ номеров $(1,\ldots,n)$ осцилляторов $x_j$.
Следовательно, вместе с дискретной бегущей волной \eqref{eq:intro:wave}существует $n!$ дискретных бегущих волн:
\begin{equation}
	\label{wave_full}
	x_k(t) = x(t + j_k\Delta).
\end{equation}

Отметим, что системы \eqref{eq:intro:sys_Phi_circ} и \eqref{eq:intro:sys_Psi_full} могут изначально включать запаздывание по времени.

\textit{Режимы кластерной синхронизации.} 
Рассмотрим полносвязную систему из $N$ осцилляторов следующего вида:
\[\dot{x}_j
=\Xi(x_j;x_1,\ldots,\hat{x}_j,\ldots,x_{N}),\quad j=1,\ldots,N,\]
где $\Xi:\mathbb{R}^{N}\to\mathbb{R}$ --- функция, симметричная по всем аргументам кроме, возможно, первого, символ $\hat{x}_j$ обозначает исключение соответствующего элемента из аргументов функции.

Под кластерной синхронизацией будем понимать такое разбиение множества осцилляторов $x_1,\ldots,x_{N}$ на подмножества, что в каждом подмножестве осцилляторы функционируют одинаково.

В настоящей работе строятся режимы двухкластерной синхронизации. Пусть $N = n + m$. Предположим, что $n$ осцилляторов описываются одной функцией, а $m$ осцилляторов --- другой:
\begin{equation}
	\label{eq:intro:cluster}
	x_1(t)=\ldots=x_n(t)=x(t),\quad x_{n+1}(t)=\ldots=x_{n+m}(t)=y(t).
\end{equation}
При этом функции $x(t)$ и $y(t)$ удовлетворяют двумерной системе
\[
\begin{cases}
	\dot{x}=\xi(x,y),\\
	\dot{y}=\eta(x,y),
\end{cases}
\]
где 
\[
\xi(x,y)=\Xi(x;\underbrace{x,\ldots,x}_{n-1},\underbrace{y\ldots,y}_{m}),\quad
\eta(x,y)=\Xi(y;\underbrace{x,\ldots,x}_{n},\underbrace{y\ldots,y}_{m-1}).
\]
Отметим, что в случае существования режима \eqref{eq:intro:cluster}, сосуществует сразу $C_{n+m}^n$ подобных режимов.

Режимы двухкластерной синхронизации строятся в работах \cite{Glyzin2016a, Glyzin2022}.

\bigskip

%Уравнение \eqref{eq:mg_equation_1:intro} нормировкой времени и заменой переменных сводится к следующему виду
%\begin{equation}
%	\label{eq:mg_normed_equation:intro}
%	\dot{u}=-\beta u+\frac{\alpha u(t-1)}{1 + (u(t-1))^\gamma},
%\end{equation}
%где $\alpha, \beta, \gamma > 0$, $u(t) > 0$.
%
%Для исследования поведения решений уравнения \eqref{eq:mg_normed_equation:intro} при $\gamma \gg 1$ определим предельное уравнение при $\gamma \to +\infty$:
%\begin{equation}
%	\label{eq:mg_relay_equation:intro}
%	\dot{u}=-\beta u + \alpha u(t-1) F(u(t-1)),
%\end{equation}
%где
%\begin{equation*}
%	F(u) = \lim\limits_{\gamma \to +\infty}\dfrac{1}{1 + u^\gamma} = \begin{cases}
%		1, & u < 1,\\
%		1/2, & u = 1,\\
%		0, & u > 1.
%	\end{cases}
%\end{equation*}
%Уравнение \eqref{eq:mg_relay_equation:intro} будем называть \emph{релейным} по причине разрыва типа <<скачок>> в правой части при $u = 1$.

%В диссертационной работе исследуются периодические режимы систем осцилляторов, функционирование которых описывается уравнением Мэки--Гласса. Для исследования релаксационных колебаний одиночного осциллятора используется метод большого параметра. Периодические режимы исследуются для систем осцилляторов релейного типа, т.е. систем, элементы которых описываются релейным уравнением Мэки--Гласса \eqref{eq:mg_relay_equation:intro}.
%
%Уравнение Мэки--Гласса исследовалось в ряде работ \cite{Junges2012, Berezansky2006, Su2011, Liz2002, Wu2007, Kubyshkin2016}. В оригинальной работе \cite{Mackey1977} приведены численные решения, как в форме периодических, так и непериодических колебаний. В работах \cite{Berezansky2006, Liz2002} изучены условия, при которых решение уравнения остаётся положительным (англ. \emph{persistence}) или стремится к нулю при $t \to +\infty$ (англ. \emph{extinction}). В работах \cite{Berezansky2006, Kubyshkin2016} также анализируется устойчивость состояния равновесия $v \equiv \theta \left(\frac{a}{b} - 1\right)^{1/\gamma}$. В работе \cite{Kubyshkin2016} рассматриваются решения, бифурцирующие из этого состояния равновесия.
%
%Уравнение Мэки--Гласса и его различные модификации широко использовались для моделирования функционирования электрогенераторов \cite{Tateno2012, Namajunas1995, Glyzin2018, Glyzin2018a}, а также для симуляции хаотического сигнала \cite{Grassberger1983, Amil2015, Amil2015a, Shahverdiev2006}. В работах \cite{Bartha2021, Krisztin2020} исследуются периодические решения уравнения Мэки--Гласса. В частности, в \cite{Bartha2021} при определённых ограничениях на параметры доказано существование и единственность предельного цикла, который является экспоненциально орбитально устойчивым.
%
%Особое внимание уделяется исследованию систем, представляющих собой цепи генераторов Мэки--Гласса \cite{Preobrazhenskaia2021, Tateno2012, Sano2007, Wan2009}. В статье \cite{Sano2007} численно и экспериментально изучалась система из четырёх генераторов Мэки--Гласса, два из которых были вещательными, а два --- принимающими, образуя систему с петлёй обратной связи. В работе приводится схема электрической цепи и описание эксперимента, в ходе которого установлена двойная синхронизация хаотического сигнала. В работе \cite{Wan2009} исследовалась потеря устойчивости состояния равновесия в этой системе.
%
%Похожая система генераторов Мэки--Гласса с двумя уравнениями и несколько иной структурой связи рассматривалась в \cite{Tateno2012}. В данной цепи генераторы делятся на два типа: один вещатель и одно принимающее устройство. Модель содержит две петли обратной связи с запаздыванием по времени. Для этой системы проводились как численные, так и экспериментальные исследования генерации хаоса. Показано, что соотношение запаздываний играет ключевую роль в усилении или подавлении хаотической динамики. Также продемонстрировано, что синхронизация хаоса может быть достигнута при высокой силе связи и условии согласования параметров между двумя электронными схемами.
%
%Рассмотрим систему произвольного (фиксированного) числа генераторов. Можно выделить две естественные структуры для связи генераторов: кольцо (рис. \ref{fig:ring:intro}) и полносвязную цепь, где каждый генератор связан с каждым (рис. \ref{fig:full_mesh:intro}). Следует отметить, что цепочки идентичных нелинейных осцилляторов используются в качестве математических моделей в различных областях естествознания: биофизике, экологии, оптике, химической кинетике, нейродинамике, генной инженерии и др. \cite{Goodwin1963}.
%
%В данной работе рассматриваются два типа периодических режимов, возникающих в полносвязной цепи генераторов Мэки--Гласса: режим дискретных бегущих волн и режим двухкластерной синхронизации.
%
%\begin{figure}[ht]
%	\begin{minipage}[b]{0.45\linewidth}
%		\centering
%		\includegraphics[width=\textwidth]{mg_generator_ring.eps}
%		\caption{Кольцо генераторов. Каждый генератор является принимающим для предыдущего, и передающим для следующего в кольце генератора.}
%		\label{fig:ring:intro}
%	\end{minipage}
%	\hspace{0.5cm}
%	\begin{minipage}[b]{0.45\linewidth}
%		\centering
%		\includegraphics[width=\textwidth]{mg_generator_full.eps}
%		\caption{Полносвязная цепь генераторов. Каждый генератор является передающим и принимающим для всех остальных генераторов в цепи.}
%		\label{fig:full_mesh:intro}
%	\end{minipage}
%\end{figure}
%
%Дискретная бегущая волна --- такое периодическое решение, в котором все компоненты представлены одной и той же функцией с кратными сдвигами, строгое описание таких решений будет приведено во второй главе диссертации. Методика поиска бегущих волн и доказательства их устойчивости в кольцевых цепях описана в работе \cite{Glyzin2012}, а для полносвязной цепи --- в \cite{Glyzin2022a}.
%
%Режим двухкластерной синхронизации --- такое решение, в котором все компоненты совпадают с одной из двух различных периодических функций. Общий подход к исследованию таких режимов описан в статье \cite{Glyzin2022}.
%
%В первой главе диссертации исследуются периодические решения уравнения \eqref{eq:mg_equation_1:intro}. Вначале рассматривается предельное при $\gamma \to +\infty$ уравнение, для которого явным образом (при некоторых ограничениях на параметры) находится решение. Затем для уравнения \eqref{eq:mg_equation_1:intro} доказываются асимптотические соотношения при $\gamma \to +\infty$ по малому параметру $\gamma^{-1}$, доказывающие сходимость решений уравнения \eqref{eq:mg_equation_1:intro} к решению предельного уравнения.
%
%Во второй и третьей главах диссертации рассматривается полносвязная система генераторов Мэки--Гласса. Как и в работах \cite{Preobrazhenskaya2020, Preobrazhenskaya2021, Preobrazhenskaia2021, Krisztin2020}, соответствующая система дифференциальных уравнений с запаздыванием заменяется предельным объектом, для которого во второй главе доказывается существование режимов дискретной бегущей волны, а в третьей главе --- существование режимов двухкластерной синхронизации.



%Обзор, введение в тему, обозначение места данной работы в
%мировых исследованиях и~т.\:п., можно использовать ссылки на~другие
%работы~(если их~нет, то~в~автореферате
%автоматически пропадёт раздел <<Список литературы>>). Внимание! Ссылки
%на~другие работы в~разделе общей характеристики работы можно
%использовать только при использовании \verb!biblatex! (из-за технических
%ограничений \verb!bibtex8!. Это связано с тем, что одна
%и~та~же~характеристика используются и~в~тексте диссертации, и в
%автореферате. В~последнем, согласно ГОСТ, должен присутствовать список
%работ автора по~теме диссертации, а~\verb!bibtex8! не~умеет выводить в~одном
%файле два списка литературы).
%При использовании \verb!biblatex! возможно использование исключительно
%в~автореферате подстрочных ссылок
%для других работ командой \verb!\autocite!, а~также цитирование
%собственных работ командой \verb!\cite!. Для этого в~файле
%\verb!common/setup.tex! необходимо присвоить положительное значение
%счётчику \verb!\setcounter{usefootcite}{1}!.

%Для генерации содержимого титульного листа автореферата, диссертации
%и~презентации используются данные из файла \verb!common/data.tex!. Если,
%например, вы меняете название диссертации, то оно автоматически
%появится в~итоговых файлах после очередного запуска \LaTeX. Согласно
%ГОСТ 7.0.11-2011 <<5.1.1 Титульный лист является первой страницей
%диссертации, служит источником информации, необходимой для обработки и
%поиска документа>>. Наличие логотипа организации на~титульном листе
%упрощает обработку и~поиск, для этого разметите логотип вашей
%организации в папке images в~формате PDF (лучше найти его в векторном
%варианте, чтобы он хорошо смотрелся при печати) под именем
%\verb!logo.pdf!. Настроить размер изображения с логотипом можно
%в~соответствующих местах файлов \verb!title.tex!  отдельно для
%диссертации и автореферата. Если вам логотип не~нужен, то просто
%удалите файл с~логотипом.

%\ifsynopsis
%Этот абзац появляется только в~автореферате.
%Для формирования блоков, которые будут обрабатываться только в~автореферате,
%заведена проверка условия \verb!\!\verb!ifsynopsis!.
%Значение условия задаётся в~основном файле документа (\verb!synopsis.tex! для
%автореферата).
%\else
%Этот абзац появляется только в~диссертации.
%Через проверку условия \verb!\!\verb!ifsynopsis!, задаваемого в~основном файле
%документа (\verb!dissertation.tex! для диссертации), можно сделать новую
%команду, обеспечивающую появление цитаты в~диссертации, но~не~в~автореферате.
%\fi

% {\progress}
% Этот раздел должен быть отдельным структурным элементом по
% ГОСТ, но он, как правило, включается в описание актуальности
% темы. Нужен он отдельным структурынм элемементом или нет ---
% смотрите другие диссертации вашего совета, скорее всего не нужен.

{\aim} данной работы является исследование периодических режимов в полносвязной цепи генераторов Мэки--Гласса.

Для~достижения поставленной цели необходимо было решить следующие {\tasks}.
\begin{enumerate}[beginpenalty=10000] % https://tex.stackexchange.com/a/476052/104425
	\item Описать условия, при которых уравнение Мэки--Гласса \eqref{eq:mg_equation_1:intro} имеет периодическое решение.
	\item Исследовать полносвязную систему генераторов Мэки--Гласса, описать условия и ограничения на параметры системы, при которых она имеет решение в виде дискретной бегущей волны.
	\item Исследовать полносвязную систему генераторов Мэки--Гласса, описать условия и ограничения на параметры системы, при которых она имеет решение в виде, соответствующем режиму двухкластерной синхронизации.
\end{enumerate}


{\novelty}
\begin{enumerate}[beginpenalty=10000] % https://tex.stackexchange.com/a/476052/104425
  \item Впервые получены асимптотические формулы решения уравнения \eqref{eq:mg_equation_1:intro} по параметру $\gamma \gg 1$ для достаточно больших значений отношения $a / b$.
  \item Впервые доказано существование периодических режимов в виде дискретной бегущей волны в полносвязной цепи генераторов Мэки--Гласса, а также сформулированы и доказаны условия их существования в виде ограничения на параметры соответствующей системы дифференциальных уравнений с запаздыванием.
  \item Впервые доказано существование периодических режимов двухкластерной синхронизации в полносвязной цепи генераторов Мэки--Гласса, а также сформулированы и доказаны условия их существования в виде ограничения на параметры соответствующей системы дифференциальных уравнений с запаздыванием. 
  %TODO: сказать про скользящие траектории.
\end{enumerate}

{\influence} Полученные в диссертации результаты могут стать основой для дальнейших исследований в области нелинейной динамики и быть использованы специалистами для решения широкого спектра научных и прикладных задач.

{\defpositions}
\begin{enumerate}[beginpenalty=10000] % https://tex.stackexchange.com/a/476052/104425
  \item Вывод и доказательство асимптотических формул периодического решения уравнения Мэки--Гласса \eqref{eq:mg_equation_1:intro} по параметру $\gamma \gg 1$ для достаточно больших значений (порядка $e^b$) отношения $a / b$ \cite{wosbib1}.
  \item Доказательство существования периодических режимов в виде дискретной бегущей волны в полносвязной цепи релейных генераторов Мэки--Гласса, формулировка и доказательство условий их существования в виде ограничения на параметры соответствующей системы дифференциальных уравнений с запаздыванием \cite{wosbib2}.
  \item Доказательство существования (в смысле обобщённого решения системы дифференциальных уравнений с разрывной правой частью) периодических режимов двухкластерной синхронизации в полносвязной цепи релейных генераторов Мэки--Гласса, формулировка и доказательство условий их существования в виде ограничения на параметры соответствующей системы дифференциальных уравнений с запаздыванием \cite{scbib1}.
\end{enumerate}

%В папке Documents можно ознакомиться с решением совета из Томского~ГУ
%(в~файле \verb+Def_positions.pdf+), где обоснованно даются рекомендации
%по~формулировкам защищаемых положений.

{\reliability} полученных результатов обеспечивается строгими математическими доказательствами, приведёнными в работе. % Результаты находятся в соответствии с результатами, полученными другими авторами.

\nocite{scbib1, wosbib1, wosbib2}

{\probation}
Основные результаты работы докладывались~на следующих конференциях и семинарах.
\begin{enumerate}
	\item Конференция <<Topological Methods in Dynamics and Related Topics VII>>, Нижний Новгород, 2024 \cite{confbib6}.
	\item Семинар по качественной теории дифференциальных уравнений в московском государственном университете имени М.В.~Ломоносова, 29 ноября 2024 года. \cite{Sergeev2024},\\\texttt{https://www.elibrary.ru/item.asp?id=75144298}
	\item Семинар по нелинейной динамике Ярославского государственного университета, 19 сентября 2024 г.,\\\texttt{https://cis.uniyar.ac.ru/index.php/event/460}.
	\item Научный семинар лаборатории динамических систем и приложений НИУ ВШЭ в Нижнем Новгороде, 25 сентября 2024 г.,\\\texttt{https://nnov.hse.ru/bipm/dsa/semtmd}.
	\item Международная конференция по дифференциальным уравнениям и динамическим системам DIFF--2022, Суздаль, 2022  \cite{confbib1}.
	\item Конференция <<Нелинейные дни в Саратове для молодых>>, Саратов, 2023 \cite{confbib2}.
	\item Международная конференция по дифференциальным уравнениям и динамическим системам DIFF-2024, Суздаль, 2024 \cite{confbib3}.
	\item Конференция <<Satellite International Conference on Nonlinear Dynamics {\&} Integrability>>, Ярославль, 2022 \cite{confbib4}.
	\item Конференция <<Integrable Systems and Nonlinear Dynamics>> (ISND – 2024), Ярославль, 2024 \cite{confbib5}.
\end{enumerate}

%{\contribution} Автор принимал активное участие \ldots

% \vspace{-11em}

{\publications} Основные результаты по теме диссертации изложены в 9 печатных изданиях, 3 из которых \cite{wosbib1,wosbib2,scbib1} изданы в журналах, рекомендованных ВАК, 3 "--- в~периодических научных журналах, индексируемых Web of~Science и Scopus \cite{wosbib1,wosbib2,scbib1}, 6 "--- в~тезисах докладов \cite{confbib1,confbib2,confbib3,confbib4,confbib5,confbib6}.

%
%%%% Реализация пакетом biblatex через движок biber
%\begin{refsection}[bl-author, bl-registered]
%    % Это refsection=1.
%    % Процитированные здесь работы:
%    %  * подсчитываются, для автоматического составления фразы "Основные результаты ..."
%    %  * попадают в авторскую библиографию, при usefootcite==0 и стиле `\insertbiblioauthor` или `\insertbiblioauthorgrouped`
%    %  * нумеруются там в зависимости от порядка команд `\printbibliography` в этом разделе.
%    %  * при использовании `\insertbiblioauthorgrouped`, порядок команд `\printbibliography` в нём должен быть тем же (см. biblio/biblatex.tex)
%    %
%    % Невидимый библиографический список для подсчёта количества публикаций:
%    \printbibliography[heading=nobibheading, section=1, env=countauthorvak,          keyword=biblioauthorvak]%
%    \printbibliography[heading=nobibheading, section=1, env=countauthorwos,          keyword=biblioauthorwos]%
%    \printbibliography[heading=nobibheading, section=1, env=countauthorscopus,       keyword=biblioauthorscopus]%
%    \printbibliography[heading=nobibheading, section=1, env=countauthorconf,         keyword=biblioauthorconf]%
%    \printbibliography[heading=nobibheading, section=1, env=countauthorother,        keyword=biblioauthorother]%
%%    \printbibliography[heading=nobibheading, section=1, env=countregistered,         keyword=biblioregistered]%
%%    \printbibliography[heading=nobibheading, section=1, env=countauthorpatent,       keyword=biblioauthorpatent]%
%    \printbibliography[heading=nobibheading, section=1, env=countauthorprogram,      keyword=biblioauthorprogram]%
%    \printbibliography[heading=nobibheading, section=1, env=countauthor,             keyword=biblioauthor]%
%    \printbibliography[heading=nobibheading, section=1, env=countauthorvakscopuswos, filter=vakscopuswos]%
%    \printbibliography[heading=nobibheading, section=1, env=countauthorscopuswos,    filter=scopuswos]%
%    %
%    \nocite{*}%
%    %
%    {\publications} Основные результаты по теме диссертации изложены в~\arabic{citeauthor}~печатных изданиях,
%    \arabic{citeauthorvak} из которых \cite{wosbib1,wosbib2,scbib1} изданы в журналах, рекомендованных ВАК\sloppy%
%    \ifnum \value{citeauthorscopuswos}>0%
%        , \arabic{citeauthorscopuswos} "--- в~периодических научных журналах, индексируемых Web of~Science и Scopus \cite{wosbib1,wosbib2,scbib1}\sloppy%
%    \fi%
%    \ifnum \value{citeauthorconf}>0%
%        , \arabic{citeauthorconf} "--- в~тезисах докладов \cite{confbib1,confbib2,confbib3,confbib4,confbib5}.
%    \else%
%        .
%    \fi%
%    \ifnum \value{citeregistered}=1%
%        \ifnum \value{citeauthorpatent}=1%
%            Зарегистрирован \arabic{citeauthorpatent} патент.
%        \fi%
%        \ifnum \value{citeauthorprogram}=1%
%            Зарегистрирована \arabic{citeauthorprogram} программа для ЭВМ.
%        \fi%
%    \fi%
%    % К публикациям, в которых излагаются основные научные результаты диссертации на соискание учёной
%    % степени, в рецензируемых изданиях приравниваются патенты на изобретения, патенты (свидетельства) на
%    % полезную модель, патенты на промышленный образец, патенты на селекционные достижения, свидетельства
%    % на программу для электронных вычислительных машин, базу данных, топологию интегральных микросхем,
%    % зарегистрированные в установленном порядке.(в ред. Постановления Правительства РФ от 21.04.2016 N 335)
%\end{refsection}%
\begin{refsection}[bl-author, bl-registered]
    % Это refsection=2.
    % Процитированные здесь работы:
    %  * попадают в авторскую библиографию, при usefootcite==0 и стиле `\insertbiblioauthorimportant`.
    %  * ни на что не влияют в противном случае
    \nocite{vakbib2}%vak
    \nocite{patbib1}%patent
    \nocite{progbib1}%program
    \nocite{bib1}%other
    \nocite{confbib1}%conf
\end{refsection}%
    %
    % Всё, что вне этих двух refsection, это refsection=0,
    %  * для диссертации - это нормальные ссылки, попадающие в обычную библиографию
    %  * для автореферата:
    %     * при usefootcite==0, ссылка корректно сработает только для источника из `external.bib`. Для своих работ --- напечатает "[0]" (и даже Warning не вылезет).
    %     * при usefootcite==1, ссылка сработает нормально. В авторской библиографии будут только процитированные в refsection=0 работы.

%При использовании пакета \verb!biblatex! будут подсчитаны все работы, добавленные
%в файл \verb!biblio/author.bib!. Для правильного подсчёта работ в~различных
%системах цитирования требуется использовать поля:
%\begin{itemize}
%        \item \texttt{authorvak} если публикация индексирована ВАК,
%        \item \texttt{authorscopus} если публикация индексирована Scopus,
%        \item \texttt{authorwos} если публикация индексирована Web of Science,
%        \item \texttt{authorconf} для докладов конференций,
%        \item \texttt{authorpatent} для патентов,
%        \item \texttt{authorprogram} для зарегистрированных программ для ЭВМ,
%        \item \texttt{authorother} для других публикаций.
%\end{itemize}
%Для подсчёта используются счётчики:
%\begin{itemize}
%        \item \texttt{citeauthorvak} для работ, индексируемых ВАК,
%        \item \texttt{citeauthorscopus} для работ, индексируемых Scopus,
%        \item \texttt{citeauthorwos} для работ, индексируемых Web of Science,
%        \item \texttt{citeauthorvakscopuswos} для работ, индексируемых одной из трёх баз,
%        \item \texttt{citeauthorscopuswos} для работ, индексируемых Scopus или Web of~Science,
%        \item \texttt{citeauthorconf} для докладов на конференциях,
%        \item \texttt{citeauthorother} для остальных работ,
%        \item \texttt{citeauthorpatent} для патентов,
%        \item \texttt{citeauthorprogram} для зарегистрированных программ для ЭВМ,
%        \item \texttt{citeauthor} для суммарного количества работ.
%\end{itemize}
%% Счётчик \texttt{citeexternal} используется для подсчёта процитированных публикаций;
%% \texttt{citeregistered} "--- для подсчёта суммарного количества патентов и программ для ЭВМ.
%
%Для добавления в список публикаций автора работ, которые не были процитированы в
%автореферате, требуется их~перечислить с использованием команды \verb!\nocite! в
%\verb!Synopsis/content.tex!.