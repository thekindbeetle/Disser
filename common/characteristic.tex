\textbf{Актуальність теми дослідження.} Розвиток прикладних інформаційних технологій на ринку транспортних послуг зумовлений посиленням жорсткої економічної конкуренції та запитом на підвищення екологічності, комфорту та ефективності роботи персоналу.

\ifsynopsis
\else
Сьогодні, світові виробники автомобілів, електроніки та телекомунікаційних технологій створюють та використовують компʼютерні інформаційні системи у спроектованих та діючих транспортних засобах. За останні десятиліття, більшість автомобілів набуло оснащення інтерактивними інформаційними системами, що включають аудіо та відео системи, супутникові навігаційні системи, гарнітури телефонії і контроль над кліматом та технічним станом автомобіля. Не дивлячись на те, що такі системи обладнанні дисплеєм, голосова взаємодія з водієм стає все більш широко використовуваною в автомобілях, що допомагає збільшити кількість контрольованих функцій і систем, кнопки яких не можуть бути встановлені на рульовому колесі та приладовій панелі, оскільки обмежено простір. Голосова технологія також дозволяє водіям не відволікатись від управління, знижуючи ймовірність виникнення небезпечних ситуацій на дорозі та підвищуючи безпеку руху. Власними назвами систем голосового управління володіють такі бренди, як Mercedes-Benz, Ford, Cadillac. На марках автомобілів Audi, BMW, Kia, Lexus установлені системи голосового управління для зручності і забезпечення комфорту водіїв. Для систем голосового управління характерна різниця, що полягає у кількості підтримуваних мов, належному рівню при розпізнаванні команд, кількості реалізації функцій управління. Найбільшою кількістю мов володіє система «Ford Sync», крім того арсенал включає і російську мову, але української – немає.
\fi

У звʼязку з дедалі активнішим використанням природного інтерфейсу і зокрема голосу для спілкування водія з технічними засобами зросло і значення систем голосового управління в самому автомобілі як носія інформації у системах диспетчерського контролю за рухом автотранспорту при здійсненні етапів дистрибуції «склад – дорога – точка доставки».

Не дивлячись на інтенсивний розвиток систем диспетчерського контролю за рухом автотранспорту при взаємодії із водієм, саме голосова інформація потребує формалізації у випадку проведення автоматизації таких систем. Проте існуючі розробки в сфері формалізації голосової інформації поки не пристосовані для аналізу мовлення водіїв, з метою покращення та полегшення їх взаємодії з диспетчерською системою. Саме модель голосової взаємодії водія в системах диспетчерського контролю потребує автоматизації для підвищення ефективності процесу дистрибуції.

Фінальна доставка до дверей клієнта, відома як «остання миля», є одним з найдорожчих та найскладніших у організації дистрибуції. Під час виконання доставки завжди відбуваються ті чи інші відхилення від плану, яким би оптимальним він не був, подібні відхилення в кожному випадку потребують коригування плану через комунікацію з диспетчером. Водії-експедитори та кур'єри скоріше починають виконувати доставки поза планом, якщо процеси комунікації з диспетчером та коригування планів недостатньо прості та ефективні. Для задачі дистрибуції може бути важко забезпечити постійний доступ до мережі інтернет, оскільки доставка може відбуватися до місць/регіонів, де навіть мобільний GPRS інтернет відсутній, або має надто низьку швидкість передачі даних для роботи зі звуком.

Інформаційних технологій які забезпечують автоматизацію голосової взаємодії в системах дистрибуції розроблено не достатньо.

Все це робить тему дисертаційного дослідження інформаційних технологій формалізації голосової інформації в системах диспетчерського контролю за рухом автотранспорту \textbf{актуальною}.

Питаннями автоматизації систем голосового управління займалися такі вчені, як: Бондарос Ю.Г., Волков А.В., Єгорченков А.В., Кравченко А.П., Козлов О.С., Корсун О.М., Любімов А.М, Пилипенко В.В., Робейко В.В., Тесля Ю.М., Чорний О.Ю., Чучупал В.Я., Фінаєв І.М., Яцко А.А., Britz D., Deng L., Heisterkamp P., Hinton G., Jonsson I.-M., Kim Y., LeCun Y., Saini P., Yu D., Zhang X., Zhao J.J. та багато інших. Зокрема, результати досліджень голосового управління, заснованих на теорії несилової взаємодії та рефлекторної системи голосового управління належать таким вченим як: Тесля Ю.М., Пилипенко В.В., Чорний О.Ю., Єгорченков А.В.

\textbf{Звʼязок роботи з науковими програмами і планами.}

Дисертаційна робота виконана відповідно до пріоритетного напряму розвитку інформаційних та комунікаційних технологій, що визначені в Законі України «Про пріоритетні напрями розвитку науки і техніки» на період до 2020 року та тематичного плану науково-дослідних робіт Київського національного університету імені Тараса Шевченка в рамках науково-дослідної роботи держбюджетної теми університету «Розробка теоретико-методологічних основ впровадження систем управління проектами для розвитку підприємств і організацій» (№ держреєстрації 0117U002694), у яких автор брав участь як виконавець.

\textbf{Обʼєктом дослідження} є процеси автоматизації голосової взаємодії в системах диспетчерського контролю за рухом автотранспорту.

\textbf{Предмет дослідження} – моделі і методи формалізації голосової взаємодії в системах диспетчерського контролю за рухом автотранспорту в системі дистрибуції.

\textbf{Метою дослідження} ідвищення ефективності управління процесом дистрибуції на основі розробки та використання інформаційної технології формалізації голосової інформації в системах диспетчерського контролю за рухом автотранспорту.

Аналіз науково-технічної задачі: розробки моделей і методів формалізації голосової інформації в системах диспетчерського контролю за рухом автотранспорту дозволив сформулювати ряд \textbf{завдань досліджень}, вирішення яких забезпечить досягнення сформульованої мети:

\begin{itemize}
	\item здійснити аналіз сучасних інформаційних систем обробки та формалізації голосової інформації та визначити теоретико-методологічні засади формалізації голосової інформації в системах дистрибуції;
	\item дослідити процес автоматизації руху автотранспорту в дистрибуції та принципи побудови рефлекторної системи голосової взаємодії;
	\item розробити модель та методи формалізації голосової взаємодії в системах диспетчерського контролю за рухом автотранспорту в системі дистрибуції;
	\item провести експериментальні дослідження формалізації голосової інформації, отриманої від водіїв, для оптимізації диспетчерського контролю за рухом автотранспорту при виконанні процесів доставки в системі дистрибуції.
\end{itemize}

\textbf{Методи дослідження}, застосованя для вирішення поставлених завдань: для опису моделі голосової взаємодії — теорія графів, для вдосконалення методу інтелектуальних рефлекторних систем - теорія інформації та теорія несилової взаємодії, для вдосконалення методу згорткових нейронних мереж - теорія штучних нейронних мереж та методи обробки природної мови.

\textbf{Наукова новизна отриманих результатів} полягає в тому, що вперше вирішено наукову проблему інтеграції моделей і методів формалізації голосової інформації з управлінням процесом дистрибуції в єдиній системі формалізації голосової інформації в системах диспетчерського контролю за рухом автотранспорту. При цьому:

\begin{itemize}
	\item вперше створено метод формалізації голосової інформації в допоміжних системах диспетчеризації автотранспорту з використанням інтелектуальних рефлекторних систем, що дозволяє автоматизувати процес голосової комунікації;
	\item удосконалено модель голосової взаємодії субʼєктів дистрибуції в системах диспетчерського контролю за рухом автотранспорту, за рахунок представлення у вигляді повного графу сценаріїв усіх етапів дистрибуції «склад – дорога – точка доставки», що дозволяє виділити контексти голосової взаємодії для підвищення точності подальшої формалізації;
	\item набув подальшого розвитку метод структурної ідентифікації згорткових нейронних мереж для класифікації голосових команд для формалізації голосової інформації в системах диспетчерського контролю за рухом автотранспорту, за рахунок його адаптації для роботи з фонемним текстом, що дозволяє класифікувати голосові команди без переведення голосу в лексичний текст;
	\item отримав подальший розвиток метод інтелектуальних рефлекторних систем для формалізації процесів взаємодії субʼєктів дистрибуції на основі використання теорії нейронних мереж, шо дає можливість оптимізувати значення параметрів шляхом навчання методом зворотного розповсюдження помилки.
\end{itemize}

\textbf{Практичне значення} отриманих результатів полягає в тому, що з використанням наукових результатів, закладається сучасний науково-практичний базис підвищення ефективності управління процесом дистрибуції на основі інформаційної технології формалізації голосової інформації в системах диспетчерського контролю за рухом автотранспорту. Розроблені на базі запропонованих особисто автором моделей і методів програмні засоби становлять практичний результат, який впроваджений на підприємстві ТОВ «Українські Інформаційні Технології».

\textbf{Особистий внесок здобувача.} Наукові положення, розробки та висновки дисертаційної роботи є результатом самостійно проведеного дослідження здобувача. Основні наукові результати, представлені в дисертації, отримані здобувачем особисто.

\textbf{Апробація результатів досліджень.} Основні положення дисертаційної роботи були апробовані на 5-х міжнародних науково-практичних конференціях, в тому числі:

\begin{itemize}
	\item XVII Мiжнародна науково-технiчна конференцiя «Системний аналiз та iнформацiйнi технологiї» (м. Київ, 2015 р.)
	\item XIII Міжнародна конференція «Управління проектами у розвитку суспільства», тема: «Проекти в умовах глобальних загроз, ризиків и викликів» (м. Київ, 2016 р.)
	\item ІІІ Міжнародна науково-практична конференція «Інформаційні технології та взаємодії» (м. Київ, 2016 р.)
	\item 16th EAGE International Conference on Geoinformatics - Theoretical and Applied Aspects (м. Київ, 2017 р.)
	\item ІV Міжнародна науково-практична конференція «Інформаційні технології та взаємодії» (м. Київ, 2017 р.)
\end{itemize}


\printbibliography[heading=countauthor, env=countauthor, keyword=biblioauthor, section=1]%
\printbibliography[heading=countauthorpaper, env=countauthorpaper, keyword=biblioauthor, notkeyword=biblioauthorconf, section=1]%
\printbibliography[heading=countauthorvak, env=countauthorvak, keyword=biblioauthorvak, section=1]%
\printbibliography[heading=countauthorindexed, env=countauthorindexed, keyword=biblioauthorvak, category=biblioauthoreng, section=1]%
\printbibliography[heading=countauthorconf, env=countauthorconf, keyword=biblioauthorconf, section=1]%
\printbibliography[heading=countauthornotvak, env=countauthornotvak, keyword=biblioauthornotvak, section=1]%
\printbibliography[heading=countauthoreng, env=countauthoreng, notkeyword=biblioauthorvak, category=biblioauthoreng, section=1]%

\textbf{Публікації.} 
За результатами дослідження опубліковано 
\formbytotal{citeauthor}{науков}{у працю}{і праці}{их праць} загальним обсягом 9,8 д.а. – 
\formbytotal{citeauthorpaper}{науков}{у статтю}{і статті}{их статей} (3,7 д. а.), у тому числі 
\formbytotal{citeauthorvak}{}{}{}{} у фахових виданнях (з них 
\formbytotal{citeauthorindexed}{стат}{тю}{ті}{ей} у виданнях, які входять до наукометричних баз даних) і 
\formbytotal{citeauthoreng}{}{}{}{} – в іноземному науковому виданні, та 
\formbytotal{citeauthorconf}{роб}{ота}{оти}{іт} в матеріалах і тезах доповідей на наукових конференціях.


