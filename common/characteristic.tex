
% common/newnames.tex & common/renames.tex
{\actuality} Биологические модели занимают важное место в теории динамических систем, охватывая широкий спектр явлений, происходящих в биологических процессах. Эти модели включают как те, которые описывают концентрацию различных веществ в биологических системах, таких как химические реакции в клетках или процессы обмена веществ в организмах, так и популяционные уравнения, моделирующие изменение численности и структуру популяций живых существ. Обычно модели популяционной динамики описываются дифференциальными или дифференциально-разностными уравнениями. 
Эти модели с течением времени претерпели значительные изменения и усложнения.

Одной из первых популяционных моделей была модель Томаса Мальтуса, который предположил, что рост популяции, не сдерживаемой ограничениями на ресурсы, будет экспоненциальным. % Массовые вымирания, такие как эпидемии, голод и войны, в модели Мальтуса объясняются тем, что ресурсы, необходимые для выживания, не могут воспроизводиться с такой же скоростью, как и популяция, что приводит к перенаселению \cite{Malthus1798}.

Пьер Ферхюльст в работе \cite{Verhulst1838} предложил добавить в модель Мальтуса квадратичное слагаемое, ограничивающее скорость роста популяции для больших значений её размера. Таким образом, он получил модель, называемую логистическим уравнением
\begin{equation}
\label{eq:intro:logistic}
	\dot{x}=\lambda x\left(1-\frac{x}{K}\right),
\end{equation}
где функция $x(t)$ показывает текущую плотность (или численность) популяции, параметр $\lambda$ характеризует скорость роста популяции, а параметр $K$ --- ёмкость среды.

Позже рассматривались различные усовершенствования приведённой модели. Одна из идей усложнения правой части уравнения \eqref{eq:intro:logistic} --- добавление запаздывания\footnote{В работе приняты следующие соглашения: явно указывается только запаздывающий аргумент функции: так, например, вместо $x(t)$ пишем $x$. Дифференцирование по времени обозначается точкой над функцией.}. Так в 1948 году Джордж Хатчинсон предложил модификацию логистического уравнения, обладающую запаздыванием по времени и простейшим образом учитывающую возрастную структуру популяции:
%
\begin{equation}
\label{eq:intro:hutch}
	\dot{x}=\lambda x\left(1 - \frac{x(t-\tau)}{K}\right),
\end{equation}
%
где $x(t)$ --- плотность популяции, $\lambda$ --- скорость роста популяции, $K$ --- ёмкость среды, запаздывание $\tau$ --- возраст половозрелости \cite{Hutchinson1948}. %https://encyclopediaofmath.org/wiki/Hutchinson_equation


Помимо уравнения Хатчинсона, рассматривались различные его обобщения с заменой квадратичной зависимости в правой части на более сложную. Например, в работе \cite{Glyzin2007} предлагается модель, более тонко учитывающая возрастную структуру
\begin{equation}
\label{eq:intro:glyzin2007}
	\dot{x}=\lambda \left(1 - \frac{1}{K}\sum\limits_{i = 1}^{m} a_i x(t-\tau_i)\right) x.
\end{equation}
%
В работе \cite{Kaschenko2012} исследуется обобщение модели \eqref{eq:intro:hutch}
\begin{equation}
	\dot{x} = \lambda \left(1 - \int\limits_{h_1}^{h_2}dr(\tau)x(t - \tau)\right) x,
\end{equation}
где $r(\tau)$ --- монотонная неотрицательная функция, $\lambda, h_1, h_2$ --- положительные параметры.

В работе \cite{Kolesov2010} предлагается обобщение модели \eqref{eq:intro:hutch}, определённое уравнением
\begin{equation}
\label{eq:intro:hutch_modified}
	\dot{x} = \lambda f(x(t - 1)) x,
\end{equation}
%
где бесконечно дифференцируемая при $t \geq 0$ функция $f$ обладает свойствами
\[
f(0) = 1, \quad f(x) = -a_0 + \sum\limits_{k = 1}^{\infty} \frac{a_k}{x^k}, \quad x \to +\infty, \quad a_0 > 0.
\]
После экспоненциальной замены $x = e^{\lambda y}$ уравнение \eqref{eq:intro:hutch_modified} принимает вид
\begin{equation}
\label{eq:intro:hutch_modified_exp}
\dot{y} = f(e^{\lambda y(t - 1)}).
\end{equation}
При достаточно большом $\lambda$ правая часть становится близкой к релейной функции, меняющей значение при смене знака аргумента. Типичным примером функции $f$ служит
\[
f(x) = \dfrac{1 - x}{1 + cx}, \quad c = \text{const} > 0.
\]
В работе \cite{Kolesov2010} отмечается, что при больших значениях параметра $\lambda$ уравнение~\eqref{eq:intro:hutch_modified} обладает лучшими биологическими характеристиками, чем уравнение~\eqref{eq:intro:hutch}: это связано с тем, что решение уравнения \eqref{eq:intro:hutch} обладает очень глубоким минимумом, который в биологическом контексте означает вымирание популяции после первого же всплеска численности (см. рис. \ref{fig:intro:hutch}).

%\fixme{Отметим, что в отличие от модели Мальтуса, модель \eqref{eq:intro:logistic} и все перечисленные её обобщения обладают насыщением, т.е. плотность популяции, характеризуемая функцией $x(t)$, ограничена сверху.}

% \fixme{Это корректно?}

Отметим, что приведённые выше модели \eqref{eq:intro:logistic} -- \eqref{eq:intro:hutch_modified} обладают свойствами, характерными для популяционных моделей. Так, решения с положительными начальными условиями остаются положительным при $t > 0$. Уравнения \eqref{eq:intro:logistic} -- \eqref{eq:intro:glyzin2007} имеют положительное состояние равновесия, соответствующее ёмкости среды.

%\fixme{Однако, уравнение \eqref{eq:intro:hutch_modified} является моделью с насыщением, т.~е., как видно из уравнения \eqref{eq:intro:hutch_modified_exp}, любое решение ограничено константой, не зависящей от параметра $\lambda$.}
	
\begin{figure}
	\centering
	\includegraphics[width=0.7\textwidth]{hutch.png}
	\caption{Слева: решение уравнения \eqref{eq:intro:hutch} при $\lambda = 2.5$, $K = 1$, $\tau = 1$; справа: решение уравнения \eqref{eq:intro:hutch_modified} при $\lambda = 2.5$, $f(x) = \frac{1 - x}{1 + 0.2x}$. Решение (а) обладает близким к нулю минимумом, что в биологическом контексте означает полное вымирание популяции; у решения (б) этот недостаток отсутствует. Иллюстрация взята из статьи \cite{Kolesov2010}.}
	\label{fig:intro:hutch}
\end{figure}

Нелинейности, представленные рациональными функциями, также встречаются в различных биологических системах, включая генные сети, которые будут описаны ниже. В данной диссертационной работе исследуется модель Мэки--Гласса, которая также имеет рациональную нелинейность.

Уравнениями Мэки--Гласса называют две модели с запаздыванием \cite{Mackey1977, Glass1988}:
\begin{equation}
	\label{eq:mg_equation_1:intro}
	\dot{v}=-b v+\frac{a \theta^{\gamma} v(t-\tau)}{\theta^{\gamma}+(v(t-\tau))^{\gamma}},
\end{equation}
\begin{equation}
	\label{eq:mg_equation_2:intro}
	\dot{v}=-b v+\frac{a \theta^{\gamma}}{\theta^{\gamma}+(v(t-\tau))^{\gamma}},
\end{equation}
где параметры $a, b, \gamma, \theta$ --- положительные вещественные числа.

Эти модели были предложены в работе \cite{Mackey1977} для описания регуляторных функций в процессах кроветворения. Уравнения \eqref{eq:mg_equation_1:intro} и \eqref{eq:mg_equation_2:intro} различаются формой нелинейности в слагаемом с запаздыванием (см. рис. \ref{fig:mg_delay_form}): в уравнении \eqref{eq:mg_equation_1:intro} она имеет форму <<горба>>, в то время как в уравнении \eqref{eq:mg_equation_2:intro} монотонно убывает. В данной работе под уравнением Мэки--Гласса будет пониматься уравнение \eqref{eq:mg_equation_1:intro}.

\begin{figure}
	\centering
	\includegraphics[width=\textwidth]{mg_delay_form.eps}
	\caption{Графики нелинейных слагаемых в уравнении \eqref{eq:mg_equation_1:intro} слева и в уравнении \eqref{eq:mg_equation_2:intro} справа (как функций от переменной $v(t - \tau)$).
	}
	\label{fig:mg_delay_form}
\end{figure}

Биологический смысл данной модели следующий: функция $v(t)$ --- плотность циркулирующих в крови человека нейтрофилов (вид лейкоцитов) в клетках на кг массы тела, $b$ --- скорость случайного распада нейтрофилов, положительное слагаемое означает текущий приток клеток в кровь, возникающий в ответ на запрос, создавшийся в некоторый момент $\tau$ времени назад в прошлом.

В широком диапазоне изменений уровня циркулирующих нейтрофилов скорость образования нейтрофилов падает с увеличением их плотности. Однако благодаря действию различных факторов можно ожидать, что при очень низких уровнях нейтрофилов скорость их образования будет падать, приближаясь к нулю \cite[с. 85]{Mackey1977}. Форма нелинейности выбирается из приведённых соображений.

Уравнение Мэки--Гласса исследовалось во множестве работ: например, см. \cite{Junges2012, Su2011, Wu2007, Kubyshkin2016, Krisztin2020, Bartha2021}, а также статью \cite{Berezansky2012}, в которой представлен обширный обзор различных известных (к 2012 г.) результатов, связанных с исследованием уравнения Мэки--Гласса и его обобщений, со ссылками на соответствующие работы. 

В оригинальной работе \cite{Mackey1977} приведены численные решения, как в случае периодических, так и непериодических колебаний. В работах \cite{Krisztin2020, Bartha2021} исследуются периодические решения уравнения Мэки--Гласса. В частности, в \cite{Bartha2021} доказано (при некоторых ограничениях на параметры и множество начальных функций) существование и единственность орбитально устойчивого предельного цикла. В работе \cite{Kubyshkin2016} изучаются периодические решения уравнения Мэки--Гласса, бифурцирующие из его единственного состояния равновесия при изменении параметров уравнения.

Уравнение допускает различные обобщения. Так, в работе \cite{Berezansky2006} исследуется аналог уравнения Мэки--Гласса с переменным запаздыванием. В работе \cite{Liz2002} изучаются асимптотические свойства решений уравнений типа Мэки--Гласса с нелинейностью похожей формы. В \cite{Wu2007} рассмотрено уравнение Мэки--Гласса с параметрами и запаздыванием, зависящими от времени, установлены условия существования положительного периодического решения. В \cite{Huang2024} изучается стохастическая версия уравнения с несколькими запаздываниями. 

Уравнение Мэки--Гласса и его различные модификации широко использовались для моделирования функционирования электрогенераторов \cite{Tateno2012, Namajunas1995, Glyzin2018, Glyzin2018a}, а также для симуляции хаотического сигнала \cite{Grassberger1983, Amil2015, Amil2015a, Shahverdiev2006}.

Помимо биологических моделей, описываемых одним уравнением, представляют интерес модели, получаемые объединением элементов, функционирующих некоторым известным образом, в сеть. Следует отметить, что цепочки идентичных нелинейных осцилляторов используются в качестве математических моделей в различных областях естествознания: биофизике, экологии, оптике, химической кинетике, нейродинамике, генной инженерии и др. \cite{Glyzin2022}. Можно выделить две естественные структуры для связи элементов сети: кольцо, где элемент связан с соседними элементами, и полносвязную сеть (рис. \ref{fig:full_mesh:intro}), где каждый элемент сети связан со всеми остальными.

\begin{figure}[ht]
	\centering
	\includegraphics[width=0.5\textwidth]{mg_generator_full.eps}
	\caption{Полносвязная сеть осцилляторов. Каждый осциллятор является передающим и принимающим для всех остальных осцилляторов в~сети.}
	\label{fig:full_mesh:intro}
\end{figure}

Примером такой системы являются искусственные генные сети. Интерес к искусственным генным осцилляторам вызван тем обстоятельством, что они являются упрощёнными моделями таких ключевых биологических процессов, как клеточный цикл и циркадные ритмы. Простейший генетический осциллятор, предложенный в \cite{Elowitz2000} и названный репрессилятором, состоит как минимум из трёх элементов, соединённых в кольцо \cite{Glyzin2017, GlyzinBook2018}. Функционирование такой сети описывается системой
\begin{equation}
	\label{eq:intro:repressilator}
	\dot{u}_j = -u_j + \dfrac{\alpha}{1 + u^{\gamma}_{j - 1}}, \quad j = 1, 2, 3,
\end{equation}
где $u_0 = u_3$, $\alpha, \gamma > 0$. Исследование генных сетей проводится в работах \cite{Likhoshvaj2003, Volokitin2004, Golubyatnikov2006, Buse2009, Buse2010}.

Аналогичным образом можно соединить в сеть элементы, функционирование которых описывается уравнением \eqref{eq:mg_equation_1:intro}. Такие системы были исследованы в работах \cite{Preobrazhenskaia2021, Tateno2012, Sano2007, Wan2009}. Так, в \cite{Sano2007} численно и экспериментально изучалась система из четырёх генераторов Мэки--Гласса, два из которых были вещательными, а два --- принимающими. В работе \cite{Wan2009} исследовалась потеря устойчивости состояния равновесия в этой системе, а также условия, при которых в результате бифуркации рождается устойчивый предельный цикл.

В диссертационной работе методами большого параметра исследуются периодические режимы, возникающие в полносвязной сети осцилляторов, функционирование которых описывается релейными уравнениями Мэки--Гласса. Доказывается существование периодических режимов специального вида: дискретных бегущих волн и двухкластерной синхронизации, описание которых даётся ниже.

{\methods} Для сложных моделей в случае, когда не удается отыскать решение непосредственным интегрированием, уместно использование специальных методов поиска решения. В частности, это относится к дифференциальным уравнениям с запаздывающим аргументом. Одна из идей упрощения исследования --- это переход к предельному объекту.

\textit{Переход к релейному уравнению.} Для упрощения исследования необходимо подобрать такую замену, чтобы при стремлении большого параметра $\gamma \to +\infty$ получилось релейное уравнение, у которого, с одной стороны, имеется достаточно сложная динамика (периодические решения различной структуры), а с другой стороны, при стремлении $\gamma \to +\infty$ можно было доказать сходимость решения исходного уравнения к периодическому решению релейной задачи. В частности, это получается сделать, если нелинейность в правой части уравнения близка к сигмоидальной функции, например, имеет вид $S_\gamma(u)=\frac{1}{1 + u^\gamma}$, $u > 0$ (см., например, \cite{Preobrazhenskaya2020, Glyzin2017, Krisztin2020, Bartha2021}). Такая функция в пределе при $\gamma\to+\infty$ устремляется к кусочно-постоянной функции, меняющей значение в 1. Тогда исходное уравнение можно заменить релейным, которое, как правило, существенно проще для анализа. После построения решения предельного уравнения доказывается существование асимптотически близкого решения исходной задачи.

Для исследуемого в первой части уравнения
\begin{equation}
\label{eq:intro:MG_norm1}
	\dot{x}=-\beta+\alpha\frac{e^{x(t-1)-x}}{1+e^{\gamma x(t-1)}}
\end{equation}
соответствующее релейное уравнение имеет вид
\[
\dot{x}=-\beta + \alpha e^{-x} F(\exp({x(t-1)})),
\]
где функция $F$ (см. рис. \ref{fig:F_relay_plot:intro}) задаётся формулой
\begin{equation}
	\label{eq:intro:F_relay}
	F(u)=\lim\limits_{\gamma\to +\infty}\frac{u}{1+u^{\gamma}} = 
	\begin{cases}
		u, & 0 \leq u < 1,\\
		\frac{1}{2}, & u = 1,\\
		0, & u > 1.
	\end{cases}
\end{equation}

\begin{figure}[ht]
	\centering
	\includegraphics[width=0.7\textwidth]{F_relay_plot_intro.eps}
	\caption{Релейная функция $F(x)$, задаваемая формулой \eqref{eq:intro:F_relay}.}
	\label{fig:F_relay_plot:intro}
\end{figure}

Другая идея состоит в том, чтобы найти решение в специальном виде. Остановимся на описании этой процедуры подробнее.

%\textit{Переход к предельному объекту.} В ряде случаев при исследовании сложного уравнения удается определить содержательный предельный объект при устремлении одного из параметров к бесконечности (см., например, \cite{Kolesov1997}). В частности, это получается сделать, если нелинейность в правой части уравнения близка к сигмоидальной функции, например, имеет вид $f_\gamma(u)=\frac{1}{1 + u^\gamma}$, $u > 0$ (см., например, \cite{Preobrazhenskaya2020, Glyzin2017, Krisztin2020, Bartha2021}). Такая функция в пределе при $\gamma\to+\infty$ устремляется к кусочно-постоянной функции, меняющей значение в 1. Тогда исходное уравнение можно подменить релейным, которое, как правило, существенно проще для анализа. После построения решения предельного уравнения можно попытаться доказать существование асимптотически близкого решения исходной задачи.

\textit{Поиск решений специального вида.} В настоящей работе используются два подхода к построению специальных решений системы дифференциально-разностных уравнений, разработанные в работах С. Д. Глызина и др. \cite{GlyKol2013, GlyKol2013a, Glyzin2014}: это построение дискретных бегущих волн (вторая глава диссертации) и поиск режимов кластерной синхронизации (третья глава диссертации). Опишем метод построения решений того и другого сорта в общем виде.

\textit{Дискретные бегущие волны.} Рассмотрим симметричную систему осцилляторов, связанных либо в кольцо, либо в полносвязную сеть. Дискретной бегущей волной называют периодическое решение, все компоненты которого представлены одной и той же периодической функцией $u(t)$ со сдвигом по времени, кратным некоторому параметру $\Delta$.

% Таким образом, решение представляется в следующем виде:

Для полносвязной сети релейных осцилляторов Мэки--Гласса, рассматриваемой во второй главе диссертации, соответствующая система имеет вид
%
\begin{equation}
	\label{eq:intro:mg_full_renormed}
	\dot{u}_j(t) = -\beta u_j(t) + \alpha F \left(u_j(t - 1) + \sum\limits_{k = 0, k\neq j}^N u_k(t)\right), \text{ где } j = 0, 1, \dots, N,
\end{equation}
функция $F$ (как и в первой части) задаётся системой \eqref{eq:intro:F_relay}.

После подстановки $u_j(t) = u(t + j\Delta)$ в систему \eqref{eq:intro:mg_full_renormed} получаем вспомогательное уравнение

\begin{equation}
	\label{eq:intro:mg_auxiliary}
	\dot{u}(t) =-\beta u(t) + \alpha F\left(u(t - 1) + \sum_{s=1}^{N}u(t-s\Delta)\right).
\end{equation}

Дискретной бегущей волне соответствует периодическое решение уравнения \eqref{eq:intro:mg_auxiliary}, период (не обязательно главный) которого кратен параметру $\Delta$. Отметим, что из существования одного решения в виде дискретной бегущей волны следует одновременное существование сразу $N!$ таких режимов, получаемых перестановкой компонент исходного решения, где $N$ --- количество уравнений в системе.


%Дискретные бегущие волны в кольцевых системах описаны в работах \cite{GlyKol2013a, Glyzin2016, Kolesov2016}, а в полносвязных --- в работах \cite{Glyzin2022, Glyzin2022a, Preobrazhenskii2024}.
%
%Пусть $n$ осцилляторов $x_1, \ldots, x_n$ описываются симметричной кольцевой системой дифференциальных уравнений
%\begin{equation}
%\label{eq:intro:sys_Phi_circ}
%	\dot{x}_j=\Phi(x_j, x_{j-1}), \quad j=1, \ldots, n, \quad x_{0} = x_{n},
%\end{equation}
%где $\Phi:\mathbb{R}^2\to\mathbb{R}$. 
%
%Дискретной бегущей волной называют периодическое решение, все компоненты которого представлены одной и той же периодической функцией $x(t)$ со сдвигом по времени, кратным некоторому параметру $\Delta$. Таким образом, решение представляется в следующем виде:
%%
%\begin{equation}
%\label{eq:intro:wave}
%	x_j(t) = x(t + j\Delta).
%\end{equation}
%%
%При подстановке \eqref{eq:intro:wave} в систему \eqref{eq:intro:sys_Phi_circ} получаем, что $x(t)$ удовлетворяет уравнению
%%
%\begin{equation}
%	\label{eq:intro:Phi_circ}
%	\dot{x}=\Phi(x, x(t-\Delta)).
%\end{equation}
%
%Условие $x_0 \equiv x_n$ требует выполнения тождества
%$x(t + n\Delta) \equiv x(t)$. Это значит, что величина $n\Delta$ кратна периоду $T = T(\Delta)$ функции $x(t)$. Следовательно, параметр $\Delta$ обязан удовлетворять уравнению периодов
%\begin{equation}
%	\label{eq_period_Delta}
%	p T(\Delta) = n\Delta
%\end{equation}
%при некотором целом $p \neq 0$.
%
%Таким образом, задача построения дискретных бегущих волн системы \eqref{eq:intro:sys_Phi_circ} сводится к поиску периодического решения вспомогательного дифференциально-разностного уравнения \eqref{eq:intro:Phi_circ} и подбору подходящего параметра $\Delta$, удовлетворяющего уравнению периодов (\ref{eq_period_Delta}). 

%В случае полносвязной системы осцилляторов соответствующая система дифференциальных уравнений имеет следующую форму:
%%
%\begin{equation}\label{eq:intro:sys_Psi_full}
%	\dot{x}_j= \Psi(x_j, x_{j-1}, \ldots, x_{j-n+1}), \quad j=1, \ldots, n, \quad x_j = x_{j+n}.
%\end{equation}
%%
%Здесь $\Psi:\mathbb{R}^{n}\to\mathbb{R}$. 
%Соответствующее вспомогательное уравнение с запаздываниями для поиска функции $x(t)$ принимает вид
%%
%\begin{equation}
%	\label{eq:intro:Psi_full}
%	\dot{x}= \Psi(x, x(t-\Delta), \ldots, x(t-n\Delta)).
%\end{equation}
%
%Для полносвязной системы \eqref{eq:intro:sys_Psi_full} последовательный сдвиг функции $x(t)$ по времени можно организовать для любой перестановки $(j_1,\ldots,j_n)$ номеров $(1,\ldots,n)$ осцилляторов $x_j$.
%Следовательно, вместе с дискретной бегущей волной \eqref{eq:intro:wave} существует $n!$ дискретных бегущих волн:
%\begin{equation}
%	\label{wave_full}
%	x_k(t) = x(t + j_k\Delta).
%\end{equation}
%
%Отметим, что системы \eqref{eq:intro:sys_Phi_circ} и \eqref{eq:intro:sys_Psi_full} могут изначально включать запаздывание по времени.

\textit{Режимы кластерной синхронизации.} 
В третьей главе диссертации рассматривается система из $N = n + m$ уравнений
\begin{equation}
	\label{eq:intro:mg_full_renormed_delta}
	\dot{u}_j(t) = -\beta u_j(t) + \alpha F_{\gamma} \left(u_j(t - 1) + \delta\sum\limits_{k = 1, k\neq j}^N u_k(t)\right), \text{ где } j = 1, \dots, N,
\end{equation}
где
\[
F_{\gamma}(u) = \dfrac{u}{1 + u^{\gamma}},
\]
с коэффициентом $\delta > 1$ в обратной связи. Решение вида 
\begin{equation}
	\label{eq:intro:cluster}
	u_1(t)=\ldots=u_m(t) = u(t),\quad u_{m+1}(t)=\ldots=u_{m+n}(t) = v(t),
\end{equation}
при котором часть осцилляторов описывается одной функцией, а остальные --- другой, называется режимом двухкластерной синхронизации. После подстановки \eqref{eq:intro:cluster} в систему \eqref{eq:intro:mg_full_renormed_delta} получается система из двух уравнений
%
\begin{equation}
	\label{eq:intro:system_uv}
	\begin{cases}
		\dot{u} = -\beta u + \alpha \, F_{\gamma} \big(u(t - 1) + \delta (m - 1) u + \delta n v\big),\\
		\dot{v} = -\beta v + \alpha \, F_{\gamma} \big(v(t - 1) + \delta m u + \delta (n - 1) v\big),
	\end{cases}
\end{equation}
%
для которой ищется периодическое решение.

%, для которой соответствующая система имеет вид
%Рассмотрим полносвязную систему из $N$ осцилляторов следующего вида:
%\[\dot{x}_j
%=\Xi(x_j;x_1,\ldots,\hat{x}_j,\ldots,x_{N}),\quad j=1,\ldots,N,\]
%где $\Xi:\mathbb{R}^{N}\to\mathbb{R}$ --- функция, симметричная по всем аргументам кроме, возможно, первого, символ $\hat{x}_j$ обозначает исключение соответствующего элемента из аргументов функции.
%
%Под кластерной синхронизацией будем понимать такое разбиение множества осцилляторов $x_1, \ldots, x_{N}$ на подмножества, что в каждом подмножестве осцилляторы функционируют одинаково.
%
%В настоящей работе строятся режимы двухкластерной синхронизации. Пусть $N = n + m$. Предположим, что $n$ осцилляторов описываются одной функцией, а $m$ осцилляторов --- другой:
%\begin{equation}
%	\label{eq:intro:cluster}
%	x_1(t)=\ldots=x_n(t)=x(t),\quad x_{n+1}(t)=\ldots=x_{n+m}(t)=y(t).
%\end{equation}
%При этом функции $x(t)$ и $y(t)$ удовлетворяют двумерной системе
%\[
%\begin{cases}
%	\dot{x}=\xi(x,y),\\
%	\dot{y}=\eta(x,y),
%\end{cases}
%\]
%где 
%\[
%\xi(x,y)=\Xi(x;\underbrace{x,\ldots,x}_{n-1},\underbrace{y\ldots,y}_{m}),\quad
%\eta(x,y)=\Xi(y;\underbrace{x,\ldots,x}_{n},\underbrace{y\ldots,y}_{m-1}).
%\]
%Отметим, что в случае существования режима \eqref{eq:intro:cluster}, сосуществует сразу $C_{n+m}^n$ подобных режимов.

Режимы двухкластерной синхронизации строятся в работах \cite{Glyzin2016a, Glyzin2022}.

\textit{Дифференциальные уравнения с разрывной правой частью.} Рассмотрим дифференциальное уравнение 
\[
\dot{x} = f_{\gamma}(x, t),
\]
где $\gamma$ --- вещественный параметр. При переходе к предельному при $\gamma \to +\infty$ уравнению
\begin{equation}
	\label{eq:intro:equiv_equation_initial}
	\dot{x} = \lim\limits_{\gamma \to +\infty}f_{\gamma}(x, t) = f(x, t),
\end{equation}
правая часть может стать разрывной функцией. При этом возникает необходимость обобщить понятие решения уравнения так, чтобы оно удовлетворяло следующим естественным требованиям.
\begin{enumerate}
	\item Для дифференциальных уравнений с непрерывной правой частью определение решения должно быть равносильно обычному.
	\item Для уравнения $\dot{x} = f(t)$ решениями (в обобщённом смысле) должны быть функции $x(t) = \int f(t)\, dt + c$ (и только они).
\end{enumerate}
Суть большинства из известных методов решения таких уравнений состоит в следующем. Пусть дано уравнение \eqref{eq:intro:equiv_equation_initial} где функция $f$ кусочно непрерывна в области $G \subset \mathbb{R}^n \times \mathbb{R}$, $x \in \mathbb{R}^n$, $M$ --- множество точек разрыва функции $f$. Для каждой точки $(x, t) \in G$ указывается множество $\mathcal{F}(x, t) \subset \mathbb{R}^n$. Если в точке $(x, t)$ функция $f$ непрерывна, то $\mathcal{F}(x, t)$ состоит в точности из этой точки. Если же $f$ разрывна, то множество $\mathcal{F}(x, t)$ задаётся некоторым образом. Тогда решением уравнения \eqref{eq:intro:equiv_equation_initial} на отрезке $I$ называется непрерывная функция $x(t)$, определённая на этом интервале, для которой почти всюду на $I$ выполнено $\dot{x}(t) \in \mathcal{F}(t, x)$.

Наиболее известные определения излагаются в книге \cite[\S 4]{Filippov1988}.

В третьей части диссертации используется доопределение методом эквивалентного управления \cite{Utkin1981}, описание которого приводится там же.

\bigskip

{\aim} данной работы является исследование периодических режимов в полносвязной сети релейных осцилляторов Мэки--Гласса.

Для~достижения поставленной цели необходимо было решить следующие {\tasks}.
\begin{enumerate}[beginpenalty=10000] % https://tex.stackexchange.com/a/476052/104425
	\item Описать достаточные условия, при которых уравнение Мэки--Гласса \eqref{eq:mg_equation_1:intro} имеет периодическое решение.
	\item Исследовать полносвязную систему релейных осцилляторов Мэки--Гласса, описать условия и ограничения на параметры системы, при которых она имеет решение:
	\begin{enumerate}
		\item[а)]в виде дискретной бегущей волны,
		\item[б)]в виде, соответствующем режиму двухкластерной синхронизации.
	\end{enumerate}
\end{enumerate}

\bigskip

{\novelty}
\begin{enumerate}[beginpenalty=10000] % https://tex.stackexchange.com/a/476052/104425
  \item Впервые получены асимптотические формулы решения уравнения Мэки--Гласса \eqref{eq:intro:MG_norm1} по параметру $\gamma \gg 1$ и доказано существование периодических решений при ограничении на параметры $\alpha > \exp\left(\beta(1 + e^{-\beta})\right)$.
  \item Впервые доказано существование периодических режимов в виде дискретной бегущей волны в полносвязной сети релейных осцилляторов Мэки--Гласса, а также сформулированы и доказаны условия их существования в виде ограничения на параметры соответствующей системы дифференциальных уравнений с запаздыванием.
  \item Впервые доказано существование периодических режимов двухкластерной синхронизации в полносвязной сети релейных осцилляторов Мэки--Гласса, а также сформулированы и доказаны условия их существования в виде ограничения на параметры соответствующей системы дифференциальных уравнений с запаздыванием.
  %TODO: сказать про скользящие траектории.
\end{enumerate}

\bigskip

{\influence} Научная работа носит теоретический характер, в ней были применены методы нелинейного анализа динамических систем в бесконечномерном фазовом пространстве. Теоретическая ценность работы определяется тем, что асимптотический метод большого параметра, а также подходы к построению дискретных бегущих волн и режимов кластерной синхронизации были адаптированы для применения к уравнению Мэки--Гласса и составленной на его основе полносвязной релейной системы из $N$ дифференциальных уравнений с запаздыванием (соответственно).

Полученные в диссертации результаты могут стать основой для дальнейших исследований в области нелинейной динамики и нелинейного функционального анализа, быть использованы специалистами для решения широкого спектра научных и прикладных задач. Они могут также применяться при разработке и чтении курсов и спецкурсов по теории дифференциальных уравнений с запаздыванием и методам их исследования.

Работа выполнена в рамках программы развития Регионального научно-образовательного математического центра Ярославского государственного университета им.~П.Г.~Демидова при финансовой поддержке Министерства науки и высшего образования Российской Федерации (Соглашение о предоставлении субсидии из федерального бюджета № 075-02-2025-1636).

\bigskip

{\defpositions}
\begin{enumerate}[beginpenalty=10000] % https://tex.stackexchange.com/a/476052/104425
	\item Получены асимптотические формулы периодического решения уравнения Мэки--Гласса \eqref{eq:intro:MG_norm1} по параметру $\gamma \gg 1$ при ограничении на параметры $\alpha > \exp\left(\beta(1 + e^{-\beta})\right)$ \cite[Теорема~5.6]{wosbib1}. На основе полученных формул доказана теорема о существовании периодического решения уравнения Мэки--Гласса \cite[Теорема~3.2]{wosbib1}.
	\item Доказано существование периодических режимов в виде дискретной бегущей волны в полносвязной сети релейных осцилляторов Мэки--Гласса, сформулированы и доказаны достаточные условия их существования в виде ограничения на параметры соответствующей системы дифференциальных уравнений с запаздыванием \cite[Теорема~16]{wosbib2}.
	\item Доказана теорема о существовании (в смысле обобщённого решения системы дифференциальных уравнений с разрывной правой частью) периодических режимов двухкластерной синхронизации в полносвязной  релейных осцилляторов Мэки--Гласса, сформулированы и доказаны достаточные условия их существования в виде ограничения на параметры соответствующей системы дифференциальных уравнений с запаздыванием \cite[Теорема~5.2]{scbib1}.
\end{enumerate}

\bigskip

\textbf{Список работ, выносимый на защиту:}
\begin{enumerate}[beginpenalty=10000] % https://tex.stackexchange.com/a/476052/104425
	\item Анализ асимптотической сходимости периодического решения уравнения Мэки–-Гласса к решению предельного релейного уравнения / В.~В.~Алексеев, М.~М.~Преображенская // Теоретическая и математическая физика. --- 2024. --- Т. 220, № 2. --- С. 213--236. \cite{wosbib1}
	\item Existence of Discrete Traveling Waves in Fully Coupled Network of Mackey--Glass Relay Generators / V.~Alekseev, M.~Preobrazhenskaia, V.~Vorontsova // Differential Equations. --- 2024. --- Vol. 60, No 9. --- P.~1217--1231 \cite{wosbib2}
	\item Two-cluster synchronization on a fully coupled network of Mackey--Glass generators // V.~Alekseev // Partial Differential Equations in Applied Mathematics. --- 2024. --- Vol. 12. --- P. 100930. \cite{scbib1}
	%TODO: сказать про скользящие траектории.
\end{enumerate}


%В папке Documents можно ознакомиться с решением совета из Томского~ГУ
%(в~файле \verb+Def_positions.pdf+), где обоснованно даются рекомендации
%по~формулировкам защищаемых положений.

\bigskip

{\reliability} полученных результатов обеспечивается строгими математическими доказательствами, приведёнными в работе. % Результаты находятся в соответствии с результатами, полученными другими авторами.

\nocite{scbib1, wosbib1, wosbib2}

\bigskip

{\probation}
Основные результаты работы докладывались~на следующих конференциях и семинарах.
\begin{enumerate}
	\item Семинар кафедры <<Функциональный анализ и его приложения>> Владимирского государственного университета им.~А.~Г.~и~Н.~Г.~Столетовых, 13 февраля 2025 года.
	\item Семинар по качественной теории дифференциальных уравнений в Московском государственном университете им.~М.~В.~Ломоносова, 29~ноября 2024~года. \cite{Sergeev2024},\\\texttt{https://www.elibrary.ru/item.asp?id=75144298}
	\item Научный семинар лаборатории динамических систем и приложений НИУ ВШЭ в Нижнем Новгороде, 25 сентября 2024 г.,\\\texttt{https://nnov.hse.ru/bipm/dsa/semtmd}.
	\item Семинар по нелинейной динамике Ярославского государственного университета им.~П.~Г.~Демидова, 19 сентября 2024 г.,\\\texttt{https://cis.uniyar.ac.ru/index.php/event/460}.
	\item Конференция <<Integrable Systems and Nonlinear Dynamics>> (ISND – 2024), Ярославль, 2024 \cite{confbib5}.
	\item Конференция <<Topological Methods in Dynamics and Related Topics VII>>, Нижний Новгород, 2024 \cite{confbib6}.
	\item Международная конференция по дифференциальным уравнениям и динамическим системам DIFF-2024, Суздаль, 2024 \cite{confbib3}.
	\item Конференция <<Нелинейные дни в Саратове для молодых>>, Саратов, 2023 \cite{confbib2}.
	\item Конференция <<Satellite International Conference on Nonlinear Dynamics {\&} Integrability>>, Ярославль, 2022 \cite{confbib4}.
	\item Международная конференция по дифференциальным уравнениям и динамическим системам DIFF--2022, Суздаль, 2022 \cite{confbib1}.
\end{enumerate}

%{\contribution} Автор принимал активное участие \ldots

% \vspace{-11em}

\bigskip

{\publications} Основные результаты по теме диссертации изложены в 9 печатных изданиях, 3 из которых \cite{wosbib1,wosbib2,scbib1} изданы в журналах, рекомендованных ВАК, 3 "--- в~периодических научных журналах, индексируемых Web of~Science или Scopus \cite{wosbib1,wosbib2,scbib1}, 6 "--- в~тезисах докладов \cite{confbib1,confbib2,confbib3,confbib4,confbib5,confbib6}. 

{\contribution} Результаты из разделов 1.1 и 1.2 первой главы (постановка задачи и анализ релейного уравнения) получены в соавторстве с М.~М.~Преображенской, они соответствуют разделам 1--4 работы \cite{wosbib1}. Авторство постановки задачи во второй главе диссертации (раздел 2.1) принадлежит М.~М.~Преображенской, см. раздел 2 работы \cite{wosbib2}. Остальные результаты, представленные в диссертации, получены автором лично.
%Авторство результатов из подраздела 2.2.2 принадлежит В.~К.~Воронцовой (Зеленовой), см. лемму 3 и вывод формул (24), (27), (30), (31) в работе \cite{wosbib2}.

Список опубликованных в рецензируемых журналах работ по теме диссертации.
\begin{enumerate}
	\item Анализ асимптотической сходимости периодического решения уравнения Мэки–-Гласса к решению предельного релейного уравнения / В.~В.~Алексеев, М.~М.~Преображенская // Теоретическая и математическая физика. --- 2024. --- Т. 220, № 2. --- С. 213--236. \cite{wosbib1}
	\item Existence of Discrete Traveling Waves in Fully Coupled Network of Mackey--Glass Relay Generators / V.~Alekseev, M.~Preobrazhenskaia, V.~Vorontsova // Differential Equations. --- 2024. --- Vol. 60, No 9. --- P.~1217--1231 \cite{wosbib2}
	\item Two-cluster synchronization on a fully coupled network of Mackey--Glass generators // V.~Alekseev // Partial Differential Equations in Applied Mathematics. --- 2024. --- Vol. 12. --- P. 100930. \cite{scbib1}
	\item О семинаре по качественной теории дифференциальных уравнений в Московском государственном университете им.~М.~В.~Ломоносова (хроника). // Дифференциальные уравнения. --- 2024. --- Т.~60, №~11, С.~1580--1582. \cite{Sergeev2024}
\end{enumerate}

\medskip

Список опубликованных работ по теме диссертации в сборниках тезисов конференций и семинаров.
\begin{enumerate}
	\item Two-cluster Synchronization in a Fully Coupled Network of	Mackey--Glass Generators / V.~V.~Alekseev // Topological Methods in Dynamics and Related Topics. --- 2024. --- P. 9--10. \cite{confbib6}
	\item Two-cluster synchronisation in a fully coupled network of	Mackey--Glass generators / V.~V.~Alekseev // Integrable Systems and Nonlinear Dynamics (ISND -- 2024). --- 2024. --- P. 9--10. \cite{confbib5}
	\item Анализ асимптотической сходимости периодического решения уравнения Мэки-Гласса к решению предельного релейного уравнения	/ В.~Алексеев, М.~Преображенская // Сборник тезисов международной конференции и международной школы молодых учёных (Суздаль). --- 2024. --- С. 86. \cite{confbib3}
	\item Существование и исследование устойчивости решений в форме дискретной бегущей волны в полносвязной цепи генераторов	Мэки-Гласса / В.~В.~Алексеев, М.~М.~Преображенская, В.~К.~Зеленова // Нелинейные дни в Саратове для молодых --- 2023. Сборник научных трудов конференции. --- 2023. --- С. 63--64. \cite{confbib2}
	\item Existence of discrete traveling waves in a fully connected relay system of Mackey–Glass type equations / V.~Alekseev, M.~Preobrazhenskaia, V.~Zelenova // Satellite International Conference on Nonlinear Dynamics and Integrability and Scientific School "Nonlinear Days". --- 2022. --- P. 14--15. \cite{confbib4}
	\item Существование дискретных бегущих волн в полносвязной цепи релейных генераторов Мэки-Гласса / В.~В.~Алексеев, М.~М.~Преображенская, В.~К.~Зеленова // Сборник тезисов международной конференции и международной школы молодых учёных (Суздаль). --- 2022. --- С. 81. \cite{confbib1}
\end{enumerate}

%
%%%% Реализация пакетом biblatex через движок biber
%\begin{refsection}[bl-author, bl-registered]
%    % Это refsection=1.
%    % Процитированные здесь работы:
%    %  * подсчитываются, для автоматического составления фразы "Основные результаты ..."
%    %  * попадают в авторскую библиографию, при usefootcite==0 и стиле `\insertbiblioauthor` или `\insertbiblioauthorgrouped`
%    %  * нумеруются там в зависимости от порядка команд `\printbibliography` в этом разделе.
%    %  * при использовании `\insertbiblioauthorgrouped`, порядок команд `\printbibliography` в нём должен быть тем же (см. biblio/biblatex.tex)
%    %
%    % Невидимый библиографический список для подсчёта количества публикаций:
%    \printbibliography[heading=nobibheading, section=1, env=countauthorvak,          keyword=biblioauthorvak]%
%    \printbibliography[heading=nobibheading, section=1, env=countauthorwos,          keyword=biblioauthorwos]%
%    \printbibliography[heading=nobibheading, section=1, env=countauthorscopus,       keyword=biblioauthorscopus]%
%    \printbibliography[heading=nobibheading, section=1, env=countauthorconf,         keyword=biblioauthorconf]%
%    \printbibliography[heading=nobibheading, section=1, env=countauthorother,        keyword=biblioauthorother]%
%%    \printbibliography[heading=nobibheading, section=1, env=countregistered,         keyword=biblioregistered]%
%%    \printbibliography[heading=nobibheading, section=1, env=countauthorpatent,       keyword=biblioauthorpatent]%
%    \printbibliography[heading=nobibheading, section=1, env=countauthorprogram,      keyword=biblioauthorprogram]%
%    \printbibliography[heading=nobibheading, section=1, env=countauthor,             keyword=biblioauthor]%
%    \printbibliography[heading=nobibheading, section=1, env=countauthorvakscopuswos, filter=vakscopuswos]%
%    \printbibliography[heading=nobibheading, section=1, env=countauthorscopuswos,    filter=scopuswos]%
%    %
%    \nocite{*}%
%    %
%    {\publications} Основные результаты по теме диссертации изложены в~\arabic{citeauthor}~печатных изданиях,
%    \arabic{citeauthorvak} из которых \cite{wosbib1,wosbib2,scbib1} изданы в журналах, рекомендованных ВАК\sloppy%
%    \ifnum \value{citeauthorscopuswos}>0%
%        , \arabic{citeauthorscopuswos} "--- в~периодических научных журналах, индексируемых Web of~Science и Scopus \cite{wosbib1,wosbib2,scbib1}\sloppy%
%    \fi%
%    \ifnum \value{citeauthorconf}>0%
%        , \arabic{citeauthorconf} "--- в~тезисах докладов \cite{confbib1,confbib2,confbib3,confbib4,confbib5}.
%    \else%
%        .
%    \fi%
%    \ifnum \value{citeregistered}=1%
%        \ifnum \value{citeauthorpatent}=1%
%            Зарегистрирован \arabic{citeauthorpatent} патент.
%        \fi%
%        \ifnum \value{citeauthorprogram}=1%
%            Зарегистрирована \arabic{citeauthorprogram} программа для ЭВМ.
%        \fi%
%    \fi%
%    % К публикациям, в которых излагаются основные научные результаты диссертации на соискание учёной
%    % степени, в рецензируемых изданиях приравниваются патенты на изобретения, патенты (свидетельства) на
%    % полезную модель, патенты на промышленный образец, патенты на селекционные достижения, свидетельства
%    % на программу для электронных вычислительных машин, базу данных, топологию интегральных микросхем,
%    % зарегистрированные в установленном порядке.(в ред. Постановления Правительства РФ от 21.04.2016 N 335)
%\end{refsection}%
\begin{refsection}[bl-author, bl-registered]
    % Это refsection=2.
    % Процитированные здесь работы:
    %  * попадают в авторскую библиографию, при usefootcite==0 и стиле `\insertbiblioauthorimportant`.
    %  * ни на что не влияют в противном случае
    \nocite{vakbib2}%vak
    \nocite{patbib1}%patent
    \nocite{progbib1}%program
    \nocite{bib1}%other
    \nocite{confbib1}%conf
\end{refsection}%
    %
    % Всё, что вне этих двух refsection, это refsection=0,
    %  * для диссертации - это нормальные ссылки, попадающие в обычную библиографию
    %  * для автореферата:
    %     * при usefootcite==0, ссылка корректно сработает только для источника из `external.bib`. Для своих работ --- напечатает "[0]" (и даже Warning не вылезет).
    %     * при usefootcite==1, ссылка сработает нормально. В авторской библиографии будут только процитированные в refsection=0 работы.

%При использовании пакета \verb!biblatex! будут подсчитаны все работы, добавленные
%в файл \verb!biblio/author.bib!. Для правильного подсчёта работ в~различных
%системах цитирования требуется использовать поля:
%\begin{itemize}
%        \item \texttt{authorvak} если публикация индексирована ВАК,
%        \item \texttt{authorscopus} если публикация индексирована Scopus,
%        \item \texttt{authorwos} если публикация индексирована Web of Science,
%        \item \texttt{authorconf} для докладов конференций,
%        \item \texttt{authorpatent} для патентов,
%        \item \texttt{authorprogram} для зарегистрированных программ для ЭВМ,
%        \item \texttt{authorother} для других публикаций.
%\end{itemize}
%Для подсчёта используются счётчики:
%\begin{itemize}
%        \item \texttt{citeauthorvak} для работ, индексируемых ВАК,
%        \item \texttt{citeauthorscopus} для работ, индексируемых Scopus,
%        \item \texttt{citeauthorwos} для работ, индексируемых Web of Science,
%        \item \texttt{citeauthorvakscopuswos} для работ, индексируемых одной из трёх баз,
%        \item \texttt{citeauthorscopuswos} для работ, индексируемых Scopus или Web of~Science,
%        \item \texttt{citeauthorconf} для докладов на конференциях,
%        \item \texttt{citeauthorother} для остальных работ,
%        \item \texttt{citeauthorpatent} для патентов,
%        \item \texttt{citeauthorprogram} для зарегистрированных программ для ЭВМ,
%        \item \texttt{citeauthor} для суммарного количества работ.
%\end{itemize}
%% Счётчик \texttt{citeexternal} используется для подсчёта процитированных публикаций;
%% \texttt{citeregistered} "--- для подсчёта суммарного количества патентов и программ для ЭВМ.
%
%Для добавления в список публикаций автора работ, которые не были процитированы в
%автореферате, требуется их~перечислить с использованием команды \verb!\nocite! в
%\verb!Synopsis/content.tex!.


