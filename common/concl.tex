%% Согласно ГОСТ Р 7.0.11-2011:
%% 5.3.3 В заключении диссертации излагают итоги выполненного исследования, рекомендации, перспективы дальнейшей разработки темы.
%% 9.2.3 В заключении автореферата диссертации излагают итоги данного исследования, рекомендации и перспективы дальнейшей разработки темы.

\begin{enumerate}
  \item Получены асимптотические формулы периодического решения уравнения Мэки--Гласса для достаточно больших значений коэффициента нелинейности. Показано, что из асимптотических соотношений следует сходимость решения уравнения Мэки--Гласса к решению соответствующего предельного релейного уравнения.
  \item Сформулированы и доказаны достаточные условия существования периодических режимов (а) в виде дискретной бегущей волны, (б) двухкластерной синхронизации в полносвязной сети релейных генераторов Мэки--Гласса в виде ограничения на параметры соответствующей системы дифференциальных уравнений с запаздыванием. 
  \item Проведено численное моделирование полносвязной сети релейных генераторов Мэки--Гласса, позволяющее сделать вывод об устойчивости режимов дискретных бегущих волн и кластерной синхронизации.
\end{enumerate}
