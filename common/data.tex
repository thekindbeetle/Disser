%%% Основные сведения %%%
\newcommand{\thesisAuthorLastName}{Алексеев}
\newcommand{\thesisAuthorOtherNames}{Владислав Владимирович}
\newcommand{\thesisAuthorInitials}{В.\,В.}
\newcommand{\thesisAuthor}             % Диссертация, ФИО автора
{%
    \texorpdfstring{% \texorpdfstring takes two arguments and uses the first for (La)TeX and the second for pdf
        \thesisAuthorLastName~\thesisAuthorOtherNames% так будет отображаться на титульном листе или в тексте, где будет использоваться переменная
    }{%
        \thesisAuthorLastName, \thesisAuthorOtherNames% эта запись для свойств pdf-файла. В таком виде, если pdf будет обработан программами для сбора библиографических сведений, будет правильно представлена фамилия.
    }
}
\newcommand{\thesisAuthorShort}        % Диссертация, ФИО автора инициалами
{\thesisAuthorInitials~\thesisAuthorLastName}
%\newcommand{\thesisUdk}                % Диссертация, УДК
%{\fixme{xxx.xxx}}
\newcommand{\thesisTitle}              % Диссертация, название
{Режимы кластерной синхронизации и дискретных бегущих волн в~полносвязной сети осцилляторов Мэки--Гласса}
\newcommand{\thesisSpecialtyNumber}    % Диссертация, специальность, номер
{1.1.2}
\newcommand{\thesisSpecialtyTitle}     % Диссертация, специальность, название (название взято с сайта ВАК для примера)
{Дифференциальные уравнения и математическая физика}
%% \newcommand{\thesisSpecialtyTwoNumber} % Диссертация, вторая специальность, номер
%% {\fixme{XX.XX.XX}}
%% \newcommand{\thesisSpecialtyTwoTitle}  % Диссертация, вторая специальность, название
%% {\fixme{Теория и~методика физического воспитания, спортивной тренировки,
%% оздоровительной и~адаптивной физической культуры}}
\newcommand{\thesisDegree}             % Диссертация, ученая степень
{кандидата физико-математических наук}
\newcommand{\thesisDegreeShort}        % Диссертация, ученая степень, краткая запись
{канд. физ.-мат. наук}
\newcommand{\thesisCity}               % Диссертация, город написания диссертации
{Ярославль}
\newcommand{\thesisYear}               % Диссертация, год написания диссертации
{2025}
\newcommand{\thesisOrganization}       % Диссертация, организация
{Федеральное государственное бюджетное образовательное учреждение высшего образования\linebreak<<Ярославский государственный университет им. П.Г. Демидова>>}
\newcommand{\thesisOrganizationShort}  % Диссертация, краткое название организации для доклада
{Периодические режимы в полносвязной системе уравнений Мэки--Гласса}

\newcommand{\thesisInOrganization}     % Диссертация, организация в предложном падеже: Работа выполнена в ...
{Федеральном государственном бюджетном образовательном учреждении высшего образования <<Ярославский государственный университет им. П.Г. Демидова>>}

%% \newcommand{\supervisorDead}{}           % Рисовать рамку вокруг фамилии
\newcommand{\supervisorFio}              % Научный руководитель, ФИО
{Преображенская Маргарита Михайловна}
\newcommand{\supervisorRegalia}          % Научный руководитель, регалии
{кандидат физ.-мат. наук}
\newcommand{\supervisorFioShort}         % Научный руководитель, ФИО
{М.\,М.~Преображенская}
\newcommand{\supervisorRegaliaShort}     % Научный руководитель, регалии
{к.~ф.-м.~н.}

\newcommand{\opponentOneFio}           % Оппонент 1, ФИО
{Починка Ольга Витальевна}
\newcommand{\opponentOneRegalia}       % Оппонент 1, регалии
{доктор физико-математических наук, профессор}
\newcommand{\opponentOneJobPlace}      % Оппонент 1, место работы
{НИУ ВШЭ в Нижнем Новгороде}
\newcommand{\opponentOneJobPost}       % Оппонент 1, должность
{главный научный сотрудник}

\newcommand{\opponentTwoFio}           % Оппонент 2, ФИО
{Родина Людмила Ивановна}
\newcommand{\opponentTwoRegalia}       % Оппонент 2, регалии
{доктор физико-математических наук, профессор}
\newcommand{\opponentTwoJobPlace}      % Оппонент 2, место работы
{Владимирский государственный университет}
\newcommand{\opponentTwoJobPost}       % Оппонент 2, должность
{профессор}

\newcommand{\opponentThreeFio}           % Оппонент 3, ФИО
{Сергеев Игорь Николаевич}                         
\newcommand{\opponentThreeRegalia}       % Оппонент 3, регалии
{доктор физико-математических наук, профессор}    
\newcommand{\opponentThreeJobPlace}      % Оппонент 3, место работы
{Московский государственный университет}        
\newcommand{\opponentThreeJobPost}       % Оппонент 3, должность
{профессор}

\newcommand{\opponentFourFio}           % Оппонент 4, ФИО
{Соколов Сергей Викторович}                         
\newcommand{\opponentFourRegalia}       % Оппонент 4, регалии
{доктор физико-математических наук}     
\newcommand{\opponentFourJobPlace}      % Оппонент 4, место работы
{Московский физико-технический институт}           
\newcommand{\opponentFourJobPost}       % Оппонент 4, должность
{зав. кафедрой теоретической механики}

\newcommand{\opponentFiveFio}           % Оппонент 5, ФИО
{Казаков Алексей Олегович}                         
\newcommand{\opponentFiveRegalia}       % Оппонент 5, регалии
{доктор физико-математических наук, профессор}     
\newcommand{\opponentFiveJobPlace}      % Оппонент 5, место работы
{НИУ ВШЭ в Нижнем Новгороде}           
\newcommand{\opponentFiveJobPost}       % Оппонент 5, должность
{главный научный сотрудник}

%\newcommand{\leadingOrganizationTitle} % Ведущая организация, дополнительные строки. Удалить, чтобы не отображать в автореферате
%{Федеральное государственное автономное образовательное учреждение высшего образования <<Национальный исследовательский университет <<Высшая школа экономики>>}

\newcommand{\defenseDate}              % Защита, дата
{\fixme{DD mmmmmmmm YYYY~г.~в~XX часов}}
\newcommand{\defenseCouncilNumber}     % Защита, номер диссертационного совета
{\fixme{Д\,123.456.78}}
\newcommand{\defenseCouncilTitle}      % Защита, учреждение диссертационного совета
{\fixme{Название учреждения}}
\newcommand{\defenseCouncilAddress}    % Защита, адрес учреждение диссертационного совета
{\fixme{Адрес}}
\newcommand{\defenseCouncilPhone}      % Телефон для справок
{\fixme{+7~(0000)~00-00-00}}

\newcommand{\defenseSecretaryFio}      % Секретарь диссертационного совета, ФИО
{\fixme{Фамилия Имя Отчество}}
\newcommand{\defenseSecretaryRegalia}  % Секретарь диссертационного совета, регалии
{\fixme{д-р~физ.-мат. наук}}            % Для сокращений есть ГОСТы, например: ГОСТ Р 7.0.12-2011 + http://base.garant.ru/179724/#block_30000

\newcommand{\synopsisLibrary}          % Автореферат, название библиотеки
{\fixme{Название библиотеки}}
\newcommand{\synopsisDate}             % Автореферат, дата рассылки
{\fixme{DD mmmmmmmm}\the\year~года}

% To avoid conflict with beamer class use \providecommand
\providecommand{\keywords}%            % Ключевые слова для метаданных PDF диссертации и автореферата
{уравнение Мэки--Гласса, дифференциальные уравнения с запаздываниями, системы дифференциальных уравнений, динамические системы}
