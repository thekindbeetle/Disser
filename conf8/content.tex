Дистрибуція — це діяльність, пов'язана з отриманням продукції, її зберіганням до моменту отримання замовлення і наступної доставки до клієнтів. Управління дистрибуцією включає в себе планування, організацію та контроль. Для управління доставкою вантажів у дистрибуції вкрай важливим є етап моніторингу руху автомобілів у режимі реального часу. Це дозволяє аналізувати ефективність водія, а також передбачати певні небажані інциденти. Для такого моніторингу використовують GPS дані руху автомобіля \cite{Gonzalez_2013}. На жаль лише GPS треку не достатньо для однозначного розуміння стану справ. Для отримання докладнішої інформації необхідна додаткова комунікація водія з диспетчером. Тому потрібна система, яка б дала змогу виявляти необхідну інформацію в голосових даних водія і відправляти її диспетчеру у формалізованому вигляді.

Найбільш подібна до цього система Pick-by-Voice \cite{Pick-to-Voice}. Це система, що використовується в іншій сфері управління дистрибуцією — управлінні складськими процесами. Pick-by-Voice дає змогу відбірнику по черзі отримувати голосові команди у вигляді: де, що і в якій кількості треба відібрати, а також у формі діалогу повідомляти про необхідність повторити завдання чи переходити до наступного. На жаль для управління транспортними доставками потрібна більш складна система, ніж наявні можливості Pick-by-Voice.

Сучасні системи розпізнання мови в більшій своїй частині засновані на статистичних методах, використовують потужній апарат теорії ймовірностей та математичної статистики, що дає змогу суттєво підвищити якість розпізнання. Основні методи розпізнання мови — це приховані Марківські моделі та штучні нейронні мережі \cite{Makovkin_2006}. Існує досить багато таких систем і вони доволі якісно виконують свою задачу. Але в більшості своїй ці системи розраховані на роботу в приміщеннях без сильних шумів, залучення дикторів з чіткою вимовою та використання потужних комп’ютерів або віддалених серверів, як, наприклад, Google Voice Search. Для дистрибуціє такі недоліки критичні, адже немає можливості а ні встановити у кабіні водія потужне обладнання, а ні забезпечити стабільний та швидкісний доступ до Інтернету. Проблема багатодикторності також актуальна для дистрибуції, адже у таких компаніях зазвичай працює від декількох десятків до кількох сотень. 

Японський дослідник з університету Осаки Ішігуро Хіроші з колегами вивчали різні аспекти комунікації та інформаційно-комунікаційних технологій. Зокрема він проводив дослідження голосової комунікації двох людей опосередковано через комп’ютер \cite{Ishiguro_2016}. У цьому дослідженні пара спілкувалася на загальні теми, обираючи варіанти своєї репліки із заздалегідь написаного дерева варіантів, свого роду сценарію. Жодному з партнерів не потрібно було нічого промовляти вголос: людині надавався набір з варіантів реплік на вибір, потрібно було лише натиснути на ту з них, яку б вона хотіла промовити, і ця репліка лунала з динаміків. У залежності від використаної репліки програма вибирала з дерева сценаріїв можливі варіанти відповідей і надавала їх на вибір співрозмовнику. Співрозмовник у свою чергу, чуючи репліку першої людини, обирав свою з наданих варіантів.

На жаль для управління дистрибуцією постає зворотне завдання — водій має повідомити певну інформацію в систему і при цьому не повинен відволікатися на натискання кнопок на екрані. Тому пряме використання такої технології неможливе. Але застосування підходу описання всіх можливих сценаріїв комунікації в залежності від контексту дозволить знизити кількість інформації, яку треба розпізнати, а отже і підвищити якість.

Наразi існує новий підхід до голосового управління, заснований на теорії несилової взаємодії \cite{Teslia_2010} — рефлекторна система голосового управління. Ідея, покладена в основу цього підходу, полягає в тому, щоб замість переведення голосової інформації в текстову репрезентацію, аналізувати безпосередньо інформаційну складову сказаного, визначаючи, яку з відомих реакцій потрібно виконати. Оскільки в такій системі не потрібні словники, складні інтелектуальні моделі аналізу тексту та граматики, вони мають низку переваг порівняно з традиційними, зокрема багатодикторність та можливість роботи офлайн \cite{Teslia_2013}.

У результаті аналізу виявлено два найбільш перспективні напрями, поєднання яких дає змогу запропонувати нове принципове рішення і побудувати рефлекторну модель голосової взаємодії в задачах управління дистрибуцією. В основу моделі покладено логічні сценарії взаємодії на тему управління дистрибуцією, які мають враховувати параметри основних причин невідповідності реальної ситуації запланованому маршруту, наприклад, запізнення або відмови обслуговування на точці доставки тощо. 